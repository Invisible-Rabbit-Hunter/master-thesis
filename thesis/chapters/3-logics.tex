\documentclass[../thesis.tex]{subfiles}

\ifSubfilesClassLoaded{%
    \externaldocument{../build/2-preliminaries}%
    % \IfFileExists{references.bib}{}{\addbibresource{./references.bib}}
  }{}

  

\begin{document}
\chapter{Finite \texorpdfstring{$\cF$}{F}-limit theories}\label{ch:F-theories}
In this chapter we introduce the notion of theories and models that underlie our treatment of
the semantics of type theory in the subsequent chapter. Since the models of our theories ought
to be categories equipped with some structure, it is sensible take take a 2-dimensional approach.
The main notion underlying our treatment will be that of $\sF$-categories (\cref{def:F-category}),
as these will allow fine-grained control of the strictness of homomorphisms.

There are several possible notions in the literature that would apply, notably the various 2-dimensional
generalizations of Lawvere theories as found in \cite{gray1973, nishizawa2009}. However,
as weill see later, we will generally require more 2-limits than simply finite products and cotensor.
In particular, we will make extensive use of 2-pullbacks, comma objects, and equalizers. Rather than
Lawvere theories, we thus want to consider 2-dimensional limit theories, such as the enriched limit
theories of \cite{kelly1982a} in the special case of enrichement in $\Cat$. A useful consequence of
using enriched limit theories, that would also apply to Lawvere theories, is that we can present
the theories using \emph{limit sketches}, and from that generate the theory we expect. 

\begin{remark}
  One issue that arises when considering enriched limit theories, however, is that generally the
  homomorphisms will be either too strict or too weak. If we take the notion of homomorphisms to
  be what falls out directly from the notion of enriched limit theories as presented in \cite{kelly1982a},
  enriched natural transformations, we find that the homomorphisms are too strict.

  Consider, for example, a theory for categories with finite products. In this case, the models are
  categories equipped with a choice of finite products, and homomorphisms are 2-natural transformations
  strictly preserving them. This suggests that we ought to use pseudonatural transformations, rather than
  strict 2-natural transformations, as homomorphisms, as then the homomorphisms are functors 
  preserving products in the traditional sense.

  However, now we instead suffer from the opposite problem: in the theory of functors, whose models are
  pairs of categories with a functor between them, we would expect the morphisms to be commuting squares
  of functors, but with pseudonatural transformations they would only be squares commuting up to isomorphism.
\end{remark}

Given the issues of having the right strictness of morphisms when considering $\Cat$-enriched finite
limit theories, we will instead be following \cite{arkor2024}, where they replace $\Cat$ with another
category $\sF$ (essentially) consisting of categories equipped with full subcategories, and then take
homomorphisms of models to allow $\sF$-enriched finite limit theories to specify strictness.
\begin{definition}[{\cite[§3.1]{lack2012}}]\label{def:F}
  We define $\sF$ to be the full subcategory of $\Cat^\to$, the arrow category of $\Cat$, on fully faithful and
  injective-on-objects functors. Given an object $A_\tau \hookrightarrow A_\lambda$ of $\sF$, we call $A_\tau$ the
  tight part (with the object of $A_\tau$ being tight object), and we call $A_\lambda$ the loose part (with the objects
  of $A_\lambda$ being loose objects). We will generally identify $A_\tau$ with its image in $A_\lambda$, thereby
  viewing $A_\tau$ as a class of objects of $A_\lambda$.
\end{definition}
\begin{remark}
  $\sF$-categories were first introduced in \cite{lack2012}, where they were used to study what 2-limits
  lifted to different 2-categories of algebras for 2-monads.
\end{remark}
To talk about enrichment in $\sF$, as defined in \cref{sec:enriched}, we require that $\sF$ is cartesian closed, which is
establshed in \cite[§3.1]{lack2012}. Unfolding the definition of $\sF$-enriched categories, we see that, since an object
of $\sF$ is essentilly a category equipped with a class of objects, an $\sF$-enriched category is essentially the same
as a 2-category equipped with a class of morphisms closed under identites and composition, and these morphisms we
will denote as tight. It is precisely this additional data that allows us to specify that homomorphisms should preserve
only some maps strictly, and the rest up to isomorphism: we take homomorphisms to be pseudonatural transformations whose
naturality 2-cells are identities on tight morphisms. This approach therefore allows the specific theory precise control over
the strictness of morphisms. For example, in the theory of categories with finite products, we take the maps giving finite
products to be loose, in other words to not be tight, so that homomorphisms preserve these maps only up to isomorphism,
whereas in the theory of functors we take the one generating morphism to be tight, forcing homomorphisms to preserve it strictly.

% We denote by $\sF$ the full subcategory of $\Cat^\to$ on full embeddings, where a functor is a
% full embedding if it is fully faithful and injective on objects. Thus the objects of $\sF$ are
% full embeddings, and a morphism $A \to B$ in $\sF$ is a commuting squares
% \[\begin{codi}
%   \obj {A_\tau & B_\tau \\
%   A_\lambda & B_\lambda \\};
%
%   \mor A_tau -> B_tau ->,hook B_lambda;
%   \mor A_tau ->,hook A_lambda -> B_lambda;
% \end{codi}\]
% where $A_\tau$ denotes the domain of $A$ and $A_\lambda$ the codomain. 
% \begin{remark}
%   An object of $\sF$ is equivalently a category $A_\lambda$ equipped with a class $A_\tau$ of
%   objects. We call the objects of $A_\lambda$ loose and the objects in $A_\tau$ tight. Taking this
%   perspective, we find that a lot of structure will be uniqely determined by $A_\lambda$, with
%   $A_\tau$ placing restrictions on the structure. For example, a morphism in $\sF$ is simply a
%   functor $F : A_\lambda \to B_\lambda$ with the restriction that if $a \in A_\tau$, then $Fa \in
%   B_\tau$, and so $\sF$-morphisms are uniquely determined by their action on the loose part $A_\lambda$.
% \end{remark}
% \begin{corollary}
%   $\sF$ is a reflective subcategory of $\Cat^\to$.
% \end{corollary}
%
% \begin{corollary}
%   $\sF$ is cartesian closed.
% \end{corollary}
%

\begin{lemma}
  The functors $\bone_\lambda : \bzero \into \bone$ and $\bone_\tau : \bone \into \bone$ are the
  representables of $\sF$, in the sense that for any $A \in \sF$, we have $A_\tau \cong \sF(\bone_%
  \tau, A)$ and $A_\lambda
  \cong \sF(\bone_\lambda, A)$, naturally in $A$.
\end{lemma}

\begin{corollary}
  For any $A, B \in \sF$, the exponential $[A,B] \in \sF$ satisfies that $[A,B]_\tau \cong \sF(A,B)
  $ and $[A,B]_\lambda \cong \Cat(A_\lambda, B_\lambda)$, naturally in $A,B$.
\end{corollary}
\begin{proof}
  We have natural isomorphisms
  \[[A,B]_\tau \cong \sF(\bone_\tau, [A,B]) \cong \sF(\bone_\tau \times A, B) \cong \sF(A,B)\]
  and
  \[[A,B]_\lambda \cong \sF(\bone_\lambda, [A,B]) \cong \sF(\bone_\lambda \times A, B) \cong
  \Cat(A_\lambda, B_\lambda),\]
  where we have $(\bone_\lambda \times A) \cong (\bzero \into A_\lambda)$ since limits are computed
  pointwise, so that $\sF(\bone_\lambda \times A, B)$ is uniquely determined on the tight part,
  irrespective of the loose part, and so $\sF(\bone_\lambda \times A, B) \cong \Cat(A_\lambda, B_%
  \lambda)$.
\end{proof}

\begin{definition}[{\cite[§3.1]{lack2012}}]\label{def:F-category}
  An \emph{enhanced 2-category}, or $\sF$-category, is a category enriched in $\sF$.
\end{definition}
\begin{notation}
  We will use blackboard bold letters $\bA,\bB,\bC,\dots$ to denote $\sF$-categories. Furthermore,
  we use $A \loose B$ to denote loose morphisms of $\bA$, i.e.\ morphisms in $\bA_\lambda$, and
  $A \to B$ to denote tight morphisms, i.e\ morphisms in $\bA_\tau$.
\end{notation}
\begin{remark}
  Since an object of $\sF$ is equivalently a category equipped with a class of tight objects, we
  can  view an $\sF$-category as a 2-category equipped with a class of tight morphisms closed under
  identities (as the identity assigning functors preserve tightness and in the terminal object of
  $\sF$ all morphisms are tight) and composition (for similar reasons regarding the composition functor).
\end{remark}

Any $\sF$-category $\bA$ gives rise two 2-categories: the 2-category of tight morphisms $\bA_
\lambda$ and its locally full sub-2-category of tight morphisms $\bA_\tau$. We use $J_\bA$ to
denote the inclusion $\bA_\tau \embed \bA_\lambda$ and

We let $\bF$ denote the $\sF$-category of $\sF$-categories, as shown exists in \cref{ex:self enrichment}.
Its tight morphisms are $\sF$-functors and its loose morphisms are ordinary functors on the loose part.
The 2-cells are $\sF$-natural transformations, i.e. natural transformations on whose components are all
tight. Furthermore, given two $\sF$-categories $\bA,\bB$, we let $[\bA,\bB]$ denote the functor $\sF$-%
category as described in \cref{ex:enriched functor cateory}. Thus its objects are $\sF$-functors $\bA \to
\bB$, its tight morphisms are $\sF$-natural transformations, its loose morphisms are natural transformations
whose componenets need not be tight, and its 2-cells are modifications.

\begin{definition}[Chordate and inchordate $\sF$-categories]
  Any 2-category $K$ can be regarded as an $\sF$-category in two natural ways: taking only
  identities to be tight yields what we call the \emph{inchordate} $\sF$-category $K^-$, and 
  dually taking all morphisms to be tight yields the \emph{chordate} $\sF$-category $K^+$. The
  operations of taking a 2-category to its inchordate or chordate $\sF$-category extends to
  functors $2\Cat \to \sF\text-\Cat$, and these are respectively left and right adjoint to the
  functor $(-)_\lambda : \sF\text-\Cat \to 2\Cat$ taking a $\sF$-category to its underlying 2-%
  category of loose morphisms.
\end{definition}

\section{Limits in \texorpdfstring{$\sF$}{F}-categories}
Since we will be considering $\sF$-enriched finite limit theories, it will be helpful to have an
explicit description of $\sF$-enriched limits in a similar style to that of \cref{prop:chr of 2-limits}.
In fact, much of that description works out the same for $\sF$-limits, with tightness conditions
threaded throughout. As with the case of 2-limits, the usefulness of having such a characterisation
of $\sF$-limits will be instrumental in understanding what weights are correct for the notion we
want.

\begin{proposition}
  Let $\bJ$ be an $\sF$-category. A $\bJ$-shaped weight is an $\sF$-functor $W : \bJ \to \bF$, where
  $\bF$ is the $\sF$-category induced by the cartesian closed structure of $\sF$. Given a $\bJ$-shaped
  diagram $D : \bJ \to \bA$ in an $\sF$-category $\bA$, a loose $W$-weighted cone $(a,\alpha)$
  from $a \in \bA$ to $D$ is a loose natural transformation $W \loose \bA(a,D)$, i.e. a $W_\lambda$-%
  weighted cone for $D_\lambda$ in terms of 2-categories.

  If furthermore $\alpha_{x,r}$ (as given by \cref{prop:desc of 2-limits}) is a tight morphism of $\bA$
  for each tight objet $r \in W(x)$ and $x \in \bJ$, we say that the cone is \emph{tight}. 
  
  As with \cref{prop:chr of 2-limits}, we see that a morphism of two parallel cones $\alpha,\beta \in
  [\bJ,\bF](W,\bA(a,D))$, i.e. a modification, works out to be a family $(m_{x,r} : \alpha_{x,r} \to
  \beta_{x,r})_{x \in \bJ, r \in W(x)}$ such that for every $p : r \to s \in W(x)$ we have
  \[m_{x,s} \circ \alpha_{x,p} = \beta_{x,p} \circ m_{x,r},\]
  and for every $f : x \loose y \in \bJ$ and for every $r \in W(x)$ we have
  \[m_{y,Wf(r)} = Df m_{x,r} : Df \circ \alpha_{x,r} = \alpha_{y,Wf(r)} \to \beta_{y,Wf(r)}
  = Df \circ \beta_{x,r}.\]
\end{proposition}
\begin{proof}
  As with \cref{prop:desc of 2-limits}, this is simply given by unfolding the many layer of definitions.
\end{proof}

As with \cref{prop:chr of 2-limits}, we will require a notion of pullback with cones for $\sF$-weighted
digrams as well.
\begin{definition}[Pullback of cones]
  Let $\bJ$, $\bA$ be $\sF$-categories, $W : \bJ \to \bF$ a weight, and $D : \bJ \to \bA$ a diagram.
  Given a $W$-weighted cone $(a,\alpha)$ over $D$ and a morphism $f : b \to a$ in $\bA$, we define
  the pullback of $(a,\alpha)$ along $f$ as the cone $(b,f^*\alpha)$ defined by
  \[(f^*\alpha)_{x,r} = \alpha_{x,r} \circ f\]
  on objects $x \in \bJ$ and $r \in W(x)$, and
  \[(f^*\alpha)_{x,p} = \alpha_{x,p} \star f\]
  on objects $x \in \bJ$ and morphisms $p : r \to s \in W(x)$.

  Given a modification $m : \alpha \to \alpha' : W \loose \bA(a,D)$, we define the pullback $f^*m : f^*\alpha
  \to f^*\alpha'$ by
  \[(f^*m)_{x,r} = m_{x,r} \star f\]

  Furthermore, for $\phi : f \to g : b \to a$ a 2-cell in $\bA$, we define the modification $\phi^*\alpha :
  f^*\alpha \to g^*\alpha$ by \((\phi^*\alpha)_{x,r} = \alpha_{x,r} \star \phi\).
\end{definition}

We briefly recount the notion of enriched limit in the special case of $\sF$-categories:
\begin{definition}[{Weighted limit (see \cref{def:weighted limit})}]
  Given a shape $\sF$-category $\bJ$ and weight $W : \bJ \to \sF$, the $W$-weighted limit of a diagram
  $D : \bJ \to \bA$ in an $\sF$-category $\bA$ is an object $L \in \bA$ such that there exists an
  isomorphism in $\sF$
  \[\bA(-,L) \cong [\bJ,\bF](W,\bA(-,D)).\]
\end{definition}

\begin{proposition}\label{prop:chr of F-limits}
  Given a weight $W : \bJ \to \sF$, a diagram $D : \bJ \to \bA$ has a limit in $\bA$ if and only if there
  exists an object $\ell \in \bA$ and a tight cone $L : W \to \bA(\ell,D)$ with the property that
  \begin{enumerate}
    \item for every cone $\alpha : W \loose \bA(a,D)$, there exists a unique morphism $h_\alpha : a \loose
      \ell$ such that $\alpha = h_\alpha^*L$, and $h_\alpha$ is tight if and only if $\alpha$ is tight,
    \item for every modification $m : \alpha \to \alpha' : W \loose \bA(a,D)$ there exists a unique 2-cell
      $h_m : h_\alpha \to h_{\alpha'}$ such that $h_m^*\ell = m$.
  \end{enumerate}
  We will call a cone satisfying these properties a limiting cone.
\end{proposition}
\begin{proof}
  The proof for the isomorphisms is exactly the same as for \cref{prop:chr of 2-limits}, and so we only consider
  the tightness preservation and reflection of the isomorphism $h_a$ constructed as there. In particular, note
  that $h_a$ is an isomorphism in $\sF$, and so $h_a(\alpha)$ is tight if and only if $\alpha$ is.
\end{proof}
Furthermore, as aluded to throughout this section, any $\sF$-limit in an $\sF$-category $\bA$ produces a 2-limit in
the 2-category $\bA_\lambda$ of loose morphisms, and any 2-limit in $\bA_\lambda$ for which the universal cone is
tight and for which the universal morphisms induced by cones $\alpha$ are tight if and only if the cones $\alpha$
are tight is an $\sF$-limit. In particular, we see that an $\sF$-limit is essentially the same as a 2-limit that
respects and detects tightness of morphisms in a specific way. 

% Since $\sF$-limits form the core of our constructions, it will be useful to have a few examples of them, to help
% build intuition.
%
% \begin{example}[Tight limits]
%   Let $J$ be a 2-category, $W : J \to \Cat$ a weight. We call limits weighted by $W^+ : J^+ \to \bF$ \emph{tight
%   limits}. These will turn out to be important to the theories we define later, since we often want the structural
%   maps and projections to be preserved strictly for simplicity.
%
%   Some examples of tight limits include the following:
%   \begin{itemize}
%     \item  A \emph{tight terminal object} in an $\sF$-category $\bA$ is the tight limit of the empty diagram and
%       weight. It is of an object $t \in \bA$, such that for every object $a \in \bA$ there exists a unique tight
%       1-cell $!_a : a \to t$, and the only 2-cell $!_a \to\mathop{!}_a$ is identity. 
%     \item Let $J$ be a discrete $\sF$-category, meaing it has only identity 1-cells and 2-cells, and thus all
%       morphisms are tight, and let $W : J \to \Cat$ be the functor that is constantly the terminal category.
%       The tight limits corresponding to $J$, $W$ are then tight products, and so products whose projections are
%       all tight, and such that a the induced map from a morphisms is tight if and only if all of the morphisms
%       are tight.
%   \end{itemize}
% \end{example}

Since in a limit theory the models are limit preserving functors, and we're aiming to generalize those, we clearly
need to generalize the preservation of limits:
\begin{definition}[{Preservation of limits \cite[§3.2]{kelly1982a}}]
  An $\sF$-functor $F : \bA \to \bB$ preserves $W$-weighted limits for some weight $W : \bJ \to \bF$ if,
  for every diagram $D : \bJ \to \bA$ and limiting cone $L : W \to \bA(\ell, D)$, the image $F L$ of $L$
  under $F$, defined as the composite
  \[W \xrightarrow{\lim^W D} \bA(\ell, D) \xrightarrow{F} \bB(F\ell, F \circ D),\]
  is a limiting cone in $\bB$ of the diagram $F \circ D$.

  For a class $\Phi$ of weights, we say that $F$ preserves $\Phi$-weighted limits if it preserves $W$-%
  weighted limits for every weight $W \in \Phi$.
\end{definition}

Similarly, our limit theories are finitely complete categories, which means we also need completeness:
\begin{definition}[Completeness]
  For $\Phi$ a class of weights and $\bA$ an $\sF$-category, we say that $\bA$ is $\Phi$-complete if for
  every weight $W \in \Phi$ all diagrams $\bJ_W \to \bA$ have a limit.
\end{definition}

A key result for us is that, much like in the 1-categorical case, we can freely complete an $\sF$-category
under a chose class of weights. This free completion is what will allow us to turn finite limit $\sF$-sketches
into finite limit $\sF$-theories later. 
\begin{theorem}\label{thm:free completion}
  Let $\Phi$ be a set of weights with small domain. We can freely complete any $\sF$-category $\bA$ under
  $\sF$-limits, in the sense that there exists an $\sF$-category $\hat\bA$ and fully faithful $\sF$-%
  functor $\iota : \bA \into \hat \bA$, with the universal property that for any $\Phi$-complete $\sF$-%
  category $\bB$ and $\sF$-functor $F : \bA \to \bB$, there exists a unique (up to isomrphism) $\sF$-functor
  $\hat F : \hat\bA \to \bB$ such that $\hat F \circ \iota \cong F$.
\end{theorem}
\begin{proof}
  See theorem 5.35 of \cite{kelly1982a}.
\end{proof}

% As $\sF$-categories are simply enriched categories, we have a general notion of limits in $\sF$-%
% categories, as is described in \cref{sec:enriched/limits}. Here we show how limits are defined
% explicitly in $\sF$-categories.
\section{Enhanced finite limit 2-theories and sketches}
Generalizing the notion of finite limit theories, we would define an enhanced finite limit 2-theory,
as a small enhanced 2-category with finite weighted limits. To aid with constructing these categories,
and thus providing examples of such theories, we make use of enhanced finite limit 2-sketches, introduced
in \cite{arkor2024} as a specialization of enriched sketches defined by \cite{kelly1982a}. 

However, taking as $\sF$-category of models the $\sF$-category of $\sF$-functors and strict $\sF$-natural
transformations is too strict for our purposes, as this would require models to preserve all structure
of the theory strictly. Thus we would be unable to capture well the 2-category of cartesian categories,
for example. Therefore we shall introduce the following functor $\sF$-categories, which allow for a
controlled level of weakness in the categories of models, specified by the theory under consideration.

\begin{definition}[{\cite[§4.1]{lack2012}}]
  Let $w$ be a flavor of weakness $w \in \set{\text{strict},\text{pseudo}}$ and $F, G : \bA \to \bB$
  be $\sF$-functors. A \emph{loose $w$-natural transformation} $F \to G$ is a $w$-natural transformation
  of 2-functors $\alpha : F_\lambda \to G_\lambda$ such $\alpha$ is strictly 2-natural on tight morphisms,
  in the sense that $\alpha_d \circ F f = G f \circ \alpha_c$ for tight morphisms $f : c \to d \in \bA$. 
  A \emph{tight} $w$-natural transformation $F \to G$ is a loose $w$-natural transformation whose components
  are all tight.

  These determine an $\sF$-category $\bFun_w(\bA,\bB)$ whose objects are $\sF$-functors $\bA \to
  \bB$, loose (resp.\ tight) morphisms are loose (resp.\ tight) $w$-natural transformations,
  and 2-cells are modifications. 
\end{definition}

\begin{definition}[Logic]
  A \emph{$\Phi$-weighted $\sF$-theory} is a small $\sF$-category with all $\Phi$-weighted limits. A model
  for a theory $\bL$ is a $\sF$-limit preserving functor $\sF$-functor $M : \bL \to \bF$. We have for each
  flavor  of weakness $w \in \set{\text{strict}, \text{pseudo}}$ an $\sF$-category of $w$ models, namely
  the full sub-$\sF$-category $\Mod_w(\bL, \bF)$ of the functor $\sF$-category $\bFun_w(\bL,\bF)$ on $\Phi$%
  -limit preserving $\sF$-functors.
\end{definition}

\begin{definition}[{\cite[§6.3]{kelly1982a}}]
  Let $\Phi$ be a class of weights in $\sF$. An \emph{$\Phi$-weighted limit sketch} $(\bS,\cC)$ is an
  $\sF$-category $\bS$ equipped with a set of $\Phi$-weighted cones $C = \set{\alpha : W_\alpha \to \bS(
  a_\alpha,D_\alpha)}$. In other words, for each $\alpha \in C$, we have $W_\alpha \in \Phi$.

  A morphism of sketches $(\bA,\cC) \to (\bB, \cD)$ is an $\sF$-functor $F : \bA \to \bB$ such that every
  cone in $\cC$ is mapped to a cone in $\cD$.
\end{definition}

Every $\sF$-category $\bA$ gives rise to a $\Phi$-limit sketch by taking $\cC$ to be the class of
$\Phi$-weighted limit cones in $\bA$. We shall denote this sketch by $\sS_\Phi(\bA)$.

\begin{theorem}
  Let $\Phi$ be a class of weights and $(\bA,\cC)$ be a small sketch. Then there exists a $\Phi$-complete
  $\sF$-category $\Th(\bA,\cC)$ and faithful $\sF$-functor $\iota : \bA \to \Th(\bA,\cC)$ taking the cones
  in $\cC$ to limit cones in $\Th(\bA,\cC)$, and for every $\Phi$-complete $\sF$-category $\bB$, there exists
  an $\sF$-equivalence
  \[\Mod_w(\Th(\bA,\cC),\bB) \simeq [(\bA,\cC), \sS_\Phi(\bB)]_w,\]
  where $[(\bA,\cC), \sS_\Phi(\bB)]_w$ denotes the $\sF$-category of sketch-homomorphisms, $w$-natural
  transformations, and modifications. Furthermore, this equivalence natural in $\bB$.
\end{theorem}
\begin{remark}
  In the strict case, this is proven as theorem 6.23 in \cite{kelly1982a} for general enriched categories.
  A proof for the weak case generalizes relatively straightforwardly from this, though is quite invloved
  and so is omitted here.
\end{remark}


\subsection{Examples of \texorpdfstring{$\sF$}{F}-theories}
To get a better understanding of how these theories work, and what their 2-categories of models look like,
we present a few examples of $\sF$-sketches and $\sF$-theories.

\begin{example}
  The simplest theory is the theory of a single category: the sketch consists of the terminal
  $\sF$-category, with no cones. Models of this theory in $\Cat$ are exactly categories,
  homomorphisms are functors, and the transformations are exactly natural transformations.
\end{example}

\begin{example}
  There are two theories of morphisms, both of which are generated by a sketch with only a single morphism.
  The difference is whether this morphism is tight or loose. In the theory of tight morphisms, models are
  categories $A,B$ with a functor $f : A \to B$ between them, and homomorphisms $f \to f'$ are strictly
  commuting squares
  \[\begin{codi}
    \obj {A & A' \\ B & B' \\};
    \mor A -> A' f':-> B';
    \mor[swap] A f:-> B :-> B';
  \end{codi}\]
  Meanwhile, in the theory of loose morphisms, the models are the same, but the homomorphisms are squares 
  \[\begin{codi}
    \obj {A & A' \\ B & B' \\};
    \mor A -> A' f':-> B';
    \mor[swap] A f:-> B :-> B';
    \mor[mid] A \cong:phantom B';
  \end{codi}\]
  commuting up to isomorphism. This illustrates what the additional data of the class of tight morphisms
  provides: they allow us to control the strictness of the \emph{homomorphisms} of models.
\end{example}

\begin{example}
  The theory $\Cart$ of cartesian categories is generated by a single object $A$, morphisms
  $e : 1 \loose A$ and $p : A \times A \loose A$, cones witnessing $1$ as the terminal object
  and $A \times A$ as the product of $A$ with itself, and 2-cells witnessing $e$ (resp. $p$)
  right adjoint to the universal map $A \to 1$ (resp. $\seq{\id_A, \id_A} : A \to A \times A$).

  A model of $\Cart$ consists of a category $\cC$ together with adjunctions $\Delta_\cC \dashv \times_\cC
  : \cC \times \cC \to \cC$ and $!_\cC \dashv 1_\cC : 1 \to \cC$, where $\Delta_\cC : \cC \to \cC \times
  \cC$ is the diagonal and $!_\cC : \cC \to 1$ is the universal map. Thus we see that a model is exactly
  a category with (chosen) finite products, a cartesian category. Furthermore, a homomorphism of models
  $(\cC,\times_\cC,1_\cC) \to (\cD, \times_\cD, 1_\cD)$ consists of a functor $F : \cC \to \cD$ preserving
  these adjunctions \emph{up to isomorphism}. Thus these are the expected, non-strict, maps of cartesian
  categories.
\end{example}
\begin{example}
  Dually to the previous example, taking $\Cocart$ be the theory generated by the same data as $\Cart$,
  except taking $e,p$ to be \emph{left} adjoints rather than right adjoints, we find that the models
  are exactly cocartesian categories, and the homomorphisms are functors preserving finite coproducts
  only up to isomorphism.
\end{example}
\begin{example}\label{ex:fib}
  The theory of fibrations $\Fib$ is generated by a tight morphism $p : \bE \to \bB$, a loose adjunction $
  t \dashv r : \bB \comma p \loose \bE^\to$ with $tr = \id_{\bP \comma p}$ and $t$ the universal morphism
  generated by the cone $\seq{p d, \alpha, c}$ with $\alpha : d \to c : E^\to \to E$. If the unit of the
  adjunction $\eta : 1_{p \comma \bB} \to rt$ is further specified to be the identity, so that $rt = \id_%
  {\bP^\to}$, we have the theory of discrete fibrations.
\end{example}


\section{From syntax to \texorpdfstring{$\sF$}{F}-theories}
In this section we show a correspondence between the syntax of deductive theories, presented by judgements and
inference rules, and finite limit $\sF$-theories, much in the style of \cite{coraglia2024a}. The connection is
generally as follows: a (dependent) judgement $d\;\cD \vdash j\;\cJ$ is interpreted is a tight morphism $j : \cJ
\to \cD$, with the idea that the fibers of $j$ over any instance $d\;\cD$ corresponds to $d\;\cD \vdash j\;\cJ$.
A non-dependent judgement can then be viewed as a judgement over the terminal object $1$, so that the dependency
is trival. Equivalently a non-dependent judgement is just an object. An additional feature of this scheme is that,
for any non-dependent judgement $\cJ$, taking the tight arrow object $\cJ^\to$ gives a judgement of morphisms
dependent on $\cJ \times \cJ$. In modelling type theory, we would for example have a judgement of contexts
$\vdash \ctx$, and then a judgement of substitions $\Gamma\ctx, \Delta\ctx \vdash \sigma \ctx^\to$. Given a
(dependent) judgement of types over contexts $\rty : \ty \to \ctx$, we would then want to formulate a rule for
stability under such substitutions:
\begin{mathpar}
  \infer{\Gamma\ctx,\Delta\ctx \vdash {{\sigma \ctx^\to}} \\ \Delta \ctx \vdash A\ty}{\Gamma \ctx \vdash
  \sub(A,\sigma) \ty}
\end{mathpar}
We capture this rule as a loose morphism $\sub : \ctx^\to \pb[\cod]{\rty} \ty \to \ty$ for which the following
triangle commutes.
\[\begin{codi}[tetragonal=base 4.5em height 4.5em]
  \obj {|(P)| \ctx^\to \pb[\cod]{\rty} \ty & & \ty \\ & \ctx \\};
  \mor ctx \dom\pi_1:<- P \sub:-> ty \rty:-> ctx; 
\end{codi}\]
If we now note that $\ctx^\to \pb[\cod]{\rty} \ty \cong \ctx \comma \rty$, we see that $\sub$ is exactly as the
morphism $\ell : \ctx \comma \rty \to \rty$ specifying the domain of the lifting in a fibration $\ty \to \ctx$.

In short, we model non-dependent judgements as objects, dependent judgements as tight morphisms, and rules as
loose morphisms.

% \begin{proposition}
%   Let $A$ be any object of an $\sF$-category $\bA$ with finite weighted $\sF$-limits. Then any 2-cell $\rho : a
%   \to b : X \to A^\to$ is $\cod$-cartesian if and only if the following square is a pullback in $\bA(X,A)$
%   \[\begin{codi}[tetragonal=base 5.5em height 4.5em]
%     \obj {\dom a & \dom b \\ \cod a & \cod b \\};
%     \mor (dom a) \dom\rho:-> (dom b) \lambda b:-> (cod b);
%     \mor[swap] (dom a) \lambda a:-> (cod a) \cod\rho:-> (cod b);
%   \end{codi}\]
% \end{proposition}
%
% \begin{proposition}
%   Let $f : A \to B^\to$ be a morphism in an $\sF$-category $\bA$ with finite weighted $\sF$-limits. Then for any
%   2-cell $\rho : a \to b : X \to A$, the whiskering $f\rho$ corresponds to the commuting square
%   \[\begin{codi}[tetragonal=base 6.5em height 4.5em]
%     \obj {\dom fa & \dom fb \\ \cod fa & \cod fb \\};
%     \mor (dom fa) \dom f\rho:-> (dom fb) \lambda fb:-> (cod fb);
%     \mor[swap] (dom fa) \lambda fa:-> (cod fa) \cod f\rho:-> (cod fb);
%   \end{codi}\]
% \end{proposition}
%
% \begin{lemma}
%   For any fibration $p : A \to B$ and 2-cells $a \xrightarrow{n} b \xrightarrow{m} c : X \to A$, if $m\circ n$
%   and $m$ are $p$-cartesian, then so is $n$.
% \end{lemma}
% \begin{proof}
%   Suppose we are given a $p$-cartesian cone over $n$:
%   \[\begin{codi}[tetragonal=base 5.5em height 3.5em]
%     \obj { |(Y)| Y & |(X)| X & |(A)| A \\ & & |(B)| B \\};
%     \mor X ["b",name=x]:[->,bend left] A ["p",name=p]:-> B;
%     \mor X ["a",name=y,swap]:[->,bend right] A;
%     \mor y n:[-2>] x;
%     \mor Y ["x",name=a]:[->, bend left=5em] A;
%     \mor Y ["g",name=g]:-> X;
%     \mor Y ["pc",name=pa,swap]:[->, bend right=2em] B;
%     \mor[swap] a {{k}}:[-2>,slide=0.1em] X;
%     \mor[swap] pa {{l}}:[-2>,slide=0.4em, shorten=0.4em] X;
%   \end{codi}\]
%   Then we have that
%   \[p(mg \circ k) = pmg \circ pk = pmg \circ png \circ l = p(m \circ n)g \circ l,\]
%   so since $m \circ n$ is $p$-cartesian, there exists a unique map $\gamma : x \to a$ such that $p\gamma = l$
%   and $(m \circ n)g \circ \gamma = mg \circ k$. We still need to show that $ng \circ \gamma = k$, but this
%   follows from $m$ being $p$-cartesian, since $k$ is the unique map $\gamma'$ such that $mg \circ \gamma' = mg
%   \circ k$ and $p\gamma' = pk$. Since $mg \circ ng \circ \gamma = mg \circ k$ and $p(ng \circ \gamma) =
%   png \circ p\gamma = png \circ l = pk$, we see that $ng \circ \gamma$ also satisfies the properties of
%   $\gamma'$, and so we conclude that by uniqueness that $ng \circ \gamma = k$.
% \end{proof}

\ifSubfilesClassLoaded{\printbibliography}{}
  \end{document}
