\documentclass[../thesis.tex]{subfiles}

\ifSubfilesClassLoaded{%
    \externaldocument{../build/2-preliminaries}%
    % \IfFileExists{references.bib}{}{\addbibresource{./references.bib}}
  }{}

  

\begin{document}
  \chapter{Enhanced finite 2-limit theories}
  In this chapter we generalize the notion of finite limit theories to two dimensions. We want
  to define a theory $T$ to be a 2-category with finite 2-limits, and its 2-categories of models
  $\Mod(T)$ to be finite 2-limit preserving functors $T \to \Cat$. However, with this naive
  definition, the homomorphisms of models turn out to be either too strict or too loose. If we
  consider strict 2-natural transformations, then all structure has to be preserved on the nose.
  For example, consider a theory of cartesian categories; in the strict case, its homomorphisms
  must preserve the choices of products and terminal objects exactly. On the other hand, if we take
  morphisms to be for example pseudonatural transformations, then the morphisms need only preserve
  the cartesian structure up to isomorphism, aligning with the standard definition of preservation
  of products. However, in a theory of (discrete) fibrations, morphisms would be squares commuting 
  only up to isomorphism, rather than the standard strict equality. It is this issue we aim to
  resolve, as this fine control over strictness of models will be very relevant in describing
  sketches of type theories in later chapters.

  We denote by $\sF$ the full subcategory of $\Cat^\to$ on full embeddings, where a functor is a
  full embedding if it is fully faithful and injective on objects. Thus the objects of $\sF$ are
  full embeddings, and a morphism $A \to B$ in $\sF$ is a commuting squares
  \[\begin{codi}
    \obj {A_\tau & B_\tau \\
          A_\lambda & B_\lambda \\};

    \mor A_tau -> B_tau ->,hook B_lambda;
    \mor A_tau ->,hook A_lambda -> B_lambda;
  \end{codi}\]
  where $A_\tau$ denotes the domain of $A$ and $A_\lambda$ the codomain. 
  \begin{remark}
    An object of $\sF$ is in equivalently a category $A_\lambda$ equipped with a class $A_\tau$ of
    objects. We call the objects of $A_\lambda$ loose and the objects in $A_\tau$ tight. Taking this
    perspective, we find that a lot of structure will be uniqely determined by $A_\lambda$, with
    $A_\tau$ placing restrictions on what structure is allowed. For example, a morphism in $\sF$
    is simply a functor $F : A_\lambda \to B_\lambda$ with the restriction that if $f \in A_\tau$,
    then $Ff \in B_\tau$, and so $\sF$-morphisms are uniquely determined by their action on the
    loose morphisms.
  \end{remark}
  
  \todo{Move the following lemma to preliminaries.}
  \begin{lemma}
    The full embeddings are the right class of an orthogonal factorisation system on $\Cat$, where
    the left class are surjective-on-objects functors.
  \end{lemma}
  \begin{proof}
    Given any functor $F : A \to B$, let $A'$ denote the full subcategory of $B$ spanned by the
    objects in the image of $F$. Clearly $F$ then factors through $A'$ as
    \[\begin{codi}
      \obj {A & & B \\
            & A' & \\ };
      \mor A F:-> B;
      \mor A G:-> A' H:-> B;
    \end{codi}\]
    where $G$ is surjective on objects and $H$ is fully faithful and injective on objects.
  \end{proof}

  \begin{corollary}
    $\sF$ is a reflective subcategory of $\Cat^\to$.
  \end{corollary}

  \begin{corollary}
    $\sF$ is cartesian closed.
  \end{corollary}

  \begin{lemma}
    The functors $\bone_\lambda : \bzero \into \bone$ and $\bone_\tau : \bone \into \bone$ are the
    representables of $\sF$, in the sense that for any $A \in \sF$, we have $A_\tau \cong \sF(\bone_%
    \tau, A)$ and $A_\lambda
    \cong \sF(\bone_\lambda, A)$, naturally in $A$.
  \end{lemma}

  \begin{corollary}
    For any $A, B \in \sF$, the exponential $[A,B] \in \sF$ satisfies that $[A,B]_\tau \cong \sF(A,B)
    $ and $[A,B]_\lambda \cong \Cat(A_\lambda, B_\lambda)$, naturally in $A,B$.
  \end{corollary}
  \begin{proof}
    We have natural isomorphisms
    \[[A,B]_\tau \cong \sF(\bone_\tau, [A,B]) \cong \sF(\bone_\tau \times A, B) \cong \sF(A,B)\]
    and
    \[[A,B]_\lambda \cong \sF(\bone_\lambda, [A,B]) \cong \sF(\bone_\lambda \times A, B) \cong
    \Cat(A_\lambda, B_\lambda),\]
    where we have $(\bone_\lambda \times A) \cong (\bzero \into A_\lambda)$ since limits are computed
    pointwise, so that $\sF(\bone_\lambda \times A, B)$ is uniquely determined on the tight part,
    irrespective of the loose part, and so $\sF(\bone_\lambda \times A, B) \cong \Cat(A_\lambda, B_%
    \lambda)$.
  \end{proof}
  
  \begin{definition}[{\cite[§3.1]{lack2012}}]
    An \emph{enhanced 2-category}, or $\sF$-category, is a category enriched in $\sF$.
  \end{definition}
  \begin{notation}
    We will use blackboard bold letters $\bA,\bB,\bC,\dots$ to denote $\sF$-categories. Furthermore,
    we use $A \loose B$ to denote loose morphisms of $\bA$, i.e.\ morphisms in $\bA_\lambda$, and
    $A \to B$ to denote tight morphisms, i.e\ morphisms in $\bA_\tau$.
  \end{notation}
  \begin{remark}
    Since an object of $\sF$ is essentially a category equipped with a class of tight objects, we
    can equivalently view an $\sF$-category as a 2-category equipped with a class of tight morphisms
    which is closed under identity (as the identity assigning functors preserve tightness and in the
    terminal object of $\sF$ all morphisms are tight) and composition (for similar reasons regarding
    the composition functor).
  \end{remark}

  Any $\sF$-category $\bA$ gives rise two 2-categories: the 2-category of tight morphisms $\bA_
  \lambda$ and its locally full sub-2-category of tight morphisms $\bA_\tau$. We use $J_\bA$ to
  denote the inclusion $\bA_\tau \embed \bA_\lambda$ and

  We let $\bF$ denote the $\sF$-category of $\sF$-categories, whose tight morphisms are $\sF$-%
  functors and loose morphisms are functors on the loose categories, and whose 2-cells are $\sF$-%
  natural transformations, i.e. natural transformations on whose components are all tight.
  Furthermore, given two $\sF$-categories $\bA,\bB$, we let $[\bA,\bB]$ denote the $\sF$-category
  whose objects are $\sF$-functors $\bA \to \bB$, whose tight morphisms are $\sF$-natural
  transformations, loose morphisms are natural transformations whose componenets need not be
  tight, and whose 2-cells are modifications.

  Any 2-category $K$ can be regarded as an $\sF$-category in two natural ways: taking only
  identities to be tight yields what we call the \emph{inchordate} $\sF$-category $K^-$, and 
  dually taking all morphisms to be tight yields the \emph{chordate} $\sF$-category $K^+$. The
  operations of taking a 2-category to its inchordate or chordate $\sF$-category extends to
  functors $2\Cat \to \sF\text-\Cat$, and these are respectively left and right adjoint to the
  functor $(-)_\lambda : \sF\text-\Cat \to 2\Cat$ taking a $\sF$-category to its underlying 2-%
  category of loose morphisms.


  \section{Limits in \texorpdfstring{$\sF$}{F}-categories}
  % In this section we develop \emph{weighted limits}, a generalization of the classical conical
  % limits that behave much more nicely in enriched categories, in the specific context of $\sF$%
  % -categories. These weighted limits are fundamental for the enhanced limit 2-sketches we will
  % define later, and those will serve as the main method with which we later describe logics, for
  % example giving us a nice theory of dependent type theories, whose models are exactly the natural
  % models of Awodey.

  \begin{definition}
    Let $\bJ$ be an $\sF$-category. A $\bJ$-shaped weight is a functor $W : \bJ \to \bF$, where $\bF$
    is the $\sF$-category induced by the cartesian closed structure of $\sF$. Given a $\bJ$-shaped
    diagram $D : \bJ \to \bA$ in an $\sF$-category $\bA$, a loose $W$-weighted cone $(a,\alpha)$
    from $a \in \bA$ to $D$ is a loose natural transformation $W \loose \bA(a,D)$. Concretely,
    $\alpha$ consists of:
    \begin{itemize}
      \item for each object $x \in \bJ$ and $r \in W(x)$, a morphism $\alpha_{x,r} : a \loose Dx$,
        and
      \item for each $p : r \to s \in W(x)$, a 2-cell $\alpha_{x,p} : \alpha_{x,r} \to \alpha_{x,s}$,
        such that
      \item for each morphism $f : x \loose y$ in $\bJ$ and $r \in W(x)$, we have $Df \circ \alpha_%
        {x,r} = \alpha_{y,W(f)(r)}$, and
      \item for each $p : r \to s$ in $W(x)$ we have $Df \star \alpha_{x,p} = \alpha_{y,W(f)(p)}$.
    \end{itemize} 
    The cone $\alpha$ is \emph{tight} if, whenever $r \in W(x)$ is a tight object, the arrow
    $\alpha_{x,r} : a \to Dx$ is tight.

    A morphism of two parallel cones $\alpha,\alpha' \in [\bJ,\bF](W,\bA(a,D))$ is a modification,
    which explicitly works out to be a family $(m_x : \alpha_x \to \alpha'_x)_{x \in \bJ}$ such
    that for any $f : x \loose y \in \bJ$, we have $m_y \star Wf = \bA(a,Df) \star m_x$. Concretely,
    each $m_x : \alpha_x \to \alpha'_x$ is a 2-cell in $\bF$, and so an assignment for each
    $r \in W(x)$ a morphism $m_{x,r} : \alpha_{x,r} \to \alpha'_{x,r}$ such that for every $p :
    r \to s \in W(x)$, we have $m_{x,s} \circ \alpha_{x,p} = \alpha'_{x,p} \circ m_{x,r}$,
    and for every $f : x \loose y \in \bJ$, we have $m_y \star Wf = \bA(a,Df) \star m_x$
    (where $m_y \star Wf : \alpha_y \circ Wf \to \alpha'_y \circ Wf$), which
    works out to be for every $r \in W(x)$ the condition that
    \[m_{y,Wf(r)} = Df \circ m_{x,r} : Df \circ \alpha_{x,r} = \alpha_{y,Wf(r)} \to \alpha'_{y,Wf(r)}
    = Df \circ \alpha'_{x,r}.\]
  \end{definition}

  \begin{definition}[Pullback of cones]
    Let $\bJ, \bA$ be $\sF$-categories, $W : \bJ \to \bF$ a weight, and $D : \bJ \to \bA$ a diagram.
    Given a $W$-weighted cone $(a,\alpha)$ over $D$ and a morphism $f : b \to a$ in $\bA$, we define
    the pullback of $(a,\alpha)$ along $f$ as the cone $(b,f^*\alpha)$ defined by
    \[(f^*\alpha)_{x,r} = \alpha_{x,r} \circ f\]
    on objects $x \in \bJ$ and $r \in W(x)$, and
    \[(f^*\alpha)_{x,p} = \alpha_{x,p} \star f\]
    on objects $x \in \bJ$ and morphisms $p : r \to s \in W(x)$.
  
    Given a modification $m : \alpha \to \alpha' : W \loose \bA(a,D)$, we define the pullback $f^*m : f^*\alpha
    \to f^*\alpha'$ by
    \[(f^*m)_{x,r} = m_{x,r} \star f\]


    Furthermore, for $\phi : f \to g : b \to a$ a 2-cell in $\bA$, we define a modification $\phi^*\alpha :
    f^*\alpha \to g^*\alpha$ by \((\phi^*\alpha)_{x,r} = \alpha_{x,r} \star \phi\).
  \end{definition}

  \begin{definition}[Weighted Limit]
    Given a shape $\sF$-category $\bJ$ and weight $W : \bJ \to \sF$, the $W$-weighted limit of a diagram
    $D : \bJ \to \bA$ in an $\sF$-category $\bA$ is an object $L \in \bA$ such that there exists an
    isomorphism in $\sF$
    \[\bA(-,L) \cong [\bJ,\bF](W,\bA(-,D)).\]
  \end{definition}

  \begin{proposition}
    Given a weight $W : \bJ \to \sF$, a diagram $D : \bJ \to \bA$ has a limit in $\bA$ if and only if there
    exists an object $\ell \in \bA$ and cone $L : W \to \bA(\ell,D)$ (note the tightness) with the property that
    \begin{enumerate}
      \item for every cone $\alpha : W \loose \bA(a,D)$, there exists a unique morphism $h_\alpha : a \loose \ell$
        such that $\alpha = h_\alpha^*L$, and $h_\alpha$ is tight if and only if $\alpha$ is tight,
      \item for every modification $m : \alpha \to \alpha' : W \loose \bA(a,D)$ there exists a unique 2-cell
        $h_m : h_\alpha \to h_{\alpha'}$ such that $h_m^*\ell = m$.
    \end{enumerate}
    We will call a cone satisfying these properties a limiting cone.
  \end{proposition}
  \begin{proof}
    We start by unfolding the naturality squares of the isomorphism $h : [\bJ,\bF](W,\bA(-,D)) \cong \bA(-,\ell)$
    which is supposed to exist for $\ell$ to be the $W$-weighted limit of $D$. For each $a \in \bA$, we find that
    $h_a : [\bJ,\bF](W,\bA(a,D)) \cong \bA(a,\ell)$ is an isomorphism in $\bF$, i.e.\ an invertible functor which
    preserves and reflects tight morphisms. Naturality tells us that given a morphism $\phi : b \to a$, we have
    that $h_b \circ \phi^* = \bA(\phi,\ell) \circ h_a$, where $\phi^*$ denotes the operation of pulling back a 
    cone along a morphism of $\bA$. On cones $\alpha : W \loose \bA(b,\ell)$, this gives us that $h_b(\phi^*
    \alpha) = h_a(\alpha) \circ \phi$, and on modifactions $m : \alpha \to \alpha' : W \loose \bA(b,D)$, we
    have $h_b(\phi^*m) = h_a(m) \star \phi$.
    
    Now, given a natural isomorphism $h$ as above, we can consider the cone $L = h^{-1}(\id_\ell)$. Given any
    other cone $\alpha : W \loose \bA(a,D)$, naturality tells us that
    \[h_a(h_a(\alpha)^*L) = h_\ell(L) \circ h_a(\alpha) = \id_\ell \circ h_a(\alpha) = h_a(\alpha),\]
    so $h_a(\alpha)^*L = \alpha$ since $h$ is an isomorphism. Similarly, for a modification $m : \alpha \to
    \alpha'$ we find that
    \[h_a(h_a(m)^*L) = h_a(m) = h_\ell(L) \star h_a(m) = \id_\ell \star h_a(m) = h_a(m),\]
    so again $h_a(m)^*L = m$.


    In the other direction, suppose we are given an object $\ell \in \bA$ and cone $L : W \loose \bA(\ell,D)$
    such that (1) and (2) hold. Then we construct the natural isomorphism $h : [\bJ,\bF](W,\bA(-,D)) \cong
    \bA(-,\ell)$ by $h_a(\alpha) = h_\alpha$ and $h_a(m : \alpha \to \alpha') = h_m$. It is not hard to verify
    that $h_a$ is an isomorphism for each $a \in \bA$, so we focus on naturality: given $\phi : b \to a \in
    \bA$, we must show that $h_b \circ \phi^* = \bA(\phi,\ell) \circ h_a$. On cones, this amounts to showing
    that for any cone $\alpha : W \loose \bA(a,D)$, we have that $h_b(\phi^*\alpha) = h_a(\alpha) \circ \phi$.
    But
    \[
      \phi^*\alpha = \phi^*h_a(\alpha)^*L = (h_a(\alpha) \circ \phi)^*L,
    \]
    so uniqueness of $h_b(h_a(\alpha) \circ \phi)$ gives us that $h_b(\phi^*\alpha) = h_a(\alpha) \circ \phi)$.
    For the naturality condition on 2-cells, let $m : \alpha \to \alpha'$ be a modification. We must show that
    $h_b(\phi^*m) = h_a(\alpha) \star m$, which much like for cones follows from uniqueness:
    \[\phi^*m = \phi^*(h_a(m)^*L) = (h_a(m)\star\phi)^*L,\]
    so $h_b(\phi^*m) = h_a(m)\star\phi$.

    Finally, for preservation and reflection of tightness, note that by assumption $h_a(\alpha)$ is tight if and
    only if $\alpha$ is, so $h_a$ preserves and reflects tightness by definition.
  \end{proof}

  Furthermore, note that any $\sF$-limit in an $\sF$-category $\bA$ produces a 2-limit in the 2-category $\bA_
  \lambda$ of loose morphisms, and any 2-limit in $\bA_\lambda$ for which the universal cone is tight and
  for which the universal morphisms induced by cones $\alpha$ are tight if and only if the cones $\alpha$ are
  tight is an $\sF$-limit.

  \begin{example}[Tight limits]
    Given any 2-category $J$ and weight $W : J \to 2\Cat$, we can consider the corresponding
    chordate $\sF$-category $J^+$ and weight $W^+ : J^+ \to \sF\text-\Cat$ taking every object
    $x$ of $J$ to the chordate $\sF$-category $W(x)^+$. 
  \end{example}
  
  \begin{example}[Tight terminal object]
    A \emph{tight terminal object} in an $\sF$-category $\bA$ is an object $t \in \bA$, such that for
    every object $a \in \bA$ there exists a unique tight 1-cell $!_a : a \to t$, and the only 2-cell
    $!_a \to\mathop{!}_a$ is identity. 
  \end{example}


  \begin{example}[Tight products]
    Let $J$ be a discrete $\sF$-category, meaing it has only identity 1-cells and 2-cells, and thus all
    morphisms are tight.
  \end{example}

  \begin{definition}[{Preservation of limits \cite[§3.2]{kelly1982a}}]
    An $\sF$-functor $F : \bA \to \bB$ preserves $W$-weighted limits for some weight $W : \bJ \to \bF$ if,
    for every diagram $D : \bJ \to \bA$ and limiting cone $L : W \to \bA(\ell, D)$, the image $F L$ of $L$
    under $F$, defined as the composite
    \[W \xrightarrow{\lim^W D} \bA(\ell, D) \xrightarrow{F} \bB(F\ell, F \circ D),\]
    is a limiting cone in $\bB$ of the diagram $F \circ D$.

    For a class $\Phi$ of weights, we say that $F$ preserves $\Phi$-weighted limits if it preserves $W$-%
    weighted limits for every weight $W \in \Phi$.
  \end{definition}

  \begin{definition}[Completeness]
    For $\Phi$ a class of weights and $\bA$ an $\sF$-category, we say that $\bA$ is $\Phi$-complete if for
    every weight $W \in \Phi$ all diagrams $\bJ_W \to \bA$ have a limit.
  \end{definition}

  \todo{Maybe need to refine this theorem a bit, not quite sure yet.}
  \begin{theorem}\label{thm:free completion}
    Let $\Phi$ be a set of weights. For any $\sF$-category $\bA$, we can freely complete a under $\sF$-limits,
    in the sense that there exists an $\sF$-category $\hat\bA$ and fully faithful $\sF$-functor $\iota : \bA
    \into \hat \bA$, with the universal property that for any $\Phi$-complete $\sF$-category $\bB$ and $\sF$-%
    functor $F : \bA \to \bB$, there exists a unique (up to isomrphism) $\sF$-functor $\hat F : \hat\bA \to \bB$
    such that $\hat F \circ \iota \cong F$.
  \end{theorem}


  % As $\sF$-categories are simply enriched categories, we have a general notion of limits in $\sF$-%
  % categories, as is described in \cref{sec:enriched/limits}. Here we show how limits are defined
  % explicitly in $\sF$-categories.
  \section{Enhanced finite limit 2-theories and sketches}
  Generalizing the notion of finite limit theories, we would define an enhanced finite limit 2-theory,
  or as we shall call it, a logic, as a small enhanced 2-category with finite weighted limits. To aid
  with constructing these categories, and thus providing examples of such logics, we make use of
  enhanced finite limit 2-sketches, introduced in \cite{arkor2024} as a specialization of enriched
  sketches defined by \cite{kelly1982a}. 

  However, taking as $\sF$-category of models the $\sF$-category of $\sF$-functors and strict $\sF$-natural
  transformations is too strict for our purposes, as this would require models to preserve all structure
  of the theory strictly. Thus we would be unable to capture well the 2-category of cartesian categories,
  for example. Therefore we shall introduce the following functor $\sF$-categories, which allow for a
  controlled level of weakness in the categories of models, specified by the theory under consideration.
  
  \begin{definition}[{\cite[§4.1]{lack2012}}]
    Let $w$ be a flavor of weakness $w \in \set{\text{strict},\text{pseudo}}$ and $F, G : \bA \to \bB$
    be $\sF$-functors. A \emph{loose $w$-natural transformation} $F \to G$ is a $w$-natural transformation
    of 2-functors $\alpha : F_\lambda \to G_\lambda$ such $\alpha$ is strictly 2-natural on tight morphisms,
    in the sense that $\alpha_d \circ F f = G f \circ \alpha_c$ for tight morphisms $f : c \to d \in \bA$. 
    A \emph{tight} $w$-natural transformation $F \to G$ is a loose $w$-natural transformation whose components
    are all tight.

    These determine an $\sF$-category $\bFun_w(\bA,\bB)$ whose objects are $\sF$-functors $\bA \to
    \bB$, loose (resp.\ tight) morphisms are loose (resp.\ tight) $w$-natural transformations,
    and 2-cells are modifications. 
  \end{definition}

  \begin{definition}[Logic]
    A \emph{$\Phi$-weighted logic} is a small $\sF$-category with all $\Phi$-weighted limits. A model for a
    logic $\bL$ is a $\sF$-limit preserving functor $\sF$-functor $M : \bL \to \bF$. We have for each flavor 
    of weakness $w \in \set{\text{strict}, \text{pseudo}}$ an $\sF$-category of $w$ models, namely the full
    sub-$\sF$-category $\Mod_w(\bL, \bF)$ of the functor $\sF$-category $\bFun_w(\bL,\bF)$ on $\Phi$-limit
    preserving $\sF$-functors.
  \end{definition}

  \begin{definition}[{\cite[§6.3]{kelly1982a}}]
    Let $\Phi$ be a class of weights in $\sF$. An \emph{$\Phi$-weighted limit sketch} $(\bS,\cC)$ is an
    $\sF$-category $\bS$ equipped with a set of $\Phi$-weighted cones $C = \set{\alpha : W_\alpha \to \bS(
    a_\alpha,D_\alpha)}$. In other words, for each $\alpha \in C$, we have $W_\alpha \in \Phi$.

    A morphism of sketches $(\bA,\cC) \to (\bB, \cD)$ is an $\sF$-functor $F : \bA \to \bB$ such that every
    cone in $\cC$ is mapped to a cone in $\cD$.
  \end{definition}

  Every $\cF$-category $\bA$ gives rise to a $\Phi$-limit sketch by taking $\cC$ to be the class of
  $\Phi$-weighted limit cones in $\bA$. We shall denote this sketch by $\sS_\Phi(\bA)$.
  
  \todo{Remark on size issues and explain concretely what smallness means in this context.}
  \begin{theorem}
    Let $\Phi$ be a class of weights and $(\bA,\cC)$ be a small sketch. Then there exists a $\Phi$-complete
    $\sF$-category $\Th(\bA,\cC)$ and faithful $\sF$-functor $\iota : \bA \to \Th(\bA,\cC)$ taking the cones
    in $\cC$ to limit cones in $\Th(\bA,\cC)$, and for every $\Phi$-complete $\cF$-category $\bB$, there exists
    an $\cF$-equivalence
    \[\Mod_w(\Th(\bA,\cC),\bB) \simeq [(\bA,\cC), \sS_\Phi(\bB)]_w,\]
    where $[(\bA,\cC), \sS_\Phi(\bB)]_w$ denotes the $\cF$-category of sketch-homomorphisms, $w$-natural
    transformations, and modifications. Furthermore, this equivalence natural in $\bB$.
  \end{theorem}
  \todo{Check that this is actually fully correct, and maybe write down a proof.}

  \section{Examples of theories}
  \begin{example}
    The simples theory is the theory of a single category: the sketch consists of the terminal
    $\sF$-category, with no cones. Models of this theory in $\Cat$ are exactly categories,
    morphisms of models are functors, and the transformations are exactly natural transformations.
  \end{example}

  \begin{example}
    The theory of tight morphisms is generated by a singular tight morphism $A \to B$, with no additional
    data.
  \end{example}

  \begin{example}
    The theory $\Cart$ of cartesian categories is generated by a single object $A$, morphisms
    $e : 1 \loose A$ and $p : A \times A \loose A$, cones witnessing $1$ as the terminal object
    and $A \times A$ as the product of $A$ with itself, and 2-cells witnessing $e$ (resp. $p$)
    right adjoint to the universal map $A \to 1$ (resp. $\seq{\id_A, \id_A} : A \to A \times A$).

    The 2-category $\Mod_p(\Cart,\Cat^+)_\lambda$ of models of $\Cart$ in $\Cat$ and loose pseudonatural
    transformations is equivalent to the 2-category of cartesian categories. \todo{Verify, but it should
    be true.}
  \end{example}
  \begin{example}\label{ex:fib}
    The theory of fibrations $\Fib$ is generated by a tight morphism $p : \bE \to \bB$, a loose adjunction $
    t \dashv r : \bB \comma p \loose \bE^\to$ with $tr = \id_{\bP \comma p}$ and $t$ the universal morphism
    generated by the cone $\seq{p d, \alpha, c}$ with $\alpha : d \to c : E^\to \to E$. If the unit of the
    adjunction $\eta : 1_{p \comma \bB} \to rt$ is further specified to be the identity, so that $rt = \id_%
    {\bP^\to}$, we have the theory of discrete fibrations.
  \end{example}

  \ifSubfilesClassLoaded{\printbibliography}{}
\end{document}
