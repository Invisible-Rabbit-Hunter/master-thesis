\documentclass[../thesis.tex]{subfiles}

\ifSubfilesClassLoaded{
  \externaldocument{../build/2-preliminaries}%
  \externaldocument{../build/3-logics}%
}

\begin{document}
\chapter{Type Theories}
Throughout the literature, a whole host of different categorical structure have been introduced to model
type dependency and type theories, ranging from display map categories to comprehension categories to
categories with families and natural models. A good overview of the relationship between these different
structures can be found in \cite{ahrens2024}. The goal of this chapter is to show how such structures in
can in fact be described as finite 2-limit theories. Since there are so many different models, and covering
them all would be both time consuming and reduntant, we will focus only on two sorts of models:
\begin{enumerate}
  \item Comprehension categories and categories with attributes. The former was introduced by \citeauthor{jacobs1993}
    in \cite{jacobs1993} as a minimal categorical model for type dependency, and was there also shown to generalize
    the latter. The latter was introduced by \citeauthor{cartmell1984} in \cite{cartmell1984}.
  \item Categories with Families and Natural Models, which are really two different presentations of the same data.
    The former were introduced by \citeauthor{dybjer1996} as an alternative to categories with attributes which was
    closer to the syntax of type theory, and the latter introduced by \citeauthor{awodey2017} as a reformulation of
    categories with families to make formulating the rules of type theory easier in terms of universal properties.
\end{enumerate}

In essence, what we aim to show here is that models of type theories can be identified with 2-functors from a
2-category describing the theory into $\Cat$, and that the homomorphisms can be identified certain pseudo-natural
transformations between these functors. This is similar in spirit to \cite{uemura2023}, where \citeauthor{uemura2023}
introduced categories with representable maps for a similar aim. The idea there was that, since representable maps of
discrete fibrations are natural models, we can describe our type theory abstractly as a finitely complete
category equipped with an additional class of maps closed under certain constructions mimicing that of
the representable maps. Then a model is category of contexts $\cC$ and a finite limit preserving functor
into the discrete fibrations over $\cC$ mapping the representable maps of the theory to representable
maps of fibrations. 

In contrast, our here theories are finitely $\cF$-complete $\cF$-categories, the models are $\cF$-functors into
$\Cat^+$ preserving finite $\cF$-limits, and the homomorphisms are certain transformatons. The additional data
$\cF$-categories over 2-categories allows us finer control over the strictness of homomorphisms, which is what
allows us to recover the standard notions of homomorphisms of, for example, comprehension categories.
Note that an $\cF$-limit theory is a 2-theory, but the additional data of tight morphisms in $\cF$-categories allows
us fine control of the strictness of the homomorphisms. We will use this to show how we can recover the standard
notions of homomorphism of models, such as the pseudo morphisms of comprehension categories, just from the $\cF$-limit
theory. To further show the utility of this approach, we further show how to model more complex features of type
theory, such as inductive types and the modes and modal types of multimode type theory. Note that the latter is
not readily modeled in categories with representable maps, due to the requirement of multiple categories of contexts,
one for each mode.

\section{Comprehension Categories and Categories with Attributes}
The notion of comprehension categories was introduced in \cite{jacobs1993} to provide a minimal categorical
description of type dependency. They concist of a commutative triangle in $\Cat$ of the following form:
\[\begin{codi}[hexagonal=horizontal side 6em angle 45] 
  \obj{\cU &   \cC^\to \\
           & \cC \\};
  \mor  cU \chi:-> (cC ^to) \cod:-> cC;
  \mor[swap] cU u:-> cC;
\end{codi}\]
where $u: \cU \to \cC$ is a fibration, $\cC^\to$ is the arrow category of $\cC$, and $\chi$ preserves cartesian
morphisms. As a model of type theory, we want to think of the fiber $u_\Gamma$ of $u$ at a context $\Gamma \in
\cC$ as the category of types in the context $\Gamma$, so the objects are types $\Gamma \vdash A\;\cU$. The
reindexing provided by $u$ being a fibration then models the substitution of types, and the comprehension map
$\chi$ models context extension, with the preservation of cartesian morphisms giving the expected universal
property thereof, in the sense that for any type $\Gamma \vdash A\;\cU$ and substitution $\sigma : \Delta \to
\Gamma$, the following is a pullback square in $\cC$:
\[\begin{codi}
  \obj { \Delta.A[\sigma] & \Gamma.A \\ \Delta & \Gamma \\ };
  \mor (Delta A[sigma ]) \bar \sigma.A:-> (Gamma A) \pi A:-> Gamma;
  \mor[swap] (Delta A[sigma ]) {{\pi A[\sigma]}}:-> Delta \sigma:-> Gamma;
\end{codi}\]
where we write $\pi$ for the 2-cell $\lambda\chi : \dom\chi \to u$ induced by $\chi$.

To axiomatise this as a finite $\cF$-limit theory, we first consider the 2-categorical structure and limits that
are relevant. We already have a characterisation of fibrations in terms of finite 2-limits, and since $\chi$ is
supposed to preserve cartesian morphisms, it suffices to only ask for preservation of the cartesian lifts; any
cartesian morphism is, after all, isomorphic to a cartesian lift. This can be captured by asking that $\chi\ell$
is cartesian, where $\ell$ is the generic cartesian lift, or cleavage, specified by $u$ being a fibration. We can
thus see that all data in the structure of a comprehension category can be described in terms of finite 2-limits.
Next we consider the homomorphisms. In general, maps of fibrations are strict commuting squares, which tells us
that in our theory, $u$ should be a tight map. By simlar reasoning, we find that $\cod$ also should be tight, which
is further supported by taking $\cC^\to$ to be the corresponding tight limit, for which the codomain projection is
necessarily tight. However, there
is no reason to ask that homomorphisms preserve context extension strictly; it suffices to ask for preservation
only up to isomorphism. This leads us to the following $\cF$-sketch for comprehension categories:
\begin{definition}
  We define $\Comp$ to be the finite $\cF$-limit theory generated by sketch consisting of diagrams
  \begin{mathpar}
    \hspace*{-2.5em}\begin{codi}[hexagonal=horizontal side 7em angle 50] 
      \obj{\cU & \cC^\to \\ & \cC \\};
      \mor  cU \chi:≈> (cC ^to) \cod_\cC:-> cC;
      \mor[swap] cU u:-> cC;
    \end{codi}
    \and
    \begin{codi}
      \obj{\cC^\to \\ \cC\\ };
      \mor[swap]:[bend right=36mu] (cC ^to) \dom_\cC:-> cC;
      \mor :[bend left=36mu] * \cod_\cC:-> *;
      \mor (dom _cC) \lambda_\cC:-2> (cod _cC);
    \end{codi}
    \and
    \begin{codi} 
      \obj{\cC \comma u & \cU \\ |(cC1)| \cC & |(cC2)| \cC \\};
      \mor (cC comma u) \cod_\cC^u:-> cU u:-> cC2;
      \mor[swap] * \dom_\cC^u:-> cC1 1_\cC:-> *;
      \mor cC1 \lambda_\cC^u:-2> cU;
    \end{codi}
    \and
    \begin{codi}[tetragonal=base 6em height 5em] 
      \obj{|(A)| \cC \comma u \\ |(B)| \cU \\};
      \mor[swap]:[bend right] A r:≈> B;
      \mor:[bend left] A ["\cod^u_\cC",name=cod]:-> B;
      \mor[] r \ell:[-2>, shorten=0.4em] cod;
    \end{codi}
    \and
    \begin{codi}[tetragonal=base 7em height 4.5em]
      \obj{|(A)| \cU \comma r & |(B)| \cC \comma u \\ |(C)| \cU & |(D)| \cU \\};
      \mor A \cod_\cU^r:-> B r:≈> D;
      \mor[swap] * \dom_\cU^r:-> C 1_\cU:-> *;
      \mor C \lambda_\cU^r:-2> B;
    \end{codi}
    \and
    \begin{codi}[tetragonal=base 7em height 4.5em]
      \obj{|(A)| \cC^\to \comma \chi r & |(B)| \cC \comma u \\ |(C)| \cC^\to & |(D)| \cC^\to \\};
      \mor A {{\cod_{\cC^\to}^{\chi r}}}:-> B \chi r:≈> D;
      \mor[swap] * {{\dom_{\cC^\to}^{\chi r}}}:-> C 1_\cU:-> *;
      \mor C {{\lambda_{\cC^\to}^{\chi r}}}:-2> B;
    \end{codi}
    \and
    \begin{codi}[tetragonal=base 8em height 4.5em]
      \obj { |(Y)| \cU \comma r & |(X)| \cC \comma u & |(A)| \cU \\ & & |(B)| \cC \\};
      \mor X [mid, "\cod^u_\cC",name=x]:[->,bend left] A ["u",name=p]:-> B;
      \mor X [mid,"r",name=y,swap]:[≈>,bend right] A;
      \mor y \ell:[-2>,shorten=-0.2em, slide=0.3em] x;
      \mor Y ["\dom^r_\cU",name=a]:[->, bend left=5em] A;
      \mor Y ["\cod^r_\cU",name=g]:-> X;
      \mor Y ["u\dom^r_\cU",name=pa,swap]:[->, bend right=2em] B;
      \mor[swap] a {{\ell\cod^r_\cU \circ \lambda^r_\cU}}:[-2>,slide=0.1em] X;
      \mor[swap] pa {{u\lambda^r_\cU}}:[-2>,slide=0.4em, shorten=0.4em] X;
    \end{codi}
    \and
    \begin{codi}[tetragonal=base 8em height 4.5em]
      \obj { |(Y)| \cC^\to \comma \chi r & |(X)| \cC \comma u & |(A)| {\cC^\to} \\ & & |(B)| \cC \\};
      \mor X [mid, "{\chi\cod^r_\cU}",name=x]:[≈>,bend left] A ["\cod_\cC",name=p]:-> B;
      \mor X [mid,"\chi r",name=y,swap]:[≈>,bend right] A;
      \mor y \chi\ell:[-2>,shorten=-0.2em, slide=0.3em] x;
      \mor Y ["{\dom^{\chi r}_{\cC^\to}}",name=a]:[->, bend left=5em] A;
      \mor Y ["{\cod^{\chi r}_{\cC^\to}}",name=g]:-> X;
      \mor Y ["{\cod_\cC\dom^{\chi r}_{\cC^\to}}",name=pa,swap]:[->, bend right=2em] B;
      \mor[swap] a {{\chi\ell\cod^{\chi r}_{\cC^\to} \circ \lambda^{\chi r}_{\cC^\to}}}:[-2>,slide=0.1em] X;
      \mor[swap] pa {{\cod_\cC \lambda^{\chi r}_{\cC^\to}}}:[-2>,slide=0.4em, shorten=0.4em] X;
    \end{codi}
  \end{mathpar}
  subject to the equations
  \begin{mathpar}
    u = \cod_\cC \chi \and
    ur = \dom^u_\cC \and
    u\ell = \lambda^u_\cC \and
  \end{mathpar}
  and with limit cones those making $\cC^\to$ a cotensors with the tight arrow category $2$, $\cC \comma u$ the
  comma object of $1_\cC$ and $u$, $\cU \comma r$ the comma object of $1_\cU$ and $r$, $\cC^\to \comma \chi r$
  the comma object of $\cC^\to \comma \chi r$, and the cones making $\ell$ and $\chi \ell$ cartesian morphisms,
  i.e. the last two diagrams above.
\end{definition}
\begin{remark}
  In this definition, we do not require homomorphisms to preserve cartesian lifts strictly, since the lifting map
  $r : \cC \comma u \to \cU$ is loose. This aligns with the traditional definition of maps of fibrations, where
  we only ask for preservation of cartesian maps, and so only that the image of a carteisan lift under the
  homomorphisms is isomorphic to a cartesian lift in the codomain. However, if we choose to make $r$ tight
  isntead, the homomorphisms to strictly preserve the choice of cartesian lifts, and so in the language of
  models of type theory, to strictly preserve substitution.

  This shows the flexibility that $\cF$-limit theories provide in the 2-categories of models: we have precise control
  over what parts of the theory are preserved strictly by homomorphisms, and what parts are only up to isomorphism,
  without ad hoc definitions for every possible theory we could come up with.
\end{remark}

A model for $\Comp$ in $\Cat$ is exactly a (cloven) comprehension category, and given two comprehension categories
$(\cC, \cU, u, \chi)$ and $(\cD, \cV, v, \psi)$, a homomorphism $f : (\cC, \cU, u, \chi) \to (\cD, \cV, v, \psi)$
is essentially a diagram
\[\begin{codi}[hexagonal=vertical side 6.5em angle 60]
  \obj {\cU & \cC^\to \\
            & \cC     \\};

  \obj [right=5em] {\cV & \cD^\to \\
                        & \cD     \\};
  
  \mor cC u:<- cU \chi:-> (cC ^to) ["\cod_\cC",near end]:-> cC;
  \mor cD v:<- cV \psi:-> (cD ^to) \cod_\cD:-> cD;
  \mor (cC ^to) f_b^\to:-> (cD ^to);
  \mor cC ["f_b",swap]:-> cD;
  \mor cU f_e:[->,crossing over] cV;
  \node at ($(cC ^to)!0.5!(cV)$) {$\cong$};
\end{codi}\]
where all sides except the top one are commuting squares, and the top one is a coherent isomorphism, and $(f_e,
f_b)$ is a map of fibrations. Finally, we see that the 2-cells $\alpha : f \to g$ in the 2-category of
pseudomorphisms of comprehension categories is a pair of natural transformations $\alpha_b : f_b \to g_b$ and
$\alpha_e : f_e \to g_e$ compatible with the top isomorphism in a suitable manner. In this way we

Using $t : \cC \comma u \to \cU^\to$ denote the morphism induced by $u\lambda_\cU : u\dom_\cU \to u\cod_\cU$,
we see that $u$ is a discrete fibration if $rt = \dom^u_\cC$ and $\ell t = 1_{1_\cU}$, capturing that the lift
of the image of a generic morphism in $\cU$ under $u$ is that same generic morphism. Adding this morphism $t$
and the comma object and cone for $\cU^\to$ as well as these equations to the sketch for comprehension categories,
we then specify the comprehension categories whose fibration of types are discrete, also known as categories with
attributes. We denote the theory generated by this augmeneted sketch by $\Attr$, the theory of categories with
attributes.

\begin{lemma}
  Let $A, B : X \to \cU$ with $uA = uB$ be given. Then there is a correspondence between maps
  $\alpha : \set{A} \to \set{B}$ such that $\pi B \alpha = \pi A$ and maps $\beta : \set{A} \to
  \set{B[\pi A]}$ such that $\pi(B[\pi A]) \circ \beta =  1_{A}$.
\end{lemma}
\begin{proof}
  Consider the diagram
  \[\begin{codi}[tetragonal=base 6em height 4.5em]
    \obj {|(cA)| \set{A} \\ & |(cBpiA)|\set{B[\pi A]} & |(cB)| \set{B} \\
                            & |(cA2)| \set{A} & uB \\
    };
    \mor cA \alpha:[bend left=5em,->,dotted] cB;
    \mor[swap] cA {{1_{\set{A}}}}:[bend right=5em,->] cA2;
    \mor cBpiA {{\set*{\vphantom{\big\vert}\overline{\pi A}}}}:-> cB \pi B:-> uB;
    \mor[swap] cBpiA {{\pi B[\pi A]}}:-> cA2 \pi A:-> uB;
    \mor cA \beta:->,dashed cBpiA;
  \end{codi}\]
  Recall that the square to the bottom right is a pullback square, so that given $\alpha$, such a map $\beta$
  is uniquely determined by the universal property of pullbacks. In the other direction, any map $\beta$
  uniquely determines a $\alpha$ by composition with $\set*{\overline{\pi A}}$.
\end{proof}

\newcommand\GNM{\mathrm{GNM}}
\section{Categories with Families}
The second semantics we will fit into our framework is categories with families, which were introduced by
\citeauthor{dybjer1996} in \cite{dybjer1996} as a variant of Cartmell's categories with attributes with
closer ties to the syntax of type theory, in particular modeling terms as a presheaf by themselves, rather
than as a certain class of context maps. A further reformulation of CwFs, found independently by \citeauthor{
fiore2012} \cite{fiore2012} and \citeauthor{awodey2017} \cite{awodey2017}, gives context extension as a
representable map of presheaves between the presheaves of terms the presheaf of types, rather than as a
direct universal property as in \cite{dybjer1996}.  This reformulation, dubbed natural models in \cite{
awodey2017}, allows for one final change to fit into our framework: we replace the presheaves by discrete
fibrations. One advantage of this step is that representability of a map of presheaves is equivalent to the
existence of a right adjoint to the typing map between the categories of elements of the type and term
fibrations (as already noted in \cite{awodey2017}), which is easily axiomatized in our framework.

\begin{definition}[Natural models of type theory]
  A \emph{natural model of type theory}, in the sense of \cite{awodey2017}, is a diagram in $\Cat$ of the
  form
  \[\begin{codi}[hexagonal=horizontal side 6em angle 45] 
    \obj{\dot\cU &   \cU \\ & \cC \\};
    \mor  (dot cU) ["\typ",swap]:-> cU u:-> cC;
    \mor[swap] (dot cU) \dot u:-> cC;
    \mor[swap] cU \var:[bend right,->] (dot cU);
    \mor typ [shorten=0.3em, |-] var;
  \end{codi}\]
  where $u, \dot u$ are discrete fibrations and $\typ$ is a map of discrete fibrations, so that $\typ u = \dot u$.
  Note that $\var$ need not be, and generally is not, a map of fibrations.
\end{definition}
\begin{remark}
  In \cite{coraglia2024a}, the natural models are further generalized by weakening the discrete fibrations to
  arbitrary fibrations, and thus also asking that $\typ$ preserves cartesian morphisms and that the unit and
  counit of the adjunction $\typ \dashv \var$ are cartesian. In \cite{coraglia2024b}, these generalized natural
  models, there dubbed generalized categories with families, are shown to be biequivalent with comprehension
  categories. I conjecture that one can show the $\cF$-theory of these generalized categories with families is
  biequivalent to the one for comprehension categories introduced earlier, with a proof similar to the one found
  in \emph{op.\ cit.} For now we treat only the case with discrete fibrations, as the theory of those is more
  developed.
\end{remark}

As with comprehension categories, we already know how to specify the data for a natural model in an $\cF$-limit
sketch. For example, the 2-cell $u : \cU \to \cC$ is a discrete fibration precisly if the map $t : \cU^\to \to
\cC \comma u$ induced by $u \lambda_\cU : u\dom_\cU \to u\cod_\cU$ has an inverse $r : \cC \comma u \to \cU^\to$
satisfying $\cod_\cU r = \cod_\cC^u$, which is evidently describable by an $\cF$-sketch. We denote the $\cF$-%
theory induced by this sketch by $\NM$.
\begin{remark}
  Note that there are multiple equivalent ways of describing discrete fibrations in terms of finite 2-limits, and
  so multiple $\cF$-sketches for natural models. However, these $\cF$-sketches all induce equivalent $\cF$-theories,
  and so we speak of \emph{the} theory of natural models $\NM$.
\end{remark}

With the basic theory of natural models at hand, we will spend the remainder of this section showing how further
structure and properties, in particular dependent sum and product types, unit types, and identity types, can
be similarly described in terms of finite 2-limits. This allows us to construct finite $\cF$-limit theories
for any type theory with these features, whose models are the natural models with these features. The specific
constructions we use are based on \cite{coraglia2024a}, rather than \cite{awodey2017}, as we have free access
to 2-pullbacks and similar 2-limits, but not the representable discrete fibrations.

Before we can show how to model the rules for these different type constructors, we will need a few useful definitions.
First, we will need a the map corresponding to context extension. In analogy to comprehension categories, we will denote
this morphism by $\set{-} : \cU \to \cC$, and we define it by $\set{-} \coloneq \dot u \var$. The counit of the adjunction
$\typ \dashv \var$ then gives us the expected projection map $\pi \coloneq u \epsilon : \set{-} = u\typ\var \to u$. We also
want to relate the terms $\dot\cU$ to substitutions. Specifically, we have that every term $t : X \to \dot\cU$ with
$A \coloneq \typ t$ has a corresponding substitution $\id_{u A},t \coloneq \dot u \eta t : uA = \dot u t \to \dot u \var\typ t =
\set{A}$, and this substitution satisfies
\[\pi A \circ \id_{uA},t = u(\epsilon\var \circ \typ \eta \circ)t = 1_{u A},\]
so that $\id_{uA},t$ is a section of $\pi A$, which is expected of the substitution corresponding to a term, and also motivates
the chosen notation.

\subsubsection{Dependent products}
Recall the formation and introduction rules for dependent product types:
\begin{mathpar}
  \inferrule{\Gamma \vdash A\;\cU \\ \Gamma.A \vdash B\;\cU}{\Gamma \vdash \Pi(A, B)\;\cU} \and
  \inferrule{\Gamma \vdash A\;\cU \\ \Gamma.A \vdash b : B}{\Gamma \vdash \lambda(A,b) : \Pi(A,B)}
\end{mathpar}
The formation rule tells us that given any type $A$ and type $B$ dependent on $A$, we can form the product type
$\Pi(A,B)$. We model this as a morphism from some premise judgement $P$, corresponding to the pair $A$ and $A \vdash B$,
to the judgement of types $\cU$. More formally, the premise is a judgement $p : P \to \cC$ over contexts, since $A$ and
$B$ are types lying over some context, and the formation rule is a map $\Pi : P \to \cU$ such that $u\Pi = p$. To see
what the premise $P$ should be, note that a type $B$ is dependent over $A$ if $uB = \set{A}$, so we should take $P$
to be the pullback
\[\begin{codi}[tetragonal=base 5em height 4.5em]
  \obj {|(tl)| \cU \pb[\set{-}]{u} \cU & |(tr)| \cU \\ |(bl)| \cU & |(br)| \cC \\};
  \mor[swap] tl A_F:-> bl {{\set{-}}}:-> br;
  \mor tl B_F:-> tr u:-> br;
\end{codi}\]
with $p = uA_F$ picking the underlying context $\Gamma$ of $A$. In this way, a generalized element $H : X \to P$ is the same
as a pair $H_1, H_2 : X \to \cU$ with $uH_2 = \set{H_1}$, as desired, vice versa. In particular, the formation rule then
says that given any pair of types $\alpha, \beta : X \to \cU$ with $u\beta = \set{\alpha}$, we have $\Pi(\alpha,\beta) :
X \to \cU$, as expected.

Similarly, the introduction rule tells us that, given a type $A$ and a term $A \vdash b : B$, we have a term
$\lambda(A,b) : \Pi(A,B)$. This rule is also captured by a morphism, this time of the form $\lambda : P' \to \dot\cU$ for
an appropriate $P$. This time $P$ consists of a pair of a type $A$ and a term $b$, such that $\dot u b = \set{A}$, and
so is captured by the pullback
\[\begin{codi}[tetragonal=base 5em height 4.5em]
  \obj {|(tl)| \cU \pb[\set{-}]{\dot u} \dot \cU & |(tr)| \dot \cU \\ |(bl)| \cU & |(br)| \cC \\};
  \mor[swap] tl A':-> bl {{\set{-}}}:-> br;
  \mor tl b:-> tr \dot u:-> br;
\end{codi}\]
To capture the typing of the $\lambda$-term, we need to ask that the following square commutes:
\[\begin{codi}[tetragonal=base 6em height 4.5em]
  \obj {|(tl)| \cU \pb[\set{-}]{\dot u} \dot\cU & |(tr)| \dot \cU \\ |(bl)| \cU \pb[\set{-}]{u} \cU & |(br)| \cU \\};
  \mor tl \lambda:-> tr \typ:-> br;
  \mor[swap] tl {{(A', \typ b)}}:-> bl \prod:-> br;
\end{codi}\]
where $(A',\typ B)$ is the unique map into the pullback $\cU \pb[\set{-}]{u} \cU$ given by the equation $\set{A'} = \dot u b =
u\typ b$ holding.

Finally, using the fact that terms of a dependent product $\Gamma \vdash t : \Pi(A,B)$ ought to be in 1-1 correspondence with
terms $\Gamma . A \vdash t' : B$, we see that asking that the above square be a pullback gives us the expected inverse of the
lambda rule; A term $t : X \to \dot\cU$ with $\typ t = \Pi(\alpha,\beta)$ for some adequate $\alpha,\beta : X \to \cU$, we have
a unique map $\bar t : X \to \cU \pb[\set{-}]{\dot u} \dot \cU$ with $\lambda \bar t = t$, and vice versa. To capture the
elimantion rule
\[\inferrule{\Gamma \vdash t : \Pi(A,B) \\ \Gamma \vdash u : A}{\Gamma\vdash t\cdot u : B[\id_\Gamma,u]},\]
we see that the definition $t\cdot u \coloneq \bar t [\id_\Gamma,u]$ has the expected type, and by this definition for any
term $\bar t : X \to \cU \pb[\set{-}]{\dot u}$, we find that $\lambda(\bar t)\cdot u = \bar t[\id_\Gamma,u]$, so that
we validate the computation rule for dependent product types as well.

Putting all of this together, we have the following definition:
\begin{definition}
  A natural model $(\cC, \cU, \dot\cU)$ has dependent products if there exists a pullback square of the form
  \[\begin{codi}[tetragonal=base 6em height 4.5em]
    \obj {|(tl)| \cU \pb[\set{-}]{\dot u} \dot\cU & |(tr)| \dot \cU \\ |(bl)| \cU \pb[\set{-}]{u} \cU & |(br)| \cU \\};
    \mor tl \lambda:-> tr \typ:-> br;
    \mor[swap] tl {{(A', \typ b)}}:-> bl \prod:-> br;
  \end{codi}\]
\end{definition}

Since all data here is describable by finite 2-limits, we can simply add this to the sketch for natural models to obtain
the theory of natural models with dependent product types.

\subsubsection{Dependent sums}
As with dependent product types, the type former for dependent sum types is modeled by a morphism $\Sigma : \cU \pb[\set{-}]{u}
\cU \to \cU$. However, the introduction rule is more complicated:
\[\inferrule{\Gamma \vdash t : A \\ \Gamma \vdash u : B[\id_\Gamma, t]}{\Gamma \vdash \seq{t,u} : \Sigma(A,B)}\]
In particular, the premise requires a term $u$ depending on the \emph{substitution} of a term $t$. We construct the
object corresponding to the premise in steps. First we take

\begin{definition}
  A natural model $(\cC, \cU, \dot\cU)$ has depedent sums if there exists a pullback square of the form
  \[\begin{codi}[tetragonal=base 6em height 4.5em]
    \obj {|(tl)| \dot\cU \pb[\set{-}\typ]{\dot u} \dot\cU & |(tr)| \dot \cU \\ |(bl)| \cU \pb[\set{-}]{u} \cU & |(br)|
        \cU \\};
    \mor tl p:-> tr \typ:-> br;
    \mor[swap] tl {{\cU \pb[\set{-}]{\typ} \typ}}:-> bl \prod:-> br;
  \end{codi}\]
\end{definition}
\todo{Finish writing this subsubsection.}

\subsubsection{Extensional identity types}
Let $\delta : \cU \to \cU$ denote the lift of $\pi : \set{-} \to u$, so that for an arbitrary type $A : X \to \cU$, we have
that $\delta A = A[\pi A]$ is the weaking of $A$. The dependent 
\todo{Finish writing this subsubsection.}



% Similarly to comprehension categories, natural models are defined as certain fibrations over a base category.
% Formally, they consist of a diagram in $\Cat$ of the shape:
% \[\begin{codi}[hexagonal=horizontal side 6em angle 45] 
%   \obj{\dot\cU &   \cU \\ & \cC \\};
%   \mor  (dot cU) ["\typ",swap]:-> cU u:-> cC;
%   \mor[swap] (dot cU) \dot u:-> cC;
%   \mor[swap] cU \var:[bend right,->] (dot cU);
%   \mor typ [shorten=0.3em, |-] var;
% \end{codi}\]
% where the lower triangle commutes, and $u, \dot u$ are discrete fibrations. These correspond to categories with
% attributes, but can be generalized by asking $u, \dot u$ to be possibly non-discrete fibrations, that $\typ$
% be a morphism of fibrations and $\var$ preserves cartesian morphisms (though is not necessarily a morphism of
% fibrations), and that the unit and counit of the adjunction $\typ \dashv \var$ be $\dot u$-cartesian and
% respectively. We shall call these diagrams generalized natural models, and we will show that they are
% essentially equivalent to comprehension categories. Furthermore, as we have shown with comprehension categories,
% all the data in this diagram can be specified in terms of a $\cF$-sketch, so that we have $\cF$-logics for
% (generalized) natural models. We shall denote these by $\GNM$ and $\NM$ respectively.

% \begin{definition}[{\cite[Definition 3.0.1]{coraglia2024a}}]
%   A generalized natural model is a diagram
%   \[\begin{codi}[hexagonal=horizontal side 6em angle 45] 
%     \obj{\dot\cU &   \cU \\ & \cC \\};
%     \mor  (dot cU) ["\typ",swap]:-> cU u:-> cC;
%     \mor[swap] (dot cU) \dot u:-> cC;
%     \mor[swap] cU \var:[bend right,->] (dot cU);
%     \mor typ [shorten=0.3em, |-] var;
%   \end{codi}\]
%   in $\Cat$ for which $\typ$ and $\var$ preserve cartesian morphisms and the unit and counit of the adjunction
%   $\typ \dashv \var$ are cartesian.
% \end{definition}
%
% \begin{notation}
%   We use $\set{-} : \cU \to \cC$ to denote the composite $\dot u\var$, and call this the comprehension map.
%   Using $\epsilon : \typ\,\var \to 1_{\cU}$ and $\eta : 1_{\dot\cU} \to \var\,\typ$ to denote the unit
%   and counit of the adjunction $\typ \dashv \var$, we use $\pi : \set{-} \to u$ to denote the 2-cell
%   $u\epsilon : \set{-} = \dot u\var = u\,\typ\,\var \to u$, which intuitively corresponds to the projection
%   given by context extension; for $\Gamma$ a context and $\Gamma \vdash A\;\cU$ a type, we have $\pi A :
%   \Gamma.A \to A$.
%
%   We will use $\tau : \dot u \to \dot u\var = \set{\typ}$ to denote $\dot u\eta$, with the idea that this
%   corresponds to the substitution $\Gamma \to \Gamma.A$, for a term $\Gamma \vdash a : A$. Note that we have
%   \[
%     \pi{\typ} \circ \tau = u\epsilon\typ \circ \dot u\tau = u\epsilon\typ \circ u\typ\tau =
%     u(\epsilon\typ \circ \typ\tau) = 1_u,
%   \]
%   where the last equality holds by the one of the triangle identities of the adjunction $\typ \dashv \var$.
%   This notion of terms thus aligns with the standard one of terms being sections of the context projection.
%
%   For $f : A \to \var B : X \to \dot\cU$, we use $f^\sharp$ to denote the adjoint conjugate of $f$, defined by
%   $f^\sharp = \epsilon B \circ \typ f$, and for $g : \typ A \to B$, we denote the adjoint conjungate by
%   $g_\sharp = \var g \circ \eta A$. The triangle identities of the adjuntion implies that $(f^\sharp)_\sharp
%   = f$ and $(g_\sharp)^\sharp = g$. We also have that $u f^\sharp = \pi B \circ \dot u f$ and $\dot u g_\sharp
%   = \set{g} \circ \tau A$.
%
%
%   Finally, we will use $W \coloneq \typ\var$ to denote the comonad induced by the adjunction $\typ \dashv \var$,
%   which has counit $\epsilon : W \to 1_\cU$ and comultiplication $\delta \coloneq \typ\eta\var : W \to W^2$.
% \end{notation}
%
% \begin{proposition}
%   For any $b : X \to \dot\cU$, the arrow $\eta b : b \to \var\typ b$ is monic.
% \end{proposition}
% \begin{proof}
%   Let $f,g : a \to b : X \to \dot\cU$ be given such that $\eta b \circ f = \eta b \circ g$. Then 
%   \[\dot u f = u\typ f = u(\epsilon\typ b \circ \typ\eta b \circ \typ f) = 
%   u(\epsilon\typ b \circ \typ\eta b \circ \typ g) = u\typ g = \dot u g.\]
%   Since $\eta$ is cartesian and $(b,a,\eta b \circ f = \eta b \circ g, \dot u f = \dot u g)$ is a certesian
%   cone with both $f,g$ universal arrows, uniqueness of th euniversal arrow tells us that $f = g$.
% \end{proof}
%
% \begin{proposition}
%   The map $\typ : \dot\cU \to \cU$ is faithful, in the sense that for any pair of 2-cells $f,g : a \to b
%   : X \to \dot\cU$, if $\typ f = \typ g$, then $f = g$. 
% \end{proposition}
% \begin{proof}
%   Note that
%   \[\eta b \circ f = \var\typ f \circ \eta a = \var\typ g \circ \eta a = \eta b \circ g\]
%   by the interchange property, so by the previous proposition we conclude $f = g$.
% \end{proof}

% \begin{proposition}
%   For any 2-cell $f : A \to B : X \to \cU$, if $f$ is $u$-cartesian, then $\var f$ is $\dot u$-cartesian, and
%   for any morphism $\dot f : a \to b : X \to \dot \cU$, if $\typ \dot f$ is $u$-cartesian, then $\dot f$ is
%   $\dot u$-cartesian as well.
% \end{proposition}
% \begin{proof}
%   For the first part, suppose we are given a $\dot u$-cartesian cone over $\var f$:
%   \[\begin{codi}[tetragonal=base 5.5em height 3.5em]
%     \obj { |(Y)| Y & |(X)| X & |(A)| \dot\cU \\ & & |(B)| \cC \\};
%     \mor X ["\var B",name=x]:[->,bend left] A ["\dot u",name=p]:-> B;
%     \mor X ["\var A",name=y,swap]:[->,bend right] A;
%     \mor y \var f:[-2>] x;
%     \mor Y ["c",name=a]:[->, bend left=5em] A;
%     \mor Y ["g",name=g]:-> X;
%     \mor Y ["\dot u c",name=pa,swap]:[->, bend right=2em] B;
%     \mor[swap] a {{k}}:[-2>,slide=0.1em] X;
%     \mor[swap] pa {{l}}:[-2>,slide=0.4em, shorten=0.4em] X;
%   \end{codi}\]
%   We need to construct a unique map $h : c \to \var A g$ such that $\var f g \circ h = k$ and $\dot u h =
%   l$. Note that $\typ\var f$ is $u$-cartesian since $\epsilon B \circ \typ\var f = f \circ \epsilon A$ and
%   $\epsilon B$ are cartesian. Since furthermore $\dot u = u \typ$, we find that $u\typ k = u\typ\var fg \circ l$,
%   so there exists a unique map $\gamma : \typ c \to \typ\var B$ such that $\typ\var f g \circ \gamma = \typ k$
%   and $u\gamma = l$. We take $h$ to be the composite
%   \[c \xrightarrow{\eta c} \var\typ c \xrightarrow{\var \gamma} \var\typ\var A g \xrightarrow{\var\epsilon Ag}
%     \var Ag.\]
%   Then we have that
%   \begin{align*}
%     \typ h & = \typ(\var(\epsilon A g \circ \gamma) \circ \eta c) \\
%            & = \typ\var\epsilon A g \circ \typ\var\gamma \circ \typ\eta c \\
%            & = d
%   \end{align*}
%   \begin{align*}
%     \var f g \circ h & = \var(fg \circ \epsilon A g \circ \gamma) \circ \eta c \\
%                      & = \var(\epsilon B g \circ \typ\var fg \circ \gamma) \circ \eta c \\
%                      & = \var(\epsilon B g \circ \typ k) \circ \eta c \\
%                      & = \var\epsilon B G \circ \var\typ k \circ \eta c \\
%                      & = \var\epsilon B G \circ \eta\var Ag \circ k \\
%                      & = k
%   \end{align*}
%   Similarly, we have that $\dot u h = u\typ h = u\typ(\var\epsilon A g \circ \var\gamma \circ \eta c)
%   = u(\typ\var\epsilon A g \circ \typ\var\gamma \circ \eta c)
%   = u (\typ\var\epsilon A g \circ \eta A g \circ \gamma)
%   = u (\eta\typ\var Ag \circ $
%   Now note that $\epsilon B \circ \gamma : \typ c \to B$, 
%   .and $\epsilon B$ is cartesian
%   Since $\dot u = u\typ$, we have that $u\typ k = l \circ u\typ\var f g$, so \todo{Finish this}
% \end{proof}

% \begin{lemma}
%   For any $\dot u$-cartesian map $f : a \to \var A : X \to \dot \cU$, its conjugate $f^\sharp$ is $u$-cartesian,
%   and for any $u$-cartesian map $g : \typ a \to A : X \to \cU$, the conjugate $g_\sharp$ is $\dot u$-cartesian.
% \end{lemma}
% \begin{proof}
%   For $f$, note that $f^\sharp = \epsilon A \circ Tf$, so since $\epsilon$ and $f$ are cartesian and $T$
%   preserves cartesian morphisms, it follows that $f^\sharp$ is cartesian as well. A similar argument shows
%   that $g_\sharp$ is cartesian, using that $\var$ preserves cartesian morphisms.
% \end{proof}
%
% \begin{lemma}
%   Let $f : A \to B : X \to \cU$ be a $u$-cartesian morphism. Then the following square is a pullback square:
%   \[\begin{codi}
%     \obj {|(eA)| \set{A} & |(eB)| \set{B} \\ uA & uB \\};
%     \mor eA {{\set{f}}}:-> eB \pi B:-> uB;
%     \mor[swap] eA \pi A:-> uA uf:-> uB;
%   \end{codi}\]
% \end{lemma}
% \begin{proof}
%   Consider another span $uA \xleftarrow{k} \Delta \xrightarrow{l} \set{B}$ such that $uf \circ k = \pi B \circ l$.
%   We must show that there exists a unique map $\Delta \to \set{A}$. Let $\bar l : \bar \Delta \to \var B$ denote a
%   cartesian lift of $l : \Delta \to \dot u \var B$. To construct a map $\Delta \to \set{A}$, it suffices to find
%   a map $h : \bar\Delta \to \var A$, which by the $\typ \dashv \var$ adjunction is equivalent to constructing
%   a map $h^\sharp : \typ\bar\Delta \to A$. Now, consider the cartesian cone
%   \[\begin{codi}[tetragonal=base 5.5em height 3.5em]
%     \obj { |(Y)| X & |(X)| X & |(A)| \dot\cU \\ & & |(B)| \cC \\};
%     \mor X ["B",name=x]:[->,bend left] A ["u",name=p]:-> B;
%     \mor X ["A",name=y,swap]:[->,bend right] A;
%     \mor y f:[-2>] x;
%     \mor Y ["\typ \bar\Delta",name=a]:[->, bend left=5em] A;
%     \mor Y ["1_X",name=g]:-> X;
%     \mor Y ["\Delta",name=pa,swap]:[->, bend right=2em] B;
%     \mor[swap] a {{\bar l^\sharp}}:[-2>,slide=0.1em] X;
%     \mor[swap] pa {{k}}:[-2>,slide=0.2em, shorten=0.2em] X;
%   \end{codi}\]
%   where $\bar l^\sharp$ is the adjoint of $\bar l$. Note that we have
%   \[u(\bar l^\sharp) = u\epsilon B \circ u\typ\bar l = \pi B \circ \dot u\bar l = \pi B \circ l = uf \circ k,\]
%   so that this is indeed a cartesian cone. The universal property of $f$ now gives us a unique map $h^\sharp
%   : \typ\bar\Delta \to A$ such that $uh^\sharp = k$ and $f \circ h^\sharp = \bar l^\sharp$. Furthermore, this
%   map is cartesian, since $f$ and $\bar l^\sharp$ are. We claim that, taking $h = (h^\sharp)_\sharp$, the map
%   $\dot u h$ is the universal map $\Delta \to \set{A}$. Thus we need to show that $\pi A \circ \dot u h =
%   k$ and $\set{f} \circ \dot u h = l$. For the former, we compute
%   \[
%     \pi A \circ \dot u h = \pi A \circ \set{h^\sharp} \circ \tau \bar\Delta
%     = uh^\sharp \circ \pi \bar\Delta \circ \tau\bar\Delta = uh^\sharp = k,
%   \]
%   and for the latter we have that
%   \[
%     \set{f} \circ \dot u h = \set{f \circ h^\sharp} \circ \tau\bar\Delta = \set{\bar l^\sharp} \circ
%     \tau\bar\Delta = u(\bar l^\sharp)_\sharp = u\bar l = l.
%   \]
%   
%   Finally, we must show that $uh$ is unique with these properties, so suppose $h' : \Delta \to \set{A}$ were
%   another map with $\pi A \circ h' = k$ and $\set{f} \circ h' = l$, and let $\bar h' : \bar\Delta' \to \var A$
%   denote a cartesian lift of $h'$. Then $\var f \circ h'$ is cartesian over $l$, so by essential uniqueness of
%   cartesian lifts, there exists a vertical isomorphism $\alpha : \bar\Delta \cong \bar\Delta'$ such that $\var
%   f \circ h' \circ \alpha = \bar l$, and since isomorphisms are cartesian, we find that $h' \circ \alpha : \bar
%   \Delta \to \var A$ is cartesian, and so equal to $h$ by again by uniqueness of cartesian lifts. Therefore $u
%   \bar h' = u (\bar h' \circ \alpha^{-1}) = uh$, as desired.
% \end{proof}
%
% \begin{lemma}
%   For any $f : A \to B : X \to \cU$, the following is a pullback square:
%   \[\begin{codi}
%     \obj {\typ\var A & \typ\var B \\ A & B \\};
%     \mor (typ var A) \typ\var f:-> (typ var B) \epsilon B: -> B;
%     \mor[swap] (typ var A) \epsilon A:-> A f:-> B; 
%   \end{codi}\]
% \end{lemma}
% \begin{proof}
%   By the previous lemma, we have that 
%   \[\begin{codi}
%     \obj {\typ\var A & \typ\var B \\ A & B \\};
%     \mor (typ var A) \typ\var f:-> (typ var B) \epsilon B: -> B;
%     \mor[swap] (typ var A) \epsilon A:-> A f:-> B; 
%   \end{codi}\]
% \end{proof}
%
% \begin{corollary}
%   Let $\delta \coloneq \typ\eta\var : \typ\var \to \typ\var\typ\var$. Then $\delta$ is the unique map such
%   that $\epsilon\typ\var \circ \delta = \typ\var\epsilon \circ \delta = 1_{\typ\var}$.
% \end{corollary}
%
% \begin{corollary}
%   Let $a : A \to \typ\var A : X \to \cU$ be such that $\epsilon A \circ a = 1_A$. 
%   $\delta A \circ a = \typ\var a \circ a$. In particular for any $A : X \to \cU$, the $\typ\var$-coalgebras are
%   exactly the sections of $\epsilon A$.
% \end{corollary}
% \begin{proof}
%   By the previous corollary, we see that $\delta A \circ a$ is the unique map such that
%   \[\epsilon A \circ \delta A \circ a = a = \typ\var \epsilon A \circ \delta A \circ a.\]
%   But we also have that $\epsilon A \circ \typ\var a \circ a = a \circ \epsilon A \circ a = a$ and
%   $\typ\var\epsilon A \circ \typ\var a \circ a = \typ\var(\epsilon A \circ a) \circ a = a$, so
%   we must conclude that $\delta A \circ a = \typ\var a \circ a$.
% \end{proof}
%
% \begin{proposition}
%   Let $\sigma : \Delta \to \Gamma : X \to \cC$, $A : X \to \cU$, and $a : X \to \dot\cU$ with $uA = \Gamma$
%   and $\typ a = A[\sigma]$ be given. Then the map $\var\bar\sigma \circ \eta a$ is cartesian, where $\bar
%   \sigma$ is the cartesian lift of $\sigma$ given by the cleavage. We denote $\dot u(\var\bar\sigma \circ
%   \eta a)$ by $(\sigma,a)$.
% \end{proposition}
% \begin{proof}
%   Since $\var$ preserves cartesian morphisms, $\eta$ and $\bar\sigma$ are cartesian, and cartesian morphisms
%   compose, this us true.
% \end{proof}
% \begin{proposition}
%   The map $\typ : \dot\cU \to \cU$ is comonadic.
% \end{proposition}
% \begin{proof}
%   \todo{This proof is a little more difficult than I expected. Will try with a new approach, showing that $\typ$
%   is a weak Eilenberg-Moore object instead.}
%   First note that $(\typ, \typ\eta)$ has a coalgebra structure: by the definition of the adjunction $\typ \dashv
%   \var$, we have that $\epsilon\typ \circ \typ\eta = 1_\typ$. Letting $W\CoAlg$ denote the Eilenberg-Moore object
%   corresponding to $W$, with forgetful map $U_W : W\CoAlg \to \cU$ and universal algebra $\alpha : U_W \to W U_W$,
%   we have a comparison map $\phi : \dot\cU \to W\CoAlg$ such that $\typ\eta = \alpha\phi$. By definition, $\typ$
%   is comonadic precisely when $\phi$ is an equivalence. 
%
%   Let $\dot r : \dot\cU \comma \dot u \to \dot\cU$ and $\dot \ell : \dot r \to \cod$ denote the cleavage for the
%   fibration $\dot u : \dot\cU \to \cC$. We show that $\phi^{-1} \coloneq \dot r\seq{u\alpha}$ is a pseudoinverse
%   to $\phi$, where $\seq{u\alpha} : W\CoAlg \to \cC \comma \dot u$ is the universal map induced by the 2-cell $u
%   \alpha : uU_W \to uWU_W = \dot u \var U_w$. Thus we require isomorphism 2-cells $\xi : \phi^{-1} \phi \to 1_{
%   \dot \cU}$ and $\zeta : \phi\phi^{-1} \to 1_{W\CoAlg}$.
%
%   Since $\dot\ell\seq{u \alpha}$ is $\dot u$-cartesian and
%   \[\dot u \dot\ell\seq{u\alpha}\phi = u\alpha\phi = u T\eta = \dot u \eta,\]
%   there exists a unique map $\xi^{-1} : 1_{\dot\cU} \to \phi^{-1}\phi$ such that $\dot u \xi^{-1} = 1_{\dot u}$
%   and $\dot\ell\seq{u\alpha}\phi \circ \xi^{-1} = \eta$. Symmetrically, since $\eta$ is $\dot u$-cartesian, there
%   exists a unique map $\xi : \phi^{-1}\phi \to 1_{\dot\cU}$ such that $\dot u \xi = 1_{\dot u}$. Furthermore,
%   the universal properties of cartesian morphisms tells us that $\xi^{-1}\xi$ and $\xi\xi^{-1}$ are inverses,
%   so that $\xi$ is indeed an isomorphism 2-cell.
%
% \end{proof}
%
% \begin{corollary}
%   The following is a bipullback square in $\NM$:
%   \[\begin{codi}
%     \obj {\dot\cU & \cC^S \\ \cU & \cC^\to \\};
%     \mor (dot cU) {{(\tau,\pi\typ)}}:-> (cC ^S) \cC^g:-> (cC ^to);
%     \mor[swap] (dot cU) \typ:-> cU (\pi):-> (cC ^to);
%   \end{codi}\]
%   where $\cC^S$ is the cotensor with $S$ the walking section. Concretely, $S$ consists of two tight objects
%   $s,t \in S$ and morphisms $f : x \to y$ and $g : y \to x$ with $gf = 1_x$.  
% \end{corollary}
% \begin{proof}
%   Suppose we are given $A : X \to \cU$ and $\sigma : X \to \cC^S$ such that $(\pi)A = \cC^g \sigma$. By the
%   universal property of $\cC^S$, we know that $s = (t, p)$ with $t : \Gamma \leftrightarrow \Delta : p$ and
%   $p t = 1_{\Gamma}$. From $(\pi)A = \cC^g \sigma$, we find that $\Gamma = uA$, $\Delta = \set{A} = \Gamma.A$
%   and $p = \pi A$. By the previous proposition we then have a unique term $\bar t : X \to \dot\cU$ with
%   $\typ \bar t = A$ and $\tau\bar t = t$. In particular, we 
% \end{proof}
% \todo{I think this is true, though as a bilimit rather than a strict 2-pullback.}
%
% \begin{proposition}
%   Let $A : X \to \cU$ be a type and $\sigma : uA \to \set{A}$ be a section to $\pi A : \set{A} \to uA$. Then
%   there exists a term $a : X \to \dot\cU$, unique up to isomorphism, with $\typ a \cong A$ such that
%   $\sigma = \tau a$.
% \end{proposition}
% \begin{proof}
%   We show that $a \coloneq (\var A)[\sigma]$ suffices. For ease of notation, let $\Gamma \coloneq \set{A}$. To see
%   that $\typ a \cong A$, it suffices to show that there exists a cartesian map $A \to \typ(\var A)$ over $\sigma$.
%   Let $\bar\sigma : (\var A)[\sigma] \to \var A$ be the $\dot u$-cartesian lift of $\sigma : uA \to \dot u V A$.
%   Then we have a cartesian cone
%   \[\begin{codi}[tetragonal=base 5.5em height 3.5em]
%     \obj { |(Y)| X & |(X)| X & |(A)| \dot\cU \\ & & |(B)| \cC \\};
%     \mor X ["\var A",name=x]:[->,bend left] A ["u",name=p]:-> B;
%     \mor X ["\var A",name=y,swap]:[->,bend right] A;
%     \mor y f:[-2>] x;
%     \mor Y ["\var A",name=a]:[->, bend left=5em] A;
%     \mor Y ["1_X",name=g]:-> X;
%     \mor Y ["\Delta",name=pa,swap]:[->, bend right=2em] B;
%     \mor[swap] a {{\bar l^\sharp}}:[-2>,slide=0.1em] X;
%     \mor[swap] pa {{k}}:[-2>,slide=0.2em, shorten=0.2em] X;
%   \end{codi}\]
% \end{proof}
%
% \begin{proposition}
%   The logics $\GNM$ (resp. $\NM$) and $\Comp$ (resp. $\Attr$) are biequivalent as $\cF$-categories.
% \end{proposition}
% \todo{Go through all notation to make sure it's consistent. (Currently it is most definitely not.)}
% \todo{Consider going through weakening-contraction comonads?}
% \begin{proof}
%   A biequivalence of $\cF$-categories is .....
%
%   We first construct the $\cF$-functor $F : \Comp \to \GNM$, for which it suffices to construct a model of
%   $\Comp$ inside $\GNM$. We keep the fibration of types the same, so we only need to find a map $\chi : \cU \to
%   \cC^\to$ such that $\chi^\to \ell$ is $\cod$-cartesian, where $\ell : r \to \cod : \cC \comma u \to \cU$ is the
%   cleaveage for $u$. Let $\chi$ be the universal map induced by $\pi : \set{-} \to u$. To see that $\chi^\to\ell$
%   is $\cod$-cartesian, it suffices to show that the square
%   \[\begin{codi}
%     \obj {|(r)| \set{r} & |(cod)| \set{\cod} \\ \dom & u\cod \\};
%     \mor r \set{\ell}:-> cod \pi\cod:-> ucod;
%     \mor[swap] r \pi r:-> dom \lambda:-> ucod;
%   \end{codi}\]
%   is a pullback.
%
%   In the other direction we require a $\cF$-functor $G : \NM \to \Attr$, so we construct a model of $\NM$
%   inside $\Attr$. As this construction is quite a bit more involved, we will start by laying out some basic
%   notation. Since we will deal with both the arrow object $\cC^\to$ and the comma object $\cC \comma u$, we
%   need to distinguish their universal cones. We write $\lambda : \dom \to u\cod : \cC \comma u \to \cC$ for
%   the universal cone belonging to $\cC\comma u$ and $\lambda' : \dom' \to \cod' : \cC^\to \to \cC$ for the
%   universal cone for $\cC^\to$. We write $e \coloneq \dom'\chi$ and $q \coloneq \lambda' \chi : e \to u :
%   \cU \to \cC$, and for a generalized element $A : X \to \cU$, will write $\Gamma \vdash A$ to denote that
%   $\Gamma \coloneq uA$. Furthermore, we write $\Gamma.A$ for $eA$. By the construction of $\Attr$, we have
%   a fixed cleavage $(r,\ell)$ showing that $u$ is a fibration, where $r : \cC \comma u \to \cU$ is such
%   that $ur = \dom$ and $\ell : r \to \cod : \cC\comma u \to \cU$ is $u$-cartesian with $u\ell = \lambda$.
%   Furthermore, we have that $\chi\ell : \chi r \to \chi\cod$ is $\cod$-cartesian, since that is how we capture
%   that $\chi$ preserves cartesian-ness of the cartesian lifts for $u$.
%
%   We will start with constructing the object of terms $\dot\cU$. Intuitively, the terms over a context
%   $\Gamma : X \to \cC$ ought to be sections to some context projection $q A$ for some $\Gamma \vdash A$,
%   and we encode this as a pullback. First we need an object of sections, however. Let $S$ denote the tight
%   walking section (an object of $\cF$), so it has two tight objects $s,t \in S$ and morphisms $f : x \to y$
%   and $g : y \to x$ with $gf = 1_x$. The power object $\cC^S$ consists of sections in $\cC$, in the sense
%   that any map $X \to \cC^S$ is the same data as a section in the $\cF$-category $X \to \cC$. Now the object
%   of terms is defined as the pullback
%   \[\begin{codi}
%     \obj {\dot\cU & \cC^S \\ \cU & \cC^\to \\};
%     \mor (dot cU) p:-> (cC ^S) \cC^g:-> (cC ^to);
%     \mor[swap] (dot cU) \typ:-> cU \chi:-> (cC ^to);
%   \end{codi}\]
%   where $\cC^g : \cC^S \to \cC^\to$ is the projection induced by inclusion $g : 2 \to S$ in $\cF$. We next
%   define $\dot u = u\typ$, as that is a required equation for $\dot\cU$ to be a fibration of terms in
%   a natural model. We will use $\Gamma \vdash a : A$ to denote an generalized element $a : X \to \dot\cU$,
%   with $A = \typ a$ and $\Gamma = uA$. Thus if we have an element $\Gamma \vdash a : A$, we also have
%   $\Gamma \vdash A$ and $\Gamma : X \to \cC$.
%
%   To show that $\dot u$ is a fibration, we need for every $\Gamma \vdash a : A$ and $\typ : \var \to
%   \Gamma : X \to \cC$ a term $\var \vdash a[\typ] : A[\typ]$, where $\Gamma \vdash A[\typ]$ is
%   $r(\typ,A)$. Using $\bar a : \Gamma \to \Gamma.A$ to denote the section of $q\typ$ induced by $p$,
%   we see that we require a section $\bar a[\typ] : \var \to \var.A[\typ]$ to $qA[\typ]$.
%   Since $\chi\epsilon(\typ,A) : \var.A[\typ] \to \Gamma.A$ is cartesian, we find that the following
%   square is a pullback:
%   \[\begin{codi}[tetragonal=base 8em height 4.5em]
%     \obj {|(tl)| \var.A[\typ] & |(tr)| \Gamma.A \\ |(bl)| \var & |(br)| \Gamma\\};
%     \mor tl {{e\epsilon(\typ,A)}}:-> tr qA:-> br;
%     \mor[swap] tl {{qA[\typ]}}:-> bl \typ:-> br;
%   \end{codi}\]
%   Since $qA \circ \bar a = 1_\Gamma$, we have that $\typ \circ 1_{\var} = qA \circ \bar a\typ$, which
%   induces a universal morphism $\bar a[\typ] : \var \to \var.A[\typ]$ which satisfies $qA[\typ]
%   \circ \bar a[\typ] = 1_{\var}$, as desired. Now the pair $(\bar a[\typ], qA[\typ])$ constitutes a
%   morphism $pa[\typ] : X \to \cC^S$ such that $\cC^g pa[\typ] = \chi A[\typ]$, which means we have
%   a morphism $a[\typ] : X \to \dot\cU$ with $\typ a[\typ] = A[\typ]$ and $p a[\typ]$ as just
%   defined. We also require a cartesian 2-cell $\dot \epsilon (\typ,a) : a [\typ] \to a$. \todo{Show that
%   this exists.}
%
%
%   Next we require a right adjoint $\var$ to $\typ$. This requires us to find maps $w : \cU \to \cU$ and
%   $v : \cU \to \cC^S$ such that $\chi w = \cC^g v$. Let $\bar q : \cU \to \cC \comma u$ denote the universal
%   map induced by $q : e \to u1_\cU : \cU \to \cC$, so that $\lambda\bar q = q$. Take $w = r\bar q : \cU \to
%   \cU$. Since $\chi\ell\bar q : \chi r\bar q \to \chi\cod\bar q = \chi$ is $\cod$-cartesian, we have the
%   following pullback square in $\cU \to \cC$
%   \[\begin{codi}[tetragonal=base 6em height 4.5em]
%     \obj {er\bar q & |(e1)| e \\ |(e2)| e & u \\};
%     \mor (erbar q) -> e1 -> u;
%     \mor[swap] * -> e2 -> *;
%   \end{codi}\]
%   where the left map is $\lambda'\chi r\bar q = qr\bar q$, the right map is $\lambda'\chi = q$, the top map
%   is $\dom'\chi\ell\bar q = e\ell\bar q$, and the bottom map is $\cod'\chi\ell\bar q = u\ell\bar q = \lambda
%   \bar q = q$. But since the square is a pullback, we have a map $v_1 : e \to er\bar q$ induced by $(1_e, 1_e)$
%   with the property that $qr\bar q \circ v = 1_e$. Thus we have an induced map $v = (v_1,v_2) : \cU \to \cC^S$
%   where $v_2 = r\bar q$, and so also $\lambda'\cC^g v = v_2$, so that $\cC^g v = \chi r\bar q = \chi w$, as
%   desired. Thus we have an induced map $\var$ with $\typ\var = w = r\bar q$ and $p\var = v$.
%   
%   \todo{Show that $\var$ is a a right adjoint to $\typ$.}
% \end{proof}
% \begin{remark}
%   This proof is heavily based on the proof in \cite{coraglia2024b} of the biequivalence of comprehension categories
%   and generalized categories with families. Note that this biequivalence of $\cF$-logics then provides a biequivalence
%   on all categories of models for free, including those in cat with strict, pseudo, lax, and colax morphisms, since
%   $\Mod(-)$ is 3-functorial, and biequivalences are preserved by 3-functors.
% \end{remark}

\section{Inductive Types}
In this section we show how to describe inductive and coinductive types using finite 2-limits. While the details
differ quite a
bit, the presentation of inductive types here was inspired by the presentation in \cite{basold2015}, where
inductive types are defined as initial dialgebras for a pair of functors, the signature for the type, of a
specific shape. To make sure that we have a well-behaved induction principle, we will require the existence
of dependent sum types. This allows the derivation of the induction principle from the non-dependent recursion
principle given by initiality.

We will base the presentation on comprehension categories, since the definition of terms as substitutions
turns out to be more convenient in describing the constructors, eliminators. and computation rules of
inducitve types. However, the specfic construction is not inherently limited to comprehension categories,
and should be possible to generalize to other models of type theory, such as natural models.

\begin{definition}
  A \emph{data type signature} with parameters $p : P \to \cC$ is a tuple $(F,\phi)$ where $F$ is a 1-cell
  $P \pb[p]{u} \cU \loose \cC^n$ and $\phi$ is a 2-cell $F \to \delta^n u \pi_2$, where $\delta^n : \cC
  \to \cC^n$ is the diagonal map. The intuition is that $F$ determines the arguments for each constructor
  by extending the given context, and $\phi$ connects the extend context back to the original one to
  facilitate the elimination rule.

  An algebra for a signature $(F,\phi)$ is a map $(\alpha, A) : X \to P \pb[p]{u} \cU$ equipped with a 2-cell
  $a : F(\alpha,A) \to \delta^n\set{A}$ such that $\phi(\rho,A) = \delta^n\pi A \circ a$, and an algebra homomorphism
  $(\alpha,A,a) \to (\beta, B,b)$ is a 2-cell $(\rho,f) : (\alpha,A) \to (\beta,B)$ such that $\delta^n\set{f}
  \circ a = b \circ F(\rho,f)$.
\end{definition}
\begin{remark}
  An algebra for a signature $(F,\phi)$ is an $(F,\delta^n\set{\pi_2})$-\emph{dialgebra} subject to an
  additional condition ensuring compatibility with the 2-cell $\phi$. Algebra homomorphisms are exactly
  maps of dialgebras. 
\end{remark}

\begin{remark}
  For a fixed signature $(F,\phi)$, we can using only finite $\cF$-weighted limits construct an object
  of algebras $(F,\phi)\Alg$, such that any $(F,\phi)$-algebra is the same as a map into $(F,\phi)\Alg$,
  and vice versa. This object comes equipped with a universal $(F,\phi)$-algebra, with underlying map
  $R_{(F,\phi)} : (F,\phi) \Alg \to P \pb[u]{p} \cU$ and 2-cell $\rho : F R_{(F,\phi)} \to \delta^n\set
  {R_{(F,\phi)}}$.
\end{remark}

We want an inductive type over a signature $(F,\phi)$ to be an inital algebra. Since this algebra is supposed
to exist over aribtary parameters, this is best modeled by a map from the object of parameters to the object
of algebras.

\begin{definition}
  An inductive type for a signature $(F,\phi)$ is a map $\mu(F,\phi) : P \to (F,\phi)\Alg$ such that
  $\pi_1 R_{(F,\phi)} \mu(F,\phi) = 1_P$ and for every $A : X \to (F,\phi)\Alg$, there exists a unique
  map $h : \mu(F,\phi)\pi_1R_{(F,\phi)} A \to A$ such that $\pi_1R_{(F,\phi)} h = 1_{\pi_1R_{(F,\phi)}A}$
\end{definition}
\begin{lemma}
  Let a signature $(F,\phi)$ and $\mu(F,\phi) : P \to (F,\phi)\Alg$ be given. Then the following are equivalent:
  \begin{enumerate}
    \item $\mu(F,\phi)$ is an inductive data type.
    \item $\mu(F,\phi)$ is left adjoint to $\pi_1 R_{(F,\phi)} : (F,\phi)\Alg \to P$ with identity unit.
  \end{enumerate}
\end{lemma}
\begin{proof}
  For $1 \implies 2$, note that by taking $A = 1_{(F,\phi)\Alg}$, we see that there exists a unique map $
  \epsilon : \mu(F,\phi)\pi_1R_{(F,\phi)} \to 1_{(F,\phi)\Alg}$ such that $\pi_1 R_{(F,\phi)}\epsilon =
  1_{\pi_1 R_{(F,\phi)}}$. Thus it only remains to show that $\epsilon\mu(F,\phi) = 1_{\mu(F,\phi)}$. But
  there exists by uniqueness only one map $h : \mu(F,\phi)\pi_1R_{(F,\phi)}\mu(F,\phi) = \mu(F,\phi) \to
  \mu(F,\phi)$ such that $\pi_1 R_{(F,\phi)} h = 1_{\pi_1 R_{(F,\phi)}\mu(F,\phi)} = 1_{1_{(F,\phi)\Alg}}.$
  Since the identity map $1_{\mu(F,\phi)}$ and $\epsilon\mu(F,\phi)$ both satisfy this property, we conclude
  that they are the same, thereby establishing that $\epsilon$ is the counit of an adjunction $\mu(F,\phi)
  \dashv \pi_1R_{(F,\phi)}$ with identity unit.

  For $2 \implies 1$, let $A : X \to (F,\phi)\Alg$ be given. Clearly $\epsilon A$ is a map $\mu(F,\phi)\pi_1
  R_{(F,\phi)}A \to A$ satisfying the desired property, so it remains to show that this map is unique. To that
  end, let $h : \mu(F,\phi)\pi_1R_{(F,\phi)}A \to A$ be given such that $\pi_1R_{(F,\phi)}h = 1_{\pi_1R_{(F,
  \phi)}A}$. Note that
  \[\epsilon A \circ \mu(F,\phi) \pi_1 R_{(F,\phi)} h = h \circ \epsilon \mu(F,\phi)\pi_1 R_{(F,\phi)}A = h,\]
  but $\pi_1 R_{(F,\phi)} h = 1_{\pi_1 R_{(F,\phi)}}$, so we see that
  \[\epsilon A = \epsilon A \circ \mu(F,\phi)\pi_1 R_{(F,\phi)} h = h,\]
  as desired.
\end{proof}

\begin{remark}
  The data types described here satisfy the $\eta$-rule, since $\epsilon\mu(F,\phi) = 1_{\mu(F,\phi)}$. We could
  drop this property by asking for a weakly initial object, rather than an initial object. This would be removing
  the uniqueness requirement from the definition of an inductive data type, or equivalently asking for a map
  $\epsilon : \mu(F,\phi)\pi_1R_{(F,\phi)} \to 1_{(F,\phi)\Alg)}$ such that $\pi_1 R_{(F,\phi)}\epsilon =
  1_{\pi_1R_{(F,\phi)}}$.
\end{remark}

\begin{example}[Empty type]
  The empty type has no parameters, meaning we take $p = 1_\cC : \cC \to \cC$, and no constructors. Thus we
  take $F$ to be the unique map into the 2-terminal object, and $\phi$ is trivially determined by this.
  We use $\bot = \mu(F,\phi)$ to denote this inductive type. 

  An algebra for $(F,\phi)$ is simply a type $A : X \loose \cU$ (up to isomorphism), and the universal property
  then always gives us a map $\set{\bot uA} \to \set{A}$, which we interpret to say that for any context
  containing $\bot$, we have an element of any type.
\end{example}

\begin{example}[Unit type]
  The unit type also has no parameters, so $p = 1_\cC : \cC \to \cC$. It has exactly one constructor, and
  this constructor takes no arguments. Thus we take $F = p\pi_1$ and $\phi = 1_{p\pi_1}$. We denote the
  inductive type for this signature by $\top$.

  An algebra for $(F,\phi)$ is a type $A : X \to \cU$ with a substitution $a : uA \to \set{A}$ such that
  $\pi A \circ a = 1_{uA}$, i.e.\ a term of type $A$. We use $\top$ also to denote the underlying type
  $\pi_2 R_{(F,\phi)}\top$ of the initial algebra. The constructor $* : u\top \to \set{\top}$ is the map
  given by the algebra, formally $\pi_2 \rho_{(F,\phi)}$. The universal property then gives us a map
  $\set{\pi_2R_{(F,\phi)}\epsilon A} : \set{\top uA} \to \set{A}$ such that $\set{\pi_2 R_{(F,\phi)}\epsilon A}
  \circ * = a \circ p\pi_1R_{(F,\phi)}\epsilon A = a$, which is exactly the
  $\beta$-rule.
\end{example}

\begin{example}[Binary sum types]
  Two constructors, $P = \cU \pb[u]{u} \cU$, $p = u\pi_1 = u\pi_2$, $F_1 = \set{\pi_1\pi_1}$, $F_2 =
  \set{\pi_2\pi_1}$. 
\end{example}

\begin{example}[Natural numbers]
  No parameters, so $p = 1_\cC : \cC \to \cC$. Two constructors, the first with no arguments, so take
  $F_1 = u\pi_2 = p\pi_1$, and the second with one, so take $F_2 = \set{\pi_2}$. An algebra is a type
  $A : X \to \cU$ with two substitutions, $a_z : uA \to \set{A}$ and $a_s : \set{A} \to \set{A}$.
  The initial algebra is a type $\bN : \cC \to \cU$ with two constructors $z : 1_\cC \to \set{\bN}$ and
  $s : \set{\bN} \to \set{\bN}$, and the eliminator into $A$ gives $\epsilon A : \bN \to A$ such that
  $\set{\pi_2 \epsilon A} \circ s = a_s \circ \set{\pi_2 \epsilon A}$ and $\set{\pi_2 \epsilon A} \circ z
  = a_z$.
\end{example}

\begin{example}[W-types]
  Suppose the theory has function types, whose type former we denote by $Fun : \cU \pb[u]{u} \cU \to \cU$.
  $W$-types are parameterized by a type $A$ and a family $B$ indexed over the type. Thus we take $P = (\cU
  \pb[\set{-}] {u} \cU)$ and $p = u\pi_1$. It has a single constructor, so $F : P \pb[p]{u} \cU \to \cC$,
  and this constructor takes an element of the first type $A$ and, letting $X = \pi_2$, a map $Fun(B, X[
  \pi A])$. Thus we take $F = \set{Fun(\pi_2\pi_1, r(\pi\pi_1\pi_1, \pi_2))}$, where $r : \cC \comma u \to
  \cU$ is the lifting map for the fibration $u$ and $(\pi\pi_1\pi_1, \pi_2) : P \pb[p]{u} \cU \to \cC
  \comma u$ is the map induced by the 2-cell $\pi\pi_1\pi_2 : \set{\pi_1\pi_1} \to u\pi_1\pi_1 = u\pi_2$.
\end{example}

\todo{Find better notation than the pointfree style used here. Also, $\pi_i$ is maybe a bad choice for the
projections.}

The one thing missing now is the general induction principle, which is where we require dependent sum types.
Consider the induction rule for natural numbers:
\[\infer{\Gamma.\bN \vdash B\;\cU \\ b_z : \Gamma \to \Gamma.B[z] \\ \pi B[z] \circ b_z = 1_\Gamma \\
  b_s : \Gamma.\bN.B  \to \Gamma.\bN.B \\ \pi B \circ b_s = s \circ \pi B}
  {\Gamma.\bN\vdash \elim(B,b_z,b_s) : B}\]
Using dependent sum types, we can change the premise for the successor to be a map $b_s' : \Gamma.\sum_\bN B
\to \Gamma.\sum_\bN B$ such that $\pi(\sum_\bN B) \circ b_s' = \pi(\sum_\bN B)$ and $\mathtt{fst} \circ b_s'
= s \circ \mathtt{fst}$. Similarly, we can change the premise for the zero case to be $b_z' : \Gamma \to
\Gamma.\sum_\bN B$ such that $\pi(\sum_\bN B) \circ b_z' = 1_\Gamma$ and $\mathtt{fst} \circ b_z' = z$.
Effectively then, this depedent algebra $B$ over $\bN$ is equivalent to an algebra $b'$ on $\sum_\bN B$,
so a map $b' : F(\sum_\bN B) \to \set{\sum_\bN B}^n$, such that $\fst^n \circ b' = a' \circ F\fst$, i.e.
such that $\fst$ is an algebra homomorphism.



% \todo{Talk about induction principle and the use of $\Sigma$-types.}
\section{Multimode and Modal type theories}
Multimode dependent type theory (MTT) was introduced in \cite{gratzer2021} to provide a general family of type theories
with different modalities. MTT is parameterized by a 2-category $\cM$, a so-called mode theory, the objects of which
are called \emph{modes}, the 1-cells are called \emph{modalities}, and the 2-cells are called \emph{operations}.
The goal of this section is to show how the semantics presented in \cite{gratzer2021} can be seen as the models of
a finite $\cF$-limit sketch. A interesting feature of the semantics of MTT which diffrentiates it from most other
type theories, is that it more or less consists of a family of type theories, one at each mode, together with rules
connecting the different type theories with the modalities and operations. as dictated by the mode theory. This is
reflected in the semantics presented in \cite{gratzer2021} by having a family of natural models indexed by the modes,
together with functors and natural transformations between the models corresponding to the modalities and operations.
Because the semantics requires multiple categories of contexts, it is difficult to fit neatly into the functorial
semantics of \cite{uemura2023} or the framework of \cite{coraglia2024a}, which single out the context judgement, and
so the category of contexts, as a primitive notion.

To show that the semantics of \cite[{§5}]{gratzer2021} can be recovered as an finite $\cF$-limit theory, we will
presen the important features of \emph{op.\ cit.\ } and show that these can be constructed by finite 2-limits. We
begin with the defintions of a \emph{modal context structure} and \emph{modal natural models}.
\begin{definition}
  Let $\cM$ be a mode theory. A \emph{modal context structure} is a 2-functor $\cC[-] : \cM^{coop} \to \Cat$. 

  A \emph{modal natural model} on a modal context structure $\cC[-]$ is an family of maps of discrete fibrations
  $\typ_m : \dot\cU[m] \to \cU[m]$ over $\cC[m]$, indexed over the modes $m \in \cM$, such that for each modality
  $\mu : m \to n$, the map $\cC[\mu]^*\typ_m : \cC[\mu]^*\dot\cU[m] \to \cC[\mu]^*\cU[m]$ is a representable map of
  discrete fibrations.
\end{definition}
\begin{remark}
  In \cite{gratzer2021}, they require the category of contexts to have a terminal object, corresponding to the empty
  context. Since we have elided such a requirement from the start, we will continue to do so now. Note, however, that
  it is trivial to add the requirement by simply adding a terminal object $1$ in the sketch and right adjoints to the
  unique maps $\cC[m] \to 1$.
\end{remark}
We will describe a sketch for modal natural models directly, since the $\cF$-sketch for modal context structures is
just $((\cM^{coop})^-, \emptyset)$, the sketch with underlying $\cF$-category $\cM^{coop}$ with only identities tight
and with no cones. 
\begin{definition}
  The sketch for modal natural models consists of:
  \begin{itemize}
    \item For each mode $m$, a diagram
      \[\begin{codi}[hexagonal=horizontal side 6em angle 45] 
        \obj{|(tm _m)| \dot\cU_m &   |(ty _m)| \cU_m \\ & |(ctx _m)| \cC_m \\};
        \mor  (tm _m) ["\typ_m",swap]:≈> (ty _m) u_m:-> (ctx _m);
        \mor[swap] (tm _m) \dot u_m:-> (ctx _m);
        % \mor[swap] (ty _m) \var_m:[bend right,≈>] (tm _m);
        % \mor (typ _m) [shorten=0.3em, |-] (var _m);
      \end{codi}\]
      with cones and additional maps making $u_m, \dot u_m$ discrete fibrations.
    \item For each modality $\mu : m \to n$, squares
      \[\begin{codi}[tetragonal=base 4.5em height 4.5em]
        \obj { \lock_\mu^* \cU_m & \cU_m \\ \cC_n & \cC_m \\ };
        \mor (lock _mu ^* cU _m) -> (cU _m) u_m:-> (cC _m);
        \mor[swap] (lock _mu ^* cU _m) \lock_\mu^*u_m:-> (cC _n) \lock_\mu:-> (cC _m);
      \end{codi} \qquad
      \begin{codi}[tetragonal=base 4.5em height 4.5em]
        \obj { \lock_\mu^* \dot\cU_m & \dot\cU_m \\ \cC_n & \cC_m \\ };
        \mor (lock _mu ^* dot cU _m) -> (dot cU _m) \dot u_m:-> (cC _m);
        \mor[swap] (lock _mu ^* dot cU _m) \lock_\mu^*\dot u_m:-> (cC _n) \lock_\mu:-> (cC _m);
      \end{codi}\]
      and an adjunction $\lock_\mu^* \typ_m \dashv \var_\mu : \lock_\mu^* \cU_m \to \lock_\mu^*\dot\cU_m$,
      as well as cones and diagrams making the two squares pullbacks and $\lock_\mu^*\typ_m : \lock_\mu^*\dot\cU_m
      \to \lock_\mu^*\cU_m$ the map induced by $\typ_m : \dot\cU_m \to \cU_m$.
    \item For each operation $\alpha : \mu \to \nu$, a 2-cell $\key_\alpha : \lock_\nu \to \lock_\mu$.
  \end{itemize}
  Additionally we add equalites so that the assignments $m \mapsto \cC_m$, $(\mu : m \to n) \mapsto \lock_\mu$, and
  $(\alpha : \mu \to \nu) \mapsto \key_\alpha$ together constitute a strict 2-functor from $\cM^{coop}$.
\end{definition}

This does not yet account for the formation, introduction, elimination, and computation rules for the modal types,
which is thus the next step. The modal type formation and introduction rules are given by
\[\inferrule{\mu : m \to n \\ \Gamma.\lock_\mu \vdash A\;\cU_n}{\Gamma \vdash \seq{\mu \given A}\;\cU_m}
\qquad\inferrule{\mu : m \to n \\ \Gamma.\lock_\mu \vdash a : A}{\Gamma \vdash \mathrm{mod}_\mu(a)
: \seq{\mu \given A}}\]
These then correspond to, for each $\mu : m \to n$, additional commuting square
\[\begin{codi}[tetragonal=base 4.5em height 4.5em]
  \obj { |(tl)| \lock_\mu^*\dot\cU_m & |(tr)| \dot\cU_n \\ |(bl)| \lock_\mu^*\cU_m & |(br)| \cU_n \\};
  \mor tl {{\mathrm{mod}_\mu}}:-> tr \typ_n:-> br;
  \mor[swap] tl \lock_\mu^*\typ_m:-> bl {{\seq{\mu\given -}}}:-> br;
\end{codi}\]
such that $u_n \circ \seq{\mu\given -} = \lock_\mu^* u_m$. For each $\nu : o \to n$ and $\mu : n \to m$, the elimination
rule is given as
\[\inferrule{\Gamma\;\cC_m \\ \Gamma.\lock_\mu.\lock_\nu \vdash A \; \cU_o \\
\Gamma.\lock_\mu \dashv M_0 : \seq{\nu \given A} \\ \Gamma.(\mu\;\vert\;\seq{\nu \given A}) \vdash B\;\cU_m \\
\Gamma.(\mu \circ \nu\;\vert\; A) \vdash M_1 : B[\set{\mathrm{mod}_\nu(\tau \var A)}]}
{\Gamma \vdash \elim_\mu^\nu}\]
\todo[inline]{Show how to encode the eliminator, it's not hard, just annoying.}

\ifSubfilesClassLoaded{\printbibliography}{}
\end{document}
