\documentclass[../thesis.tex]{subfiles}

\ifSubfilesClassLoaded{
  \externaldocument{../build/2-preliminaries}%
  \externaldocument{../build/3-logics}%
}

\begin{document}
\chapter{Type Theories}

\section{Fibrations as algebraic structure}
A fundamental tool in the different semantics of type theories is the notion of fibration. For example, they allow for
modeling substitution in types and terms in a principled manner, and provide thus provide an alternative to presheaves.
Furthermore, since they make no mention of $\Cat$ directly, we can attempt to describe a theory of fibrations in terms
of an $\cF$-limit sketch, and this is what provides the basis for using $\cF$-logics and sketches to describe categorical
semantics of type theories.

\begin{definition}[Cartesian morphism]
  Let $p : A \to B$ be a tight morphism of an $\cF$-category $\bA$. A 2-cell $\phi : a_1 \to a_2 : X \loose A$ is
  called \emph{$p$-cartesian} if for every $g : Y \loose X$, $a_0 : Y \loose A$, $\alpha : a_0 \to a_2g$, and
  $\beta : pa_0 \to pa_1g$ with $\beta \circ p\phi = p\alpha$, there exists a unique $\gamma : a_0 \to a_1g$ such
  that $\beta = p\gamma$ and $\alpha = \phi g \circ \gamma$. We will call the data $(g,a_0,\alpha,\beta)$
  satisfying the equality $\beta \circ p\phi = p\alpha$ a cartesian cone.

  If $A^\to$ exists in $\bA$, we say that a 1-cell $f : X \to A^\to$ is cartesian when $\lambda f : \dom f \to
  \cod f$, the 2-cell induced by $f$, is cartesian.
\end{definition}
\begin{proposition}
  Let $p : A \to B$ be a tight morphism in an $\cF$-category $\bA$ with finite weighted limits and $\phi : a_1 \to a_2 : X
  \loose A$ a 2-cell. Consider the diagram $D : \bJ \to \cA$ (with $\bJ$ the obvious shape $\cF$-category) given by
  \[\begin{codi}
    \obj { X & A & B \\};
    \mor X ["a_1",name=x]:[->,bend left] A p:-> B;
    \mor X ["a_2",name=y,swap]:[->,bend right] A;
    \mor x \phi:-2> y;
  \end{codi}\]
  and weight $W : \bJ \to \bF$ generated by
  \[\begin{codi}
    \obj { |(X)| 1 & |(A)| H & |(B)| 3 \\};
    \mor X ["1",name=x]:[->,bend left] A c:-> B;
    \mor X ["2",name=y,swap]:[->,bend right] A;
    \mor x \phi:-2> y;
  \end{codi}\]
  where $H = \set{0 \to 2 \from 1}$ is the generic (tight) cospan, and $c : H \to 3$ maps the morphism $1 \to 2$ to
  the corresponding morphism in $3$ and $0 \to 2$ to the composite $0 \to 1 \to 2$ in $3$.

  A $W$-weighted cone over $D$ is exactly a cartesian cone over $\phi$.
\end{proposition}

\begin{proposition}\label{prop:cartesian by limit}
  Let $\bA$ be an $\cF$-category with finite weighted limits, $p : A \to B$ a 1-cell in $\bA$, and $\phi : a_1 \to
  a_2 : X \loose A$ a 2-cell in $\bA$. Then $\phi$ is cartesian if and only if $\ell = (\cod, \dom, \phi\cod \circ
  \lambda, p\lambda)$ is a universal cartesian cone with apex $A \comma a_1$.
\end{proposition}
\begin{proof}
  In the left-to-right direction, suppose $\phi$ is cartesian. We must show that for every object $Y$, the following
  hold:
  \begin{enumerate}[label=(\arabic*)]
    \item for every cartesian cone $(g,a,\alpha,\beta)$ with apex $Y$, there exists a unique morphism $h : Y \to A
      \comma a$ such that $g = \cod h$, $a = \dom h$, $\alpha = (\phi\cod \circ \lambda)h$ and $\beta = p\lambda h$.
    \item for every morphism of cones $m : (g,a,\alpha,\beta) \to (g',a',\alpha',\beta')$ there exists a unique
      morphism $h_m : h \to h'$ such that $m = h_m^* \ell$.
  \end{enumerate}
  For (1), note that since $(g,a,\alpha,\beta)$ is a cartesian cone and $\phi$ is cartesian, there exists a unique
  2-cell $\gamma : a \to a_1g$ such that $\alpha = \phi \circ \gamma$ and $\beta = p\gamma$. By the universal
  property of $A \comma a_1$, there then exists a unique morphism $\bar\gamma : Y \to A \comma a_1$ with $\lambda\bar
  \gamma = \gamma$. In particular, we have that $\cod\bar\gamma = g$, $\dom\bar\gamma = a$, $\alpha = \phi \circ
  \lambda\bar\gamma$, and $\beta = p\lambda\bar\gamma$, so taking $h = \bar\gamma$ we are done. Furthermore, $h =
  \bar\gamma$ is the unique such map by uniqueness of $\bar\gamma$ and $\gamma$.

  For (2), suppose we are given a modification $m : (g,a,\alpha,\beta) \to (g',a',\alpha',\beta')$. Concretely, $m$
  consists of two morphisms $m_1 : g \to g'$ and $m_2 : a \to a'$ such that the following squares commute:
  \[\begin{codi}[tetragonal=base 6em height 4.5em]
    \obj { a & a' & & pa & pa' \\ a_2g & a_2g' & & pa_1g & pa_1g' \\};
    \mor[swap] a \alpha:-> a_2g a_2m_1:-> a_2g';
    \mor a m_2:-> a' \alpha':-> a_2g';

    \mor pa pm_2:-> pa' \beta':-> pa_1g';
    \mor[swap] pa \beta:-> pa_1g pa_1m_1:-> pa_1g';
  \end{codi}\]
  We want to show that the square
  \[\begin{codi}[tetragonal=base 6em height 4.5em]
    \obj {a & a' \\ a_1g & a_1g' \\};
    \mor a m_2:-> a' \gamma':-> a_1g';
    \mor[swap] a \gamma:-> a_1g a_1m_1:-> a_1g';
  \end{codi}\]
  commutes as well, as that induces the desired unique morphism $\bar m : \bar \gamma \to \bar \gamma'$ with the
  desired properties. To that end, consider the cartesian cone over $Y$ given by $(g', a, \alpha'' = \alpha' \circ
  m_2, \beta'' = \beta'  \circ pm_2)$. To see that this is indeed a cone, we must show that $p\alpha'' = p\phi \circ
  \beta''$, but this follows from simple computation:
  \[p\alpha'' = p(\alpha' \circ m_2) = p\alpha' \circ pm_2 = p\phi \circ \beta' \circ pm_2 = p\phi \circ \beta''\]
  Since $(g',a,\alpha'',\beta'')$ is a cartesian cone and $r$ is cartesian, there exists a unique lift $\gamma''$
  with $p\gamma'' = \beta''$ and $\alpha'' = \phi g' \circ \gamma''$. But we have already that
  \[p(\gamma' \circ m_2) = p\gamma' \circ pm_2 = \beta' \circ pm_2 = \beta'' \text{\quad and \quad}
  \phi g' \circ \gamma' \circ m_2 = \alpha' \circ m_2 = \alpha'',\]
  and similarly
  \[p(a_1m_1 \circ \gamma) = pa_1m_1 \circ p\gamma = pa_1m_1 \circ \beta = \beta''\]
  and
  \[\phi g' \circ a_1m_1 \circ \gamma = a_2m_1 \circ \phi g \circ \gamma = a_2 m_1 \circ \alpha = \alpha'',\]
  where we use the interhcange law to commute $\phi$ and $m_1$.
  In particular, we see that both $\gamma' \circ m_2$ and $a_1m_1 \circ \gamma$ provide universal morphisms for
  the cartesian cone $(g',a,\alpha'', \beta'')$, from which we conclude that they must be identical by the uniqueness
  of such morphisms. Hence the diagram
  \[\begin{codi}[tetragonal=base 6em height 4.5em]
    \obj {a & a' \\ a_1g & a_1g' \\};
    \mor a m_2:-> a' \gamma':-> a_1g';
    \mor[swap] a \gamma:-> a_1g a_1m_1:-> a_1g';
  \end{codi}\]
  commutes.

  For the right-to-left direction, suppose that $(\cod,\dom,\phi\cod \circ \lambda, p\lambda)$ is a limiting cone
  with apex $A \comma a_1$, and let $(g,a,\alpha,\beta)$ be any cartesian cone with apex $Y$. By the universal
  property of limiting cones, there exists a unique morphism $\bar\gamma : Y \to A \comma a_1$ such that
  $\cod\bar\gamma = g$, $\dom\bar\gamma = a$, $(\phi\cod \circ \lambda)\bar\gamma = \alpha$, and
  $p\phi\bar\gamma = \beta$. In particular, taking $\gamma = \lambda\bar\gamma : a \to a_1g$, we have the
  desired morphism, and it is unique by the universal property of $A \comma a_1$.
\end{proof}


\begin{definition}[Fibration in an $\cF$-category]
  A \emph{fibration} in an $\cF$-category $\bA$ is a tight morphism $p : A \to B$ such that for every morphism $\phi :
  a_1 \to p a_2 : X \to B$ (with $a_1 : X \to B$ and $a_2 : X \to A$) there exists a $p$-cartesian morhism $\bar \phi :
  a_1' \to p a_2$ such that $\phi = p\bar\phi$.
\end{definition}

\begin{proposition}
  Let $p : A \to B$ be a tight morphism in an $\cF$-category $\bA$ with finite weighted limits. Then the
  following are equivalent:
  \begin{enumerate}[label=(\arabic*)]
    \item $p$ is a fibration.
    \item There exists a map $\ell : B \comma p \to A$ with $p\ell = \dom$ and $p$-cartesian 2-cell $\epsilon :
      \ell \to \cod$ with $p\epsilon = \lambda$, where $\lambda : \dom \to p\cod$ is the universal 2-cell for $B
      \comma p$.
    \item The map $t : A^\to \to B \comma p$ has a section $r : B \comma p \to A^\to$ which is $p$-cartesian.
    \item The canonical map $t : A^\to \to B \comma p$ has a right adjoint $r : B \comma p \to A^\to$
      with identity counit.
    \item The map $i : A \to B \comma p$ induced by the identity $1_p$ has a right adjoint $\ell : B \comma p
      \to A$ in the slice $\bA/B$ with invertible unit.
  \end{enumerate}
\end{proposition}
\begin{proof}
  We show that $(1) \iff (2) \implies (3) \implies (4) \implies (2)$ and $(2) \iff (5)$.

  For $(1) \implies (2)$, note that since $p$ is a fibration, there exists a $p$-cartesian lift $\epsilon : \ell
  \to \cod : B \comma p \to A$ for the 2-cell $\lambda : \dom \to p\cod : B \comma p \to B$, and this lift
  satisfies $p\epsilon = \lambda$. This is exactly the map we require.

  For $(2) \implies (1)$, note that since $\epsilon : \ell \to \cod$ is $p$-cartesian, so is $\epsilon x$ for
  all $f : X \loose B \comma p$. In particular, for any 2-cell $\phi : b \to pa$ with $b : X \to B$ and $a : X
  \to A$, we have an induced 1-cell $f : X \loose B \comma p$ with $\lambda f = \phi$. Then $\epsilon f$ is $p$-%
  cartesian, and by the properties of $\epsilon$, we have that $p\epsilon f = \lambda f = \phi$, so $\epsilon f$
  is a cartesian lift for $\phi$. Since $\phi$ was an arbitrary such 2-cell, we conclude that every such 2-cell
  has a caresian lift, which is exactly what it means for $p$ to be a fibration.


  For $(2) \implies (3)$, note that $\epsilon$ induces a map $r : B \comma p \to A^\to$ with $\lambda'r =
  \epsilon$, where $\lambda' : \dom' \to \cod' : A^\to \to A$ is the universal 2-cell associated with $A^\to$.
  This map then satisfies that $\lambda tr = p\lambda' r = p\epsilon = \lambda$, so that $tr = 1_{B \comma p}$,
  and $r$ is $p$-cartesian by definition.

  suppose that $p$ is a fibration. We require first of all a map $r : B \comma p \to
  A^\to$, which we then show is a $p$-cartesian section to $t : A^\to \to B \comma p$. Such a map $r$ is
  equivalently a 2-cell $\lambda r : \dom r \to \cod r : B \comma p \to A$. Since $p$ is a fibration, there
  exists a $p$-cartesian lift $\bar r : x \to \cod' : B \comma p \to A$ for the map $\lambda' : \dom' \to p
  \cod' : B \comma p \to B$, where $x : B \comma p \to A$ is such that $px = \dom'$. This map $\bar r$ induces
  a 1-cell $r : B \comma p \to A^\to$ with $\lambda \bar r = r$. In particular, we see that $\lambda' t r =
  p\lambda r = p\bar r = \lambda'$, so that $tr = 1_{B \comma p}$, as desired. Hence $r : B \comma p \to A^\to$
  is a $p$-cartesian section of $t$.

  For $(3) \implies (4)$, suppose $r$ is a $p$-cartesian 1-cell with $tr = 1_{B \comma p}$. We show that $r$
  is also a right-adjoint to $t$, for which we must find a unit $\eta : 1_{A^\to} \to rt$. By the universal
  property of $A^\to$, $\eta$ is equivalent to a commuting square
  \[\begin{codi}[tetragonal=base 6em height 4.5em]
    \obj {\dom' & \dom'rt \\ \cod' & \cod'rt \\};
    \mor       (dom ') \eta_1:-> (dom 'rt) \lambda'rt:-> (cod 'rt);
    \mor[swap] (dom ') \lambda':-> (cod ') \eta_2:-> (cod 'rt);
  \end{codi}\]
  Since $\cod'rt = \cod trt = \cod t = \cod'$, it would suffices to find a map $\eta_1$ such that
  $\lambda' = \lambda'rt \circ \eta_1$, since $\eta_2 = 1_{\cod'}$ then makes the above square commute. Since
  $r$ is a $p$-cartesian 1-cell, $\lambda'rt$ is a $p$-cartesian 2-cell, whereby there exists a unique universal
  map $\gamma : \dom' \to \dom'rt$ induced by the cone $(\cod', \dom', \lambda', 1_{p\dom'})$ with $p\gamma =
  1_{p\dom'}$ and $\lambda' = \lambda'rt \circ \gamma$, so we may take $\eta_1 = \gamma$, and then $\eta$ is
  the universal map induced by the square
  \[\begin{codi}[tetragonal=base 6em height 4.5em]
    \obj {\dom' & \dom'rt \\ \cod' & \cod'rt \\};
    \mor       (dom ') \gamma:-> (dom 'rt) \lambda'rt:-> (cod 'rt);
    \mor[swap] (dom ') \lambda':-> (cod ') 1_{\cod'}:-> (cod 'rt);
  \end{codi}\]
  To verify that $\eta : 1_{A^\to} \to rt$ is indeed the unit of an adjunction $t \dashv r$, we must verify that
  $t\eta = 1_t$ and $\eta r = 1_r$. For the former, it suffices to show that $\dom t\eta = 1_{\dom t}$ and
  $\cod t\eta = 1_{\cod t}$. But these hold since
  \[\dom t\eta = p\dom'\eta = p\gamma = 1_{p\dom'} = 1_{\dom t} \qquad\text{ and }\qquad
  \cod t\eta = \cod'\eta = 1_{\cod'} = 1_{\cod t}.\]
  Hence $t\eta = 1_t$. For the latter equation $\eta r = 1_r$, we must show that $\gamma r = 1_{\dom' r}$,
  for which it suffices to see that $1_{\dom' r}$ and $\gamma r$ both are universal morphisms for the same
  $p$-cartesian cone over $\lambda'r : \dom' r \to \cod' r : B\comma p \to A$. The cone we consider is
  specifically $(1_{B \comma p}, \dom'r, \lambda'r, 1_{p\dom'r})$ with apex $B \comma p$, which clearly
  satisfies $p\lambda'r = p\lambda'r \circ 1_{p\dom'}$. Furthermore, we clearly see that $1_{\dom' r}$ is
  a universal map over the cone, and that $p\gamma r = 1_{p\dom' r}$ and $\lambda'r \circ \gamma r =
  \lambda'rtr \circ \gamma r = (\lambda'rt \circ \gamma)r = \lambda'r$, so $\gamma r$ is another such
  universal map. Hence $\gamma r = 1_{\dom' r}$, from which we conclude that $\eta r = 1_r$. Hence
  $\eta$ is the unit of an adjunction $t \dashv r$ with identity counit.

  For $(4) \implies (2)$, we must show that the right adjoint $r : B \comma p \to A^\to$ is $p$-cartesian, so let
  $(g,a,\alpha,\beta)$ be a $p$-cartesian cone over $\lambda' r : \dom' r \to \cod' r$ with apex $Y$. Thus
  $g : Y \to B \comma p$, $a : Y \to A$, $\alpha : a \to \cod'r g$, and $\beta : pa \to p\dom'rg = \dom trg
  = \dom g$.
\end{proof}

\begin{notation}
  For any fibration $p : A \to B$ with a chosen lifting $\epsilon : \ell \to \cod$ with $\ell : B \comma p \to A$
  as in (2), and any object $X$ and morphisms $a : X \to A$, $b : X \to B$, and $\sigma : b \to pa$, we write
  $A[\sigma] \coloneq \ell\bar\sigma$, where $\bar \sigma : X \to B \comma p$ is the map induced by $\sigma$.
\end{notation}

\section{From syntax to \texorpdfstring{$\cF$}{F}-logics}
In this section we show a correspondence between the syntax of a deductive theory, presented by judgements and
inference rules, and finite limit $\cF$-logics. The connection is generally as follows: a (dependent) judgement
$d\;\cD \vdash j\;\cJ$ is interpreted is a tight morphism $j : \cJ \to \cD$, with the idea that the fibers of
$j$ over any instance $d\;\cD$ corresponds to $d\;\cD \vdash j\;\cJ$. A non-dependent judgement can then be
viewed as a judgement over the terminal object $1$, so that the dependency is trival. Equivalently a non-dependent
judgement is just an object. The 2-categorical structure gives us, for each judgement $\cJ$, a judgement of
morphisms $\cJ^\to$. In modelling type theory, we would for example have a judgement of contexts $\vdash \ctx$,
and then a judgement of substitions $\Gamma\ctx, \Delta\ctx \vdash \sigma \ctx^\to$. Given a (dependent) judgement
of types over contexts $\rty : \ty \to \ctx$, we would then want to formulate a rule for stability under such
substitutions:
\begin{mathpar}
  \infer{\Gamma\ctx,\Delta\ctx \vdash {{\sigma \ctx^\to}} \\ \Delta \ctx \vdash A\ty}{\Gamma \ctx \vdash
  \sub(A,\sigma) \ty}
\end{mathpar}
We capture this rule as a loose morphism $\sub : \ctx^\to \pb[\cod]{\rty} \ty \to \ty$ for which the following
triangle commutes.
\[\begin{codi}[tetragonal=base 4.5em height 4.5em]
  \obj {|(P)| \ctx^\to \pb[\cod]{\rty} \ty & & \ty \\ & \ctx \\};
  \mor ctx \dom\pi_1:<- P \sub:-> ty \rty:-> ctx; 
\end{codi}\]
If we now note that $\ctx^\to \pb[\cod]{\rty} \ty \cong \ctx \comma \rty$, we see that $\sub$ is exactly as the
morphism $\ell : \ctx \comma \rty \to \rty$ specifying the domain of the lifting in a fibration $\ty \to \ctx$.


\section{Comprehension Categories and Categories with Attributes}
We begin by constructing an finite limit logic for comprehension categories. Recall that a comprehension category
consist of commutative triangle in $\Cat$
\[\begin{codi}[hexagonal=horizontal side 6em angle 45] 
  \obj{\cU &   \cC^\to \\
           & \cC \\};
  \mor  cU \chi:-> (cC ^to) \cod:-> cC;
  \mor[swap] cU u:-> cC;
\end{codi}\]
where $u: \cU \to \cC$ is a fibration, and $\chi$ preserves cartesian morphisms.

To axiomatise this as a finite $\cF$-limit logic, we consider the sketch whose underlying $\cF$-category is generated
by a diagram as above, with $\chi$ loose and $u$ and $\cod$ tight. Further, we want $u$ to be a fibration, so we add
as a generator the maps $t : \cU^\to \to \cC \comma u$ and a section $r : \cC \comma u \to \cU^\to$, such that $r$
will be cartesian. Since $\chi$ is supposed to preserve cartesian morphisms, we will ask that $\chi^\to r$ is cartesian
as well, as that says that the image of cartesian lifts are cartesian, which suffices. We shall call the logic generated
by this sketch $\Comp$. Another very relevant logic is that of categories with attributes, where $u$ is supposed to be
discrete. For this, we let $r$ be a full inverse, rather than just a section, to $t$, but still only require $\chi^\to r$
to be cartesian.

Note that all of this data is immediately describable as a finite limit logic;
from example \cref{ex:fib}, we see that we can ask $\cU$ to be a fibration, and similarly that $\cC^\to$ is the arrow
object of $\cC$. More formally, we let $\Comp$ be the $\cF$-theory generated by the finite limit $\cF$-sketch $(\bC,
\cS)$ with objects, morphisms and 2-cells in the diagrams \todo{Mark out which morphisms are supposed to be loose,
which I'm currently not sure about.}
\begin{mathpar}
  \hspace*{-2.5em}\begin{codi}[hexagonal=horizontal side 7em angle 50] 
    \obj{\cU & \cC^\to \\ & \cC \\};
    \mor  cU \chi:≈> (cC ^to) \cod_\cC:-> cC;
    \mor[swap] cU u:-> cC;
  \end{codi}
  \and
  \begin{codi}
    \obj{\cC^\to \\ \cC\\ };
    \mor[swap]:[bend right=36mu] (cC ^to) \dom_\cC:-> cC;
    \mor :[bend left=36mu] * \cod_\cC:-> *;
    \mor (dom _cC) \lambda_\cC:-2> (cod _cC);
  \end{codi}
  \and
  \begin{codi} 
    \obj{\cC \comma u & \cU \\ |(cC1)| \cC & |(cC2)| \cC \\};
    \mor (cC comma u) \cod_\cC^u:-> cU u:-> cC2;
    \mor[swap] * \dom_\cC^u:-> cC1 1_\cC:-> *;
    \mor cC1 \lambda_\cC^u:-2> cU;
  \end{codi}
  \and
  \begin{codi}[tetragonal=base 6em height 5em] 
    \obj{|(A)| \cC \comma u \\ |(B)| \cU \\};
    \mor[swap]:[bend right] A r:rightsquigarrow B;
    \mor:[bend left] A ["\cod^u_\cC",name=cod]:≈> B;
    \mor[] r \ell:[-2>, shorten=0.4em] cod;
  \end{codi}
  \and
  \begin{codi}[tetragonal=base 7em height 4.5em]
    \obj{|(A)| \cU \comma r & |(B)| \cC \comma u \\ |(C)| \cU & |(D)| \cU \\};
    \mor A \cod_\cU^r:-> B r:-> D;
    \mor[swap] * \dom_\cU^r:-> C 1_\cU:-> *;
    \mor C \lambda_\cU^r:-2> B;
  \end{codi}
  \and
  \begin{codi}[tetragonal=base 7em height 4.5em]
    \obj{|(A)| \cC^\to \comma \chi r & |(B)| \cC \comma u \\ |(C)| \cC^\to & |(D)| \cC^\to \\};
    \mor A {{\cod_{\cC^\to}^{\chi r}}}:-> B \chi r:-> D;
    \mor[swap] * {{\dom_{\cC^\to}^{\chi r}}}:-> C 1_\cU:-> *;
    \mor C {{\lambda_{\cC^\to}^{\chi r}}}:-2> B;
  \end{codi}
  \and
  \begin{codi}[tetragonal=base 8em height 4.5em]
    \obj { |(Y)| \cU \comma r & |(X)| \cC \comma u & |(A)| \cU \\ & & |(B)| \cC \\};
    \mor X [mid, "\cod^u_\cC",name=x]:[->,bend left] A ["u",name=p]:-> B;
    \mor X [mid,"r",name=y,swap]:[->,bend right] A;
    \mor y \ell:[-2>,shorten=-0.2em, slide=0.3em] x;
    \mor Y ["\dom^r_\cU",name=a]:[->, bend left=5em] A;
    \mor Y ["\cod^r_\cU",name=g]:-> X;
    \mor Y ["u\dom^r_\cU",name=pa,swap]:[->, bend right=2em] B;
    \mor[swap] a {{\ell\cod^r_\cU \circ \lambda^r_\cU}}:[-2>,slide=0.1em] X;
    \mor[swap] pa {{u\lambda^r_\cU}}:[-2>,slide=0.4em, shorten=0.4em] X;
  \end{codi}
  \and
  \begin{codi}[tetragonal=base 8em height 4.5em]
    \obj { |(Y)| \cC^\to \comma \chi r & |(X)| \cC \comma u & |(A)| {\cC^\to} \\ & & |(B)| \cC \\};
    \mor X [mid, "{\chi\cod^r_\cU}",name=x]:[->,bend left] A ["\cod_\cC",name=p]:-> B;
    \mor X [mid,"\chi r",name=y,swap]:[->,bend right] A;
    \mor y \chi\ell:[-2>,shorten=-0.2em, slide=0.3em] x;
    \mor Y ["{\dom^{\chi r}_{\cC^\to}}",name=a]:[->, bend left=5em] A;
    \mor Y ["{\cod^{\chi r}_{\cC^\to}}",name=g]:-> X;
    \mor Y ["{\cod_\cC\dom^{\chi r}_{\cC^\to}}",name=pa,swap]:[->, bend right=2em] B;
    \mor[swap] a {{\chi\ell\cod^{\chi r}_{\cC^\to} \circ \lambda^{\chi r}_{\cC^\to}}}:[-2>,slide=0.1em] X;
    \mor[swap] pa {{\cod_\cC \lambda^{\chi r}_{\cC^\to}}}:[-2>,slide=0.4em, shorten=0.4em] X;
  \end{codi}
\end{mathpar}
subject to the equations
\begin{mathpar}
  ur = \dom^u_\cC \and
  u\ell = \lambda^u_\cC \and
\end{mathpar}
and has as cones those making $\cC^\to$ a cotensors with the tight arrow category $2$, $\cC \comma u$ the comma
object of $1_\cC$ and $u$, $\cU \comma r$ the comma object of $1_\cU$ and $r$, $\cC^\to \comma \chi r$ the comma
object of $\cC^\to \comma \chi r$, and the cones making $\ell$ and $\chi \ell$ cartesian morphisms, i.e. the last
two diagrams above.

We also need that $\chi$ preserves the cartesian lifts, which amounts to requiring that $\chi^\to r$ is cartesian. This
is entirely algebraically characterisable by \cref{prop:cartesian by limit}.

A model for $\Comp$ in $\Cat^-$ is exactly a (cloven) comprehension category, a tight morphism of models is a strict
homomorphism of comprehension categories, and a loose morphism is a pseudo homomorphism of comprehension categories.

Using $t : \cC \comma u \to \cU^\to$ denote the morphism induced by $u\lambda_\cU : u\dom_\cU \to u\cod_\cU$, we
see that $u$ is a discrete fibration if $rt = \dom^u_\cC$ and $\ell t = 1_{1_\cU}$, capturing that the lift of the
image of a generic morphism in $\cU$ under $u$ is that same generic morphism. Adding this morphism $t$ and the comma
object and cone for $\cU^\to$ as well as these equations to the sketch for comprehension categories, we then specify
the comprehension categories whose fibration of types are discrete, also known as categories with attributes. We
denote the theory generated by this augmeneted sketch by $\Attr$, the theory of categories with attributes.

\section{Natural Models}
Similarly to comprehension categories, natural models are defined as certain fibrations over a base category.
Formally, they consist of a diagram in $\Cat$ of the shape:
\[\begin{codi}[hexagonal=horizontal side 6em angle 45] 
  \obj{\dot\cU &   \cU \\ & \cC \\};
  \mor  (dot cU) ["\Sigma",swap]:-> cU u:-> cC;
  \mor[swap] (dot cU) \dot u:-> cC;
  \mor[swap] cU \Delta:[bend right,->] (dot cU);
  \mor Sigma [shorten=0.3em, |-] Delta;
\end{codi}\]
where the triangle commutes, and $u, \dot u$ are discrete fibrations. As with comprehension categories and
categories with attributes, this is clearly describable in the framework of enhanced limit 2-sketeches, and
so we can freely generated a logic $\NM$ whose models in $\Cat$ are natural models.

\todo{Consider going through weakening-contraction comonads?}

\begin{proposition}
  The logics $\NM$ and $\Attr$ are 2-equivalent as 2-categories.
\end{proposition}
\begin{proof}
  We first construct the $\cF$-functor $F : \Attr \to \NM$, for which it suffices to construct a model of
  $\Attr$ inside $\NM$. Assuming the fibration of types should stay the same, this means we must find a
  morphism $\chi : \cU \to \cC^\to$ such that $\cod_\cC\chi = u$, which by the universal property of
  $\cC^\to$ is equivalent to finding parallel morphisms $\chi_1,\chi_2 : \cU \to \cC$ and a 2-cell $\phi
  : \chi_1 \to \chi_2$, such that $\chi_2 = u$. Let $\epsilon : \Sigma\Delta \to 1_\cU$ denote the unit
  of the adjunction $\Sigma \dashv \Delta$. Then taking $\chi_1 = u\Sigma\Delta$ and $\phi = u\epsilon
  : u\Sigma\Delta \to 1_\cU$, we have the desired universal map $\chi : \cU \to \cC^\to$, which gives
  us a model of $\Attr$ inside $\NM$. 

  In the other direction we require a $\cF$-functor $G : \NM \to \Attr$, so we construct a model of $\NM$
  inside $\Attr$. As this construction is quite a bit more involved, we will start by laying out some basic
  notation. Since we will deal with both the arrow object $\cC^\to$ and the comma object $\cC \comma u$, we
  need to distinguish their universal cones. We write $\lambda : \dom \to u\cod : \cC \comma u \to \cC$ for
  the universal cone belonging to $\cC\comma u$ and $\lambda' : \dom' \to \cod' : \cC^\to \to \cC$ for the
  universal cone for $\cC^\to$. We write $e \coloneq \dom'\chi$ and $q \coloneq \lambda' \chi : e \to u :
  \cU \to \cC$, and for a generalized element $A : X \to \cU$, will write $\Gamma \vdash A$ to denote that
  $\Gamma \coloneq uA$. Furthermore, we write $\Gamma.A$ for $eA$. By the construction of $\Attr$, we have
  a fixed cleavage $(r,\ell)$ showing that $u$ is a fibration, where $r : \cC \comma u \to \cU$ is such
  that $ur = \dom$ and $\ell : r \to \cod : \cC\comma u \to \cU$ is $u$-cartesian with $u\ell = \lambda$.
  Furthermore, we have that $\chi\ell : \chi r \to \chi\cod$ is $\cod$-cartesian, since that is how we capture
  that $\chi$ preserves cartesian-ness of the cartesian lifts for $u$.

  We will start with constructing the object of terms $\dot\cU$. Intuitively, the terms over a context
  $\Gamma : X \to \cC$ ought to be sections to some context projection $q A$ for some $\Gamma \vdash A$,
  and we encode this as a pullback. First we need an object of sections, however. Let $S$ denote the tight
  walking section (an object of $\cF$), so it has two tight objects $s,t \in S$ and morphisms $f : x \to y$
  and $g : y \to x$ with $gf = 1_x$. The power object $\cC^S$ consists of sections in $\cC$, in the sense
  that any map $X \to \cC^S$ is the same data as a section in the $\cF$-category $X \to \cC$. Now the object
  of terms is defined as the pullback
  \[\begin{codi}
    \obj {\dot\cU & \cC^S \\ \cU & \cC^\to \\};
    \mor (dot cU) p:-> (cC ^S) \cC^g:-> (cC ^to);
    \mor[swap] (dot cU) \Sigma:-> cU \chi:-> (cC ^to);
  \end{codi}\]
  where $\cC^g : \cC^S \to \cC^\to$ is the projection induced by inclusion $g : 2 \to S$ in $\cF$. We next
  define $\dot u = u\Sigma$, as that is a required equation for $\dot\cU$ to be a fibration of terms in
  a natural model. We will use $\Gamma \vdash a : A$ to denote an generalized element $a : X \to \dot\cU$,
  with $A = \Sigma a$ and $\Gamma = uA$. Thus if we have an element $\Gamma \vdash a : A$, we also have
  $\Gamma \vdash A$ and $\Gamma : X \to \cC$.

  To show that $\dot u$ is a fibration, we need for every $\Gamma \vdash a : A$ and $\sigma : \Delta \to
  \Gamma : X \to \cC$ a term $\Delta \vdash a[\sigma] : A[\sigma]$, where $\Gamma \vdash A[\sigma]$ is
  $r(\sigma,A)$. Using $\bar a : \Gamma \to \Gamma.A$ to denote the section of $q\Sigma$ induced by $p$,
  we see that we require a section $\bar a[\sigma] : \Delta \to \Delta.A[\sigma]$ to $qA[\sigma]$.
  Since $\chi\epsilon(\sigma,A) : \Delta.A[\sigma] \to \Gamma.A$ is cartesian, we find that the following
  square is a pullback:
  \[\begin{codi}[tetragonal=base 8em height 4.5em]
    \obj {|(tl)| \Delta.A[\sigma] & |(tr)| \Gamma.A \\ |(bl)| \Delta & |(br)| \Gamma\\};
    \mor tl {{e\epsilon(\sigma,A)}}:-> tr qA:-> br;
    \mor[swap] tl {{qA[\sigma]}}:-> bl \sigma:-> br;
  \end{codi}\]
  Since $qA \circ \bar a = 1_\Gamma$, we have that $\sigma \circ 1_\Delta = qA \circ \bar a\sigma$, which
  induces a universal morphism $\bar a[\sigma] : \Delta \to \Delta.A[\sigma]$ which satisfies $qA[\sigma]
  \circ \bar a[\sigma] = 1_\Delta$, as desired. Now the pair $(\bar a[\sigma], qA[\sigma])$ constitutes a
  morphism $pa[\sigma] : X \to \cC^S$ such that $\cC^g pa[\sigma] = \chi A[\sigma]$, which means we have
  a morphism $a[\sigma] : X \to \dot\cU$ with $\Sigma a[\sigma] = A[\sigma]$ and $p a[\sigma]$ as just
  defined. We also require a cartesian 2-cell $\dot \epsilon (\sigma,a) : a [\sigma] \to a$. \todo{Show that
  this exists.}


  Next we require a right adjoint $\Delta$ to $\Sigma$. This requires us to find maps $w : \cU \to \cU$ and
  $v : \cU \to \cC^S$ such that $\chi w = \cC^g v$. Let $\bar q : \cU \to \cC \comma u$ denote the universal
  map induced by $q : e \to u1_\cU : \cU \to \cC$, so that $\lambda\bar q = q$. Take $w = r\bar q : \cU \to
  \cU$. Since $\chi\ell\bar q : \chi r\bar q \to \chi\cod\bar q = \chi$ is $\cod$-cartesian, we have the
  following pullback square in $\cU \to \cC$
  \[\begin{codi}[tetragonal=base 6em height 4.5em]
    \obj {er\bar q & |(e1)| e \\ |(e2)| e & u \\};
    \mor (erbar q) -> e1 -> u;
    \mor[swap] * -> e2 -> *;
  \end{codi}\]
  where the left map is $\lambda'\chi r\bar q = qr\bar q$, the right map is $\lambda'\chi = q$, the top map
  is $\dom'\chi\ell\bar q = e\ell\bar q$, and the bottom map is $\cod'\chi\ell\bar q = u\ell\bar q = \lambda
  \bar q = q$. But since the square is a pullback, we have a map $v_1 : e \to er\bar q$ induced by $(1_e, 1_e)$
  with the property that $qr\bar q \circ v = 1_e$. Thus we have an induced map $v = (v_1,v_2) : \cU \to \cC^S$
  where $v_2 = r\bar q$, and so also $\lambda'\cC^g v = v_2$, so that $\cC^g v = \chi r\bar q = \chi w$, as
  desired. Thus we have an induced map $\Delta$ with $\Sigma\Delta = w = r\bar q$ and $p\Delta = v$.
  
  \todo{Show that $\Delta$ is a a right adjoint to $\Sigma$.}
\end{proof}

\section{Inductive Types}
We will consider mainly natural models.
\subsection{The empty type}
The empty type $\bzero$ is generally given by the following rules:
\begin{mathpar}
  \infer{\Gamma \ctx}{\Gamma \vdash \bzero}
  \and
  \infer{\Gamma.\bzero \vdash A\;\cU \\ \Gamma \vdash t : \bzero}{\Gamma \vdash \ind_\bzero(A,t) : A(t)}
\end{mathpar}
We translate these as loose morphisms $\bzero : \ctx \to \ty$ and $\ind_\bzero : \ctx \pb[(e\bzero,\bzero)]{
(\rty \times \Sigma)} (\ty \times \tm) \to \tm$ such $\Sigma \ind_\bzero = r_{\ty}$that the following square commutes:
  \[\begin{codi}[tetragonal=base 8em height 5em]
    \obj {|(tl)| \ctx \pb[(1_{\ctx}.\bzero,\bzero)]{(\rty \times \Sigma)} (\ty \times \tm) & |(tr)| \tm \\
    |(bl)| \ctx \pb[1]{2} & |(br)| \ty \\};
\end{codi}\]
\todo{Maybe consider conclusions as being a pullback of the premise, and then the rule as a section of the induced
morphism?}

The computation rules are encoded as equalities on the morphisms.

% using $\tm_\bzero$ to denote the pullback $\tm \times_\ty \bzero$, we encode the premise of the induction rule as the pullback $\times \tm_\bzero$ $\mathsf{ind}_{\bzero} : (\tm \pb[]{\ty} \bzero)$

\section{Multimode and Modal type theories}
Multimode dependent type theory (MTT) was introduced in \cite{gratzer2021} to provide a general family of type theories
with different modalities. MTT is parameterized by a 2-category $\cM$, a so-called mode-theory, where the objects are
\emph{modes}, 1-cells are \emph{modalities}, and 2-cells are \emph{operations} between modalities.

At each mode $m \in \cM$, we have an instance of a normal dependent type theory at that mode, and the modalities allow for
transporting terms at one mode to another. In constructing a logic capturing MTT, we encode this by, for each mode
$m \in \cM$, giving a diagram
\[\begin{codi}[hexagonal=horizontal side 6em angle 45] 
  \obj{\tm_m &   \ty_m \\ & \ctx_m \\};
  \mor  (tm _m) ["\Sigma_m",swap]:≈> (ty _m) \rty:-> (ctx _m);
  \mor[swap] (tm _m) \rtm_m:-> (ctx _m);
  \mor[swap] (ty _m) \Delta_m:[bend right,≈>] (tm _m);
  \mor (Sigma _m) [shorten=0.3em, |-] (Delta _m);
\end{codi}\]
and all diagrams, equalities, and cones needed to turn this into a natural model. Then, for each modality $\mu : m \to
n \in \cM$, we add a morphism $\lock_\mu : \ctx_n \loose \ctx_m$, corresponding to the rule
\[\infer{\Gamma \ctx_n}{\Gamma. \lock_\mu \ctx_m}\]
To model the type former for modal types $\Gamma \ctx_m \vdash [\mu] A \ty_m$ for a type $\Gamma.\lock_\mu \ctx n
\vdash A \ty$ and modality $\mu : m \to n$, we add a loose 1-cell $[\mu] : \ctx_m \pb[\lock_\mu\mkern-6mu]{\rty_n}
\loose \rty_m$ subject to the condition that $\rty_m [\mu] = \pi_1$


\newpage

\ifSubfilesClassLoaded{\printbibliography}{}
\end{document}
