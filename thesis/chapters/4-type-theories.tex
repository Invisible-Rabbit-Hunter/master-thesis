\documentclass[../thesis.tex]{subfiles}

\ifSubfilesClassLoaded{
  \externaldocument{../build/2-preliminaries}%
  \externaldocument{../build/3-logics}%
}

\begin{document}
\chapter{Type Theories}\label{ch:type theories}
As was explained in the introduction, there have been a whole host of different categorical structure
that have been introduced to model type dependency and type theories throughout the literature, ranging
from display map categories to comprehension categories to categories with families and natural models.
The goal of this chapter is to show how such structures in can in fact be described as finite 2-limit
theories. Since there are so many different models, and covering them all would be both time consuming
and reduntant, we will focus only on two sorts of models:
\begin{enumerate}
  \item Comprehension categories and categories with attributes. The former was introduced by \citeauthor{jacobs1993}
    in \cite{jacobs1993} as a minimal categorical model for type dependency, and was there also shown to generalize
    the latter. The latter was introduced by \citeauthor{cartmell1984} in \cite{cartmell1984}.
  \item Categories with Families and Natural Models, which are really two different presentations of the same data.
    The former were introduced by \citeauthor{dybjer1996} as an alternative to categories with attributes which was
    closer to the syntax of type theory, and the latter introduced by \citeauthor{awodey2017} as a reformulation of
    categories with families to make formulating the rules of type theory easier in terms of universal properties.
\end{enumerate}

In essence, what we aim to show here is that models of type theories can be identified with 2-functors from a
2-category describing the theory into $\Cat$, and that the homomorphisms can be identified certain pseudo-natural
transformations between these functors. This is similar in spirit to \cite{uemura2023}, where \citeauthor{uemura2023}
introduced categories with representable maps for a similar aim. The idea there was that, since representable maps of
discrete fibrations are natural models, we can describe our type theory abstractly as a finitely complete
category equipped with an additional class of maps closed under certain constructions mimicing that of
the representable maps. Then a model is category of contexts $\cC$ and a finite limit preserving functor
into the discrete fibrations over $\cC$ mapping the representable maps of the theory to representable
maps of fibrations. 

In contrast, our here theories are finitely $\sF$-complete $\sF$-categories, the models are $\sF$-functors into
$\Cat^+$ preserving finite $\sF$-limits, and the homomorphisms are certain transformatons. The additional data
$\sF$-categories over 2-categories allows us finer control over the strictness of homomorphisms, which is what
allows us to recover the standard notions of homomorphisms of, for example, comprehension categories.
Note that an $\sF$-limit theory is a 2-theory, but the additional data of tight morphisms in $\sF$-categories allows
us fine control of the strictness of the homomorphisms. We will use this to show how we can recover the standard
notions of homomorphism of models, such as the pseudo morphisms of comprehension categories, just from the $\sF$-limit
theory. To further show the utility of this approach, we further show how to model more complex features of type
theory, such as inductive types and the modes and modal types of multimode type theory. Note that the latter is
not readily modeled in categories with representable maps, due to the requirement of multiple categories of contexts,
one for each mode.

\section{Comprehension Categories and Categories with Attributes}
The notion of comprehension categories was introduced in \cite{jacobs1993} to provide a minimal categorical
description of type dependency. They concist of a commutative triangle in $\Cat$ of the following form:
\[\begin{codi}[hexagonal=horizontal side 6em angle 45] 
  \obj{\cU &   \cC^\to \\
           & \cC \\};
  \mor  cU \chi:-> (cC ^to) \cod:-> cC;
  \mor[swap] cU u:-> cC;
\end{codi}\]
where $u: \cU \to \cC$ is a fibration, $\cC^\to$ is the arrow category of $\cC$, and $\chi$ preserves cartesian
morphisms. As a model of type theory, we want to think of the fiber $u_\Gamma$ of $u$ at a context $\Gamma \in
\cC$ as the category of types in the context $\Gamma$, so the objects are types $\Gamma \vdash A\;\cU$. The
reindexing provided by $u$ being a fibration then models the substitution of types, and the comprehension map
$\chi$ models context extension, with the preservation of cartesian morphisms giving the expected universal
property thereof, in the sense that for any type $\Gamma \vdash A\;\cU$ and substitution $\sigma : \Delta \to
\Gamma$, the following is a pullback square in $\cC$:
\[\begin{codi}
  \obj { \Delta.A[\sigma] & \Gamma.A \\ \Delta & \Gamma \\ };
  \mor (Delta A[sigma ]) \bar \sigma.A:-> (Gamma A) \pi A:-> Gamma;
  \mor[swap] (Delta A[sigma ]) {{\pi A[\sigma]}}:-> Delta \sigma:-> Gamma;
\end{codi}\]
where we write $\pi$ for the 2-cell $\lambda\chi : \dom\chi \to u$ induced by $\chi$.

To axiomatise this as a finite $\sF$-limit theory, we first consider the 2-categorical structure and limits that
are relevant. We already have a characterisation of fibrations in terms of finite 2-limits, and since $\chi$ is
supposed to preserve cartesian morphisms, it suffices to only ask for preservation of the cartesian lifts; any
cartesian morphism is, after all, isomorphic to a cartesian lift. This can be captured by asking that $\chi\ell$
is cartesian, where $\ell$ is the generic cartesian lift, or cleavage, specified by $u$ being a fibration. We can
thus see that all data in the structure of a comprehension category can be described in terms of finite 2-limits.
Next we consider the homomorphisms. In general, maps of fibrations are strict commuting squares, which tells us
that in our theory, $u$ should be a tight map. By simlar reasoning, we find that $\cod$ also should be tight, which
is further supported by taking $\cC^\to$ to be the corresponding tight limit, for which the codomain projection is
necessarily tight. However, there is no reason to ask that homomorphisms preserve context extension strictly, and
in fact, the weak morphisms that appear in the literature do not require this (see \cite{ahrens2024}). Instead, we
ask for preservation only up to isomorphism, and so take the comprehension map $\chi$ to be a loose morphism.

Taking all of these properties into account leads us to the following definition of the $\sF$-theory of comprehension
categories: 
\begin{definition}
  We define $\Comp$ to be the finite $\sF$-limit theory generated by the sketch consisting of diagrams
  \begin{mathpar}
    \hspace*{-2.5em}\begin{codi}[hexagonal=horizontal side 7em angle 50] 
      \obj{\cU & \cC^\to \\ & \cC \\};
      \mor  cU \chi:≈> (cC ^to) \cod_\cC:-> cC;
      \mor[swap] cU u:-> cC;
    \end{codi}
    \and
    \begin{codi}
      \obj{\cC^\to \\ \cC\\ };
      \mor[swap]:[bend right=36mu] (cC ^to) \dom_\cC:-> cC;
      \mor :[bend left=36mu] * \cod_\cC:-> *;
      \mor (dom _cC) \lambda_\cC:-2> (cod _cC);
    \end{codi}
    \and
    \begin{codi} 
      \obj{\cC \comma u & \cU \\ |(cC1)| \cC & |(cC2)| \cC \\};
      \mor (cC comma u) \cod_\cC^u:-> cU u:-> cC2;
      \mor[swap] * \dom_\cC^u:-> cC1 1_\cC:-> *;
      \mor cC1 \lambda_\cC^u:-2> cU;
    \end{codi}
    \and
    \begin{codi}[tetragonal=base 6em height 5em] 
      \obj{|(A)| \cC \comma u \\ |(B)| \cU \\};
      \mor[swap]:[bend right] A r:≈> B;
      \mor:[bend left] A ["\cod^u_\cC",name=cod]:-> B;
      \mor[] r \ell:[-2>, shorten=0.4em] cod;
    \end{codi}
    \and
    \begin{codi}[tetragonal=base 7em height 4.5em]
      \obj{|(A)| \cU \comma r & |(B)| \cC \comma u \\ |(C)| \cU & |(D)| \cU \\};
      \mor A \cod_\cU^r:-> B r:≈> D;
      \mor[swap] * \dom_\cU^r:-> C 1_\cU:-> *;
      \mor C \lambda_\cU^r:-2> B;
    \end{codi}
    \and
    \begin{codi}[tetragonal=base 7em height 4.5em]
      \obj{|(A)| \cC^\to \comma \chi r & |(B)| \cC \comma u \\ |(C)| \cC^\to & |(D)| \cC^\to \\};
      \mor A {{\cod_{\cC^\to}^{\chi r}}}:-> B \chi r:≈> D;
      \mor[swap] * {{\dom_{\cC^\to}^{\chi r}}}:-> C 1_\cU:-> *;
      \mor C {{\lambda_{\cC^\to}^{\chi r}}}:-2> B;
    \end{codi}
    \and
    \begin{codi}[tetragonal=base 8em height 4.5em]
      \obj { |(Y)| \cU \comma r & |(X)| \cC \comma u & |(A)| \cU \\ & & |(B)| \cC \\};
      \mor X [mid, "\cod^u_\cC",name=x]:[->,bend left] A ["u",name=p]:-> B;
      \mor X [mid,"r",name=y,swap]:[≈>,bend right] A;
      \mor y \ell:[-2>,shorten=-0.2em, slide=0.3em] x;
      \mor Y ["\dom^r_\cU",name=a]:[->, bend left=5em] A;
      \mor Y ["\cod^r_\cU",name=g]:-> X;
      \mor Y ["u\dom^r_\cU",name=pa,swap]:[->, bend right=2em] B;
      \mor[swap] a {{\ell\cod^r_\cU \circ \lambda^r_\cU}}:[-2>,slide=0.1em] X;
      \mor[swap] pa {{u\lambda^r_\cU}}:[-2>,slide=0.4em, shorten=0.4em] X;
    \end{codi}
    \and
    \begin{codi}[tetragonal=base 8em height 4.5em]
      \obj { |(Y)| \cC^\to \comma \chi r & |(X)| \cC \comma u & |(A)| {\cC^\to} \\ & & |(B)| \cC \\};
      \mor X [mid, "{\chi\cod^r_\cU}",name=x]:[≈>,bend left] A ["\cod_\cC",name=p]:-> B;
      \mor X [mid,"\chi r",name=y,swap]:[≈>,bend right] A;
      \mor y \chi\ell:[-2>,shorten=-0.2em, slide=0.3em] x;
      \mor Y ["{\dom^{\chi r}_{\cC^\to}}",name=a]:[->, bend left=5em] A;
      \mor Y ["{\cod^{\chi r}_{\cC^\to}}",name=g]:-> X;
      \mor Y ["{\cod_\cC\dom^{\chi r}_{\cC^\to}}",name=pa,swap]:[->, bend right=2em] B;
      \mor[swap] a {{\chi\ell\cod^{\chi r}_{\cC^\to} \circ \lambda^{\chi r}_{\cC^\to}}}:[-2>,slide=0.1em] X;
      \mor[swap] pa {{\cod_\cC \lambda^{\chi r}_{\cC^\to}}}:[-2>,slide=0.4em, shorten=0.4em] X;
    \end{codi}
  \end{mathpar}
  subject to the equations
  \begin{mathpar}
    u = \cod_\cC \chi \and
    ur = \dom^u_\cC \and
    u\ell = \lambda^u_\cC \and
  \end{mathpar}
  and with limit cones those making $\cC^\to$ a power with the tight arrow category $2$, $\cC \comma u$ the
  comma object of $1_\cC$ and $u$, $\cU \comma r$ the comma object of $1_\cU$ and $r$, $\cC^\to \comma \chi r$
  the comma object of $\cC^\to \comma \chi r$, and the cones making $\ell$ and $\chi \ell$ cartesian morphisms,
  i.e. the last two diagrams above.
\end{definition}

A model for $\Comp$ in $\Cat$ is exactly a (cloven) comprehension category, and given two comprehension categories
$(\cC, \cU, u, \chi)$ and $(\cD, \cV, v, \psi)$, a homomorphism $f : (\cC, \cU, u, \chi) \to (\cD, \cV, v, \psi)$
is essentially a prism
\[\begin{codi}[hexagonal=horizontal side 6.5em angle 60]
  \obj {\cU & \cC^\to \\
            & \cC     \\};

  \obj [xshift=8em, yshift=-3em] {\cV & \cD^\to \\
                                   & \cD     \\};
  
  \mor cC u:<- cU \chi:-> (cC ^to) ["\cod_\cC",near end]:-> cC;
  \mor cD v:<- cV ["\psi",near start]:-> (cD ^to) \cod_\cD:-> cD;
  \mor (cC ^to) f_b^\to:-> (cD ^to);
  \mor cC ["f_b",swap]:-> cD;
  \mor cU ["f_e",swap]:[->,crossing over] cV;
  \node at ($(cC ^to)!0.5!(cV)$) {$\cong$};
\end{codi}\]
where all sides except the top one are commuting squares, the top one is a coherent isomorphism, and the pair
$(f_e, f_b)$ constitutes a map of fibrations. Finally, we see that the 2-cells $\alpha : f \to g$ in the 2-category of
pseudomorphisms of comprehension categories is a pair of natural transformations $\alpha_b : f_b \to g_b$ and
$\alpha_e : f_e \to g_e$ compatible with the top isomorphism in a suitable manner.
\begin{remark}
  In the definition of $\Comp$, we do not require homomorphisms to preserve cartesian lifts strictly, since the
  lifting map $r : \cC \comma u \to \cU$ is loose. This aligns with the traditional definition of maps of fibrations,
  where we only ask for preservation of cartesian maps, and so only that the image of a carteisan lift under the
  homomorphisms is isomorphic to a cartesian lift in the codomain. However, if we choose to make $r$ tight
  isntead, the homomorphisms to strictly preserve the choice of cartesian lifts, and so in the language of
  models of type theory, to strictly preserve substitution.

  This shows the flexibility that $\sF$-limit theories provide in the 2-categories of models: by carefully choosing
  what maps are tight and what maps are loose, we have precise control over what parts of the theory are preserved
  strictly by homomorphisms, and what parts are only up to isomorphism, without ad hoc definitions for every possible
  notion of homomorphisms we could come up with. Furthermore, we can naturally recover the different notions of
  homomorphisms present in literature, such as the weak maps of comprehension categories, the strict maps, the
  lax maps, etc. With a bit of care, we could give a theory of split comprehension categories, and then ask for maps
  which preserve the splitting strictly or loosely. 
\end{remark}

Using $t : \cC \comma u \to \cU^\to$ denote the morphism induced by $u\lambda_\cU : u\dom_\cU \to u\cod_\cU$,
we see that $u$ is a discrete fibration if $rt = \dom^u_\cC$ and $\ell t = 1_{1_\cU}$, capturing that the lift
of the image of a generic morphism in $\cU$ under $u$ is that same generic morphism. Adding this morphism $t$
and the comma object and cone for $\cU^\to$ as well as these equations to the sketch for comprehension categories,
we then specify the comprehension categories whose fibration of types are discrete, also known as categories with
attributes. We denote the theory generated by this augmeneted sketch by $\Attr$, the theory of categories with
attributes.
\begin{definition}
  We let $\Attr$ be the finite limit $\sF$-theory generated as $\Comp$, but with the diagrams and cones expressing
  that $\ell$ is cartesian with diagrams and cones expressing that the lifting $(r,\ell)$ is the cleavage of a discrete
  fibration.
\end{definition}

\subsection{Terms in comprehension categories}\label{subsec:compcat terms}
In \cref{sec:nm/equiv attr}, we will show that natural models are the same as categories with attributes. However,
unlike natural models, the notion of term is not inherently part of the definition of comprehension categories, and so
they need to be derived in some way to show this equivalence. A common interpretation is that a term $\Gamma \vdash a
: A$ of type $A : X \to \cU$ in a context $\Gamma = uA : X \to \cC$ is a section to the projection map $\pi A : \set{A}
\to A$, as is defined in \cite{jacobs1993}. Using finite 2-limits, we can in fact construct an object $\dot\cU$ with the
property that any map $a : X \to \dot\cU$ induces a unique section $\tau a : u\typ a \to \set{\typ a}$ of $\pi\typ a$,
for some adequately chosen $\typ : \dot\cU \to \cU$. 

\begin{definition}
  In $\Comp$, let $\dot\cU$ denote the $\sF$-limit of the diagram
  \[\begin{codi}
    \obj {\cU & \cC \\};
    \mor:[bend left] cU u:-> cC;
    \mor[swap]:[bend right] cU ["{\set{-}}",name=c]:-> cC;
    \mor c \pi:-2> u;
  \end{codi}\]
  with weight
  \[\begin{codi}
    \obj {1 & S \\};
    \mor:[bend left] 1 a:-> S;
    \mor[swap]:[bend right] 1 b:-> cC;
    \mor b r:-2> a;
  \end{codi}\]
  where $S = \set{s : a \rightleftarrows b : r}$ is the walking section/retraction pair, so that $rs = 1_a$.

  We denote the universal map $\dot\cU \to \cU$ by $\typ$, and the universal 2-cell $u\typ \to \set{\typ}$
  by $\tau$. Note that by definition, we thus have that $\pi \typ \circ \tau = 1_\typ$, so that $\tau$ is
  a generic term, and any such section induces a universal map into $\dot\cU$.
\end{definition}

With this definition of terms comes also a notion of 2-cells of terms. From the definition of 2-limits, we
see that a 2-cell $\alpha : a \to b : X \to \dot\cU$ is equivalent to a 2-cell $\typ\alpha : \typ a \to \typ
b$ such that the diagram below commutes:
\[\begin{codi}
  \obj {|(tl)| u\typ a & |(tm)| \set{\typ a} & |(tr)| u \typ a \\ |(bl)| u \typ b & |(bm)| \set{\typ b} & |(br)| u\typ b \\};
  \mor tl \tau a:-> tm \pi a:-> tr u\typ\alpha:-> br;
  \mor[swap] tl u\typ\alpha:-> bl \tau b:-> bm \pi\typ b:-> br;
  \mor tm {{\set{\typ\alpha}}}:-> bm;
\end{codi}\]
Note that the right square commutes trivially by the exchange law applied to $\pi : \set{-} \to u$ and $\typ\alpha : \typ a
\to \typ b$, so that really a map of terms $\alpha : a \to b : X \to \dot\cU$ is the same as a map of types $\typ\alpha :
\typ a \to \typ b$ such that the above left square commutes.

Furthermore, we have a nice relation between terms and substitution in the form of the following lemma:
\begin{lemma}\label{lem:comp term of substitution}
  Let $A : X \to \cU$ and $\sigma : \Gamma \to uA$ be given. Then there is a bijective correspondence between maps
  $\alpha : \Gamma \to \set{A}$ such that $\pi A \circ \alpha = \sigma$ and terms of type $\Gamma \vdash a : A[\sigma]$
  (where we interpret $a$ as a section of $\pi(A[\sigma]) : \set{A[\sigma]} \to \Gamma$)
\end{lemma}
\begin{proof}
  Consider the diagram
  \[\begin{codi}[tetragonal=base 6em height 4.5em]
    \obj {|(Gamma1)| \Gamma \\ & |(cAsigma)|\set{A[\sigma]} & |(cA)| \set{A} \\
                            & |(Gamma2)| \Gamma & uA \\
    };
    \mor Gamma1 \alpha:[bend left=5em,->,dotted] cA;
    \mor[swap] Gamma1 {{1_{\set{A}}}}:[bend right=5em,->] Gamma2;
    \mor cAsigma {{\set*{\vphantom{\big\vert}\overline{\sigma}}}}:-> cA \pi A:-> uA;
    \mor[swap] cAsigma {{\pi (A[\sigma])}}:-> Gamma2 \sigma:-> uA;
    \mor Gamma1 a:->,dashed cAsigma;
  \end{codi}\]
  Recall that the square to the bottom right is a pullback square, so that given $\alpha$, such a term $a$
  is uniquely determined by the universal property of pullbacks. In the other direction, any term $a$
  uniquely determines a $\alpha$ by composition with $\set*{\overline{\sigma}}$.
\end{proof}

This object of terms, and the maps $\typ$ and $\dot u \coloneq u\typ$, have some very nice properties. For example, we
can show that $\typ$ preserves and reflects cartesian morphisms, and that $\dot u$ is a fibration, so that we have a
notion of substitution on terms. We begin by showing that $\typ$ reflects cartesian morphisms, as that will be helpful
in showing the other properties.
\begin{lemma}\label{lem:comp typ reflect cartesian}
  The map $\typ$ reflects $u$-cartesian morphisms, in the sense that if $\alpha : a \to b : X \to \dot\cU$ is such that
  $\typ\alpha$ is $u$-cartesian, then $\alpha$ is $\dot u$-cartesian.
\end{lemma}
\begin{proof}
  Suppose we are given a $\dot u$-cartesian cone $(f,c,\phi, \sigma)$ over $\alpha$, so that $\dot u (\alpha f) \circ
  \sigma = \dot u \phi$. Then we also have that $u\typ\alpha f \circ \sigma = u\typ\phi$, since $\dot u = u \typ$ by
  definition. In particular, since $\typ\alpha$ is cartesian, there then exists a map $\hat\sigma : \typ c \to \typ a$
  with $\typ\alpha f \circ \hat\sigma = \typ\phi$ and $u\hat\sigma = \sigma$. If the square 
  \[\begin{codi}
    \obj {|(tl)| u\typ c & |(tm)| \set{\typ c} \\
    |(bl)| u \typ af & |(bm)| \set{\typ af}    \\};
    \mor tl \tau c:-> tm {{\set{\hat\sigma}}}:-> bm;
    \mor[swap] tl u\hat\sigma:-> bl \tau af:-> bm;
  \end{codi}\]
  commutes, then $\hat\sigma$ is a morphism of terms, as established earlier. Thus $\hat\sigma$ would be of the form
  $\typ\bar\sigma$ for some unique $\bar\sigma : c \to a$, and so by uniqueness of $\hat\sigma$, it would be the
  unique map such that $u\typ\bar\sigma = \sigma$ and $\alpha f \circ \bar\sigma = \phi$. To see that this holds,
  consider the following diagram:
  \[\begin{codi}
    \obj {|(tl)| u\typ c & |(tm)| \set{\typ c} \\
      |(ml)| u \typ af & |(mm)| \set{\typ af}  & |(mr)| u\typ a f \\
                       & |(bm)| \set{\typ bf} & |(br)| u\typ b f \\};
                       \mor tl \tau c:-> tm {{\set{\hat\sigma}}}:-> mm \pi\typ af:-> mr u\typ\alpha:-> br;
                       \mor[swap] tl u\hat\sigma:-> ml \tau af:-> mm {{\set{\typ \alpha}}}:-> bm \pi\typ bf:-> br;
  \end{codi}\]
  Since $\alpha$ and $\phi$ are maps of terms, we have that
  \[\set{\typ\alpha} \circ \tau a f \circ u\hat\sigma = \tau bf \circ u\typ\alpha \circ \sigma = \tau bf \circ u\typ\phi\]
  and
  \[\set{\typ\alpha} \circ \set{\hat\sigma} \circ \tau c = \set{\typ\alpha \circ \hat\sigma} \circ \tau c =
  \set{\typ\phi} \circ \tau c = \tau bf \circ u\typ\phi.\]
  Similarly we have that
  \[\pi\typ af \circ \set{\hat\sigma} \circ \typ\tau c = u\hat\sigma \circ \pi\typ c \circ \typ\tau c = \sigma
  = \pi\typ af \circ \tau af \circ u\hat\sigma.\]
  Since $\typ\alpha$ is cartesian, the bottom right square is a pullback, and so by
  \[u\typ\alpha \circ \sigma = u\typ\phi = \pi\typ b f \circ \tau bf \circ u\typ\phi\]
  there exists a unique map $h : u\typ c \to \set{\typ a f}$ such that $\pi\typ a f \circ h = \sigma$ and $\set{\typ\alpha}
  \circ h = \tau bf \circ u\typ\phi$. But as we saw, both $\set{\hat\sigma} \circ \tau c$ and $\tau a f \circ u\hat\sigma$
  satisfy this property, so they must be equal. Therefore the top-left square commutes.
\end{proof}

\begin{lemma}\label{lem:comp term fibration}
  The map $\dot u \coloneq u\typ : \dot\cU \to \cC$ is a fibration.
\end{lemma}
\begin{proof}
  We show that for any $u$-cartesian morphism $\alpha : A \to \typ b$ lifts to a $\dot u$-cartesian morphism $\typ\bar\alpha
  : a \to b$ with $\typ\bar\alpha = \alpha$. Note that this requires that $\typ a = A$, and so we must find a section $\tau a
  : uA \to \set{A}$ to $\pi A : \set{A} \to A$, and further this choice of section must make $\alpha$ into a map of terms $a
  \to b$. To find this section, consider the following diagram:
  \[\begin{codi}[tetragonal=base 7em height 4.5em]
    \obj {|(tl)| uA & |(tm)| \set{A} & |(tr)| uA \\
    |(bl)| u\typ b & |(bm)| \set{\typ b} & |(br)| u\typ b \\ };
    \mor tl {{\tau a}}:->,dashed tm {{\pi A}}:-> tr u\alpha:-> br;
    \mor tl {{1_{uA}}}:[->,bend left] tr;
    \mor[swap] tl u\alpha:-> bl \tau b:-> bm \pi \typ b:-> br;
    \mor tm {{\set{\alpha}}}:-> bm;
  \end{codi}\]
  Since $\alpha$ is $u$-cartesian, the right square is a pullback square, so by $\pi \typ b \circ \tau b \circ u\alpha = u\alpha$
  we have a unique map $\tau a : uA \to \set{A}$ with $\set{\alpha} \circ \tau a = \tau b \circ u\alpha$ and $\pi A \circ \tau
  a = 1_{uA}$. Thus $\tau a$ induces a term $a : X \to \dot \cU$ with $\typ a = A$, and furthermore $u\alpha$ induces a map
  of terms $\bar\alpha : a \to b$ such that $\typ\bar\alpha = \alpha$. Finally, since $\typ$ reflects cartesian morphisms and
  $\alpha$ is cartesian, it follows that $\bar\alpha$ is cartesian as well.

  Finally, to find a cartesian lift for $\sigma : \Gamma \to \dot u b$, we first lift it to a $u$-cartesian map $\hat\sigma : \typ
  b[\sigma] \to \typ b$, and then to a $\dot u$-cartesian map $\bar \sigma : b[\sigma] \to b$. Now note that $\dot u\bar\sigma =
  u\typ\bar\sigma = u\hat\sigma = \sigma$ is a $\dot u$-cartesian lift of $\sigma$, as desired.
\end{proof}

In preparation for \cref{sec:nm/equiv attr}, we will prove a few useful theorems about $\typ$:
\begin{lemma}\label{lem:comp typ preserves cartesian}
  $\typ$ preserves cartesian morphisms.
\end{lemma}
\begin{proof}
  Suppose $\alpha : a \to  ; X \to \dot\cU$ is $\dot u$-cartesian, and let $\alpha' : a' \to b$ denote the $\dot u$-cartesian
  lift of $\dot u\alpha$. Then there exists an isomorphism $h : a \to a'$ such that $\alpha' \circ h = \alpha$ and $\dot u h =
  1_{\dot u a}$ by the essential
  uniqueness of cartesian lifts, and by construction $\typ\alpha' : \typ a' \to \typ b$ is $u$-cartesian. But then since $\typ
  h$ is a vertical isomorphism, we see that $\typ \alpha = \typ \alpha' \circ \typ h$ is $u$-cartesian as well; to find the
  unique map for a $u$-cartesian cone over $\typ\alpha$, we simply find it for $\typ\alpha'$ and transport the solution accross
  $\typ h$.
\end{proof}

\begin{lemma}\label{lem:comp typ fully faithful}
  $\typ$ is faithful, in the sense that for any 2-cells $\alpha,\beta : a \to b : X \to \dot\cU$, if $\typ\alpha = \typ\beta$,
  then $\alpha = \beta$.
\end{lemma}
\begin{proof}
  This follows from the 2-dimensional universal property of $\bT$. 
\end{proof}


\section{Categories with Families}\label{sec:nm}
The second semantics we will fit into our framework is categories with families, which were introduced by
\citeauthor{dybjer1996} in \cite{dybjer1996} as a variant of Cartmell's categories with attributes with
closer ties to the syntax of type theory, in particular modeling terms as a presheaf by themselves, rather
than as a certain class of context maps. A further reformulation of CwFs, found independently by \citeauthor{
fiore2012} \cite{fiore2012} and \citeauthor{awodey2017} \cite{awodey2017}, gives context extension as a
representable map of presheaves between the presheaves of terms the presheaf of types, rather than as a
direct universal property as in \cite{dybjer1996}.  This reformulation, dubbed natural models in \cite{
awodey2017}, allows for one final change to fit into our framework: we replace the presheaves by discrete
fibrations. One advantage of this step is that representability of a map of presheaves is equivalent to the
existence of a right adjoint to the typing map between the categories of elements of the type and term
fibrations (as already noted in \cite{awodey2017}), which is easily axiomatized in our framework.

\begin{definition}[Natural models of type theory]
  A \emph{natural model of type theory}, in the sense of \cite{awodey2017}, is a diagram in $\Cat$ of the
  form
  \[\begin{codi}[hexagonal=horizontal side 6em angle 45] 
    \obj{\dot\cU &   \cU \\ & \cC \\};
    \mor  (dot cU) ["\typ",swap]:-> cU u:-> cC;
    \mor[swap] (dot cU) \dot u:-> cC;
    \mor[swap] cU \var:[bend right,->] (dot cU);
    \mor typ [shorten=0.3em, |-] var;
  \end{codi}\]
  where $u, \dot u$ are discrete fibrations and $\typ$ is a map of discrete fibrations, so that $\typ u = \dot u$.
  Note that $\var$ need not be, and generally is not, a map of fibrations.
\end{definition}
\begin{remark}
  In \cite{coraglia2024a}, the natural models are further generalized by weakening the discrete fibrations to
  arbitrary fibrations, and thus also asking that $\typ$ preserves cartesian morphisms and that the unit and
  counit of the adjunction $\typ \dashv \var$ are cartesian. In \cite{coraglia2024b}, these generalized natural
  models, there dubbed generalized categories with families, are shown to be biequivalent with comprehension
  categories. I conjecture that one can show the $\sF$-theory of these generalized categories with families
  biequivalent to the one for comprehension categories introduced earlier, with a proof similar to the one found
  in \emph{op.\ cit.} For now we treat only the case with discrete fibrations, as the theory of those is both
  simpler and more developed.
\end{remark}

As with comprehension categories, we already know how to specify the data for a natural model in an $\sF$-limit
sketch. For example, the 2-cell $u : \cU \to \cC$ is a discrete fibration precisly if the map $t : \cU^\to \to
\cC \comma u$ induced by $u \lambda_\cU : u\dom_\cU \to u\cod_\cU$ has an inverse $r : \cC \comma u \to \cU^\to$
satisfying $\cod_\cU r = \cod_\cC^u$, which is evidently describable by an $\sF$-sketch. We denote the $\sF$-%
theory induced by this sketch by $\NM$.
\begin{remark}
  Note that there are multiple equivalent ways of describing discrete fibrations in terms of finite 2-limits, and
  so multiple $\sF$-sketches for natural models. However, these $\sF$-sketches all induce equivalent $\sF$-theories,
  and so we speak of \emph{the} theory of natural models $\NM$.
\end{remark}
\begin{remark}
  Since we took much care in the definition of comprehension categories to show the ability of $\sF$-theories
  to precisely control which operations are preserved strictly and which are preserved loosely, and this often
  turns out to be quite verbose, we will in the following sections elide such considerations. However, it should
  be clear from the previous section that it is possible to recover whatever notion of homomorphisms of e.g.\
  natural models one wants by selecting the right tight morphisms. Note that we by default we will take the $\sF$-%
  limits to be as tight as possible.
\end{remark}


With the basic theory of natural models at hand, we will spend the remainder of this section showing how further
structure and properties, in particular dependent product types, dependent sum types, and extensional identity types,
can be similarly described in terms of finite 2-limits. This allows us to construct finite $\sF$-limit theories
for any type theory with these features, whose models are the natural models with these features. The specific
constructions we use are generally based on \cite{coraglia2024a}, rather than \cite{awodey2017}, as we have free
access to 2-pullbacks and similar 2-limits, but not the representable discrete fibrations and exponentiabilty of
representable maps of discrete fibrations.

Before we can show how to model the rules for these different type constructors, we will need a few useful definitions.
First, we will need a the map corresponding to context extension. In analogy to comprehension categories, we will denote
this morphism by $\set{-} : \cU \to \cC$, and we define it by $\set{-} \coloneq \dot u \var$. The counit of the adjunction
$\typ \dashv \var$ then gives us the expected projection map $\pi \coloneq u \epsilon : \set{-} = u\typ\var \to u$. We also
want to relate the terms $\dot\cU$ to substitutions. Specifically, we have a section $\tau : u\typ \to \set{\typ}$ of $\pi
\typ : \set{\typ} \to u\typ$ given by the unit, as in $\tau = \dot u \eta$. Thus every term $a : X \to \dot\cU$ with $A
\coloneq \typ a$ induces a substitution $\tau a : uA \to \set{A}$ for which $\pi A \circ \tau a = 1_{u A}$. In other words,
every term $a : X \to \dot\cU$ induces a term in the sense of comprehension categories.

\subsubsection{Dependent products}
Recall the formation and introduction rules for dependent product types:
\begin{mathpar}
  \inferrule{\Gamma \vdash A\;\cU \\ \Gamma.A \vdash B\;\cU}{\Gamma \vdash \Pi(A, B)\;\cU} \and
  \inferrule{\Gamma \vdash A\;\cU \\ \Gamma.A \vdash b : B}{\Gamma \vdash \lambda(A,b) : \Pi(A,B)}
\end{mathpar}
The formation rule tells us that given any type $A$ and type $B$ dependent on $A$, we can form the product type
$\Pi(A,B)$. We model this as a morphism from some premise judgement $P$, corresponding to the pair $A$ and $A \vdash B$,
to the judgement of types $\cU$. More formally, the premise is a judgement $p : P \to \cC$ over contexts, since $A$ and
$B$ are types lying over some context, and the formation rule is a map $\Pi : P \to \cU$ such that $u\Pi = p$. To see
what the premise $P$ should be, note that a type $B$ is dependent over $A$ if $uB = \set{A}$, so we should take $P$
to be the pullback
\[\begin{codi}[tetragonal=base 5em height 4.5em]
  \obj {|(tl)| \cU \pb[\set{-}]{u} \cU & |(tr)| \cU \\ |(bl)| \cU & |(br)| \cC \\};
  \mor[swap] tl A_F:-> bl {{\set{-}}}:-> br;
  \mor tl B_F:-> tr u:-> br;
\end{codi}\]
with $p = uA_F$ picking the underlying context $\Gamma$ of $A$. In this way, a generalized element $H : X \to P$ is the same
as a pair $H_1, H_2 : X \to \cU$ with $uH_2 = \set{H_1}$, as desired, vice versa. In particular, the formation rule then
says that given any pair of types $\alpha, \beta : X \to \cU$ with $u\beta = \set{\alpha}$, we have $\Pi(\alpha,\beta) :
X \to \cU$, as expected.

Similarly, the introduction rule tells us that, given a type $A$ and a term $A \vdash b : B$, we have a term
$\lambda(A,b) : \Pi(A,B)$. This rule is also captured by a morphism, this time of the form $\lambda : P' \to \dot\cU$ for
an appropriate $P$. This time $P$ consists of a pair of a type $A$ and a term $b$, such that $\dot u b = \set{A}$, and
so is captured by the pullback
\[\begin{codi}[tetragonal=base 5em height 4.5em]
  \obj {|(tl)| \cU \pb[\set{-}]{\dot u} \dot \cU & |(tr)| \dot \cU \\ |(bl)| \cU & |(br)| \cC \\};
  \mor[swap] tl A':-> bl {{\set{-}}}:-> br;
  \mor tl b:-> tr \dot u:-> br;
\end{codi}\]
To capture the typing of the $\lambda$-term, we need to ask that the following square commutes:
\[\begin{codi}[tetragonal=base 6em height 4.5em]
  \obj {|(tl)| \cU \pb[\set{-}]{\dot u} \dot\cU & |(tr)| \dot \cU \\ |(bl)| \cU \pb[\set{-}]{u} \cU & |(br)| \cU \\};
  \mor tl \lambda:-> tr \typ:-> br;
  \mor[swap] tl {{(A', \typ b)}}:-> bl \prod:-> br;
\end{codi}\]
where $(A',\typ B)$ is the unique map into the pullback $\cU \pb[\set{-}]{u} \cU$ given by the equation $\set{A'} = \dot u b =
u\typ b$ holding.

Finally, using the fact that terms of a dependent product $\Gamma \vdash t : \Pi(A,B)$ ought to be in 1-1 correspondence with
terms $\Gamma . A \vdash t' : B$, we see that asking that the above square be a pullback gives us the expected inverse of the
lambda rule; A term $t : X \to \dot\cU$ with $\typ t = \Pi(\alpha,\beta)$ for some adequate $\alpha,\beta : X \to \cU$, we have
a unique map $\bar t : X \to \cU \pb[\set{-}]{\dot u} \dot \cU$ with $\lambda \bar t = t$, and vice versa. To capture the
elimantion rule
\[\inferrule{\Gamma \vdash t : \Pi(A,B) \\ \Gamma \vdash u : A}{\Gamma\vdash t\cdot u : B[\id_\Gamma,u]},\]
we see that the definition $t\cdot u \coloneq \bar t [\id_\Gamma,u]$ has the expected type, and by this definition for any
term $\bar t : X \to \cU \pb[\set{-}]{\dot u}$, we find that $\lambda(\bar t)\cdot u = \bar t[\id_\Gamma,u]$, so that
we validate the computation rule for dependent product types as well.

Putting all of this together, we have the following definition:
\begin{definition}
  A natural model $(\cC, \cU, \dot\cU)$ has dependent products if there exists a pullback square of the form
  \[\begin{codi}[tetragonal=base 6em height 4.5em]
    \obj {|(tl)| \cU \pb[\set{-}]{\dot u} \dot\cU & |(tr)| \dot \cU \\ |(bl)| \cU \pb[\set{-}]{u} \cU & |(br)| \cU \\};
    \mor tl \lambda:-> tr \typ:-> br;
    \mor[swap] tl {{(A', \typ b)}}:-> bl \prod:-> br;
  \end{codi}\]
\end{definition}

Since all data here is describable by finite 2-limits, we can simply add this to the sketch for natural models to obtain
the theory of natural models with dependent product types.

\subsubsection{Dependent sums}
As with dependent product types, the type former for dependent sum types is modeled by a morphism $\Sigma : \cU \pb[\set{-}]{u}
\cU \to \cU$. However, the introduction rule is more complicated:
\[\inferrule{\Gamma \vdash a : A \\ \Gamma \vdash b : B[\id_\Gamma, t]}{\Gamma \vdash \seq{a,b} : \Sigma(A,B)}\]
In particular, the premise requires a term $u$ depending on the \emph{substitution} of a term $t$. An instance of
the premise is thus a pair of maps $a,b : X \to \dot\cU$ and a map $B : X \to \cU$ such that $\dot u a = \dot u b$,
$u B = \set{\typ a}$, and $\typ b = r(\tau a, B)$, where $(\tau a, B) : X \to \cC \comma u$ is the map induced by
the 2-cell $\tau a : \dot u a \to \set{\typ a} = uB$ and $\tau = \dot u \eta: \dot u \to \dot u \var\typ$ is as defined
above. In essence, the preimse consists of two terms $a$,$b$, and a type $B$, such that $B$ depends on the type of $a$,
and the type of $b$ is $B[a]$. To construct an object corresponding to this, first consider the pullback
\[\begin{codi}[tetragonal=base 6em height 4.5em]
  \obj {|(tl)| \dot\cU \pb[\set{\typ}]{u} \cU & |(tr)| \cU \\ |(bl)| \dot\cU & |(br)| \cC \\};
  \mor tl B:-> tr u:-> br;
  \mor[swap] tl a:-> bl {{\set{\typ}}}:-> br;
\end{codi}\]
This gives an object of terms $a$ with types $B$ depending on $\typ a$. For ease of notation, let $P \coloneq \dot\cU \pb[\set{T}]
{u} \cU$. To construct the desired premise, we now take the following pullback:
\[\begin{codi}[tetragonal=base 6em height 4.5em]
  \obj {|(tl)| P \pb[r(\tau a, B)]{\typ} \dot\cU & |(tr)| \dot\cU \\ |(bl)| P & |(br)| \cC \\};
  \mor tl b:-> tr \dot u:-> br;
  \mor[swap] tl p:-> bl {{r(\tau a, B)}}:-> br;
\end{codi}\]
where $(\tau a, B) : P \to \cC \comma u$ is the map induced by the universal property of $\cC \comma u$ with
$\tau a : \dot u a \to \set{\typ a} = uB$. With this, we say that the dependent pairing operation is a map
$\seq{-,-} : P \pb[r(\tau a,B)]{\typ} \dot\cU \to \dot\cU$, and we can define dependent sums analogously to
dependent products:
\begin{definition}
  A natural model $(\cC, \cU, \dot\cU)$ has depedent sums if there exists a pullback square of the form
  \[\begin{codi}[tetragonal=base 6em height 4.5em]
    \obj {|(tl)| P \pb[r(\tau a, B)]{\typ} \dot\cU & |(tr)| \dot \cU \\ |(bl)| \cU \pb[\set{-}]{u} \cU & |(br)| \cU \\};
    \mor tl {{\seq{-,-}}}:-> tr \typ:-> br;
    \mor[swap] tl {{(\typ a, B)p}}:-> bl \sum:-> br;
  \end{codi}\]
\end{definition}

Again all the data is given in terms of finite 2-limits, and so can be added to the sketch of natural models to obtain
the theory of natural models iwth dependent sum types, and of course it can be combined with the sketch of natural models
with dependent product types just as well.

\subsubsection{Extensional identity types}
Finally we cover extensional identity types, our treatment of which are directly adapted from \cite[§2.3]{awodey2017}.
The formation rule for the identity types is given by 
\[\infer{\Gamma \vdash A\;\cU \\ \Gamma \vdash a : A \\ \Gamma \vdash b : A}{\Gamma \vdash \Id_A(a,b)\,\cU}\]
which we model as a 1-cell $\Id : \dot\cU \pb[\typ]{\typ} \dot\cU \to \cU$ with $u\Id = u\typ p = u\typ q$. Similarly,
the introduction rule is given by
\[\infer{\Gamma \vdash a : A}{\Gamma \vdash \refl(a) : \Id_A(a,a)\,\cU}\]
and so is modeled by a 1-cell $\refl : \dot\cU \to \dot\cU$ with $\typ\refl = \Id\delta$, where
$\delta : \dot\cU \to \dot\cU \pb{\cU} \dot\cU$ is induced map by the pullback square in the following diagram:
\[
  \begin{codi}[tetragonal=base 4.5em height 4.5em]
    \obj{ |(U)| \dot\cU \\
          & |(tl)| \dot\cU \pb[\typ]{\typ} \dot\cU & |(tr)| \dot\cU \\
          & |(bl)| \dot\cU & |(br)| \cU \\ };

    \mor tl p:-> tr \typ:-> br;
    \mor[swap] tl q:-> bl \typ:-> br;

    \mor:[bend left] bl <- U -> tr;
    \mor U \delta:[->,dashed] tl;
  \end{codi}
\]
If the square
\[\begin{codi}[tetragonal=base 6em height 4.5em]
  \obj {|(tl)| \dot\cU & |(tr)| \dot \cU \\ |(bl)| \dot\cU \pb[\typ]{\typ} \dot\cU  & |(br)| \cU \\};
  \mor tl {{\refl}}:-> tr \typ:-> br;
  \mor[swap] tl {{\delta}}:-> bl \Id:-> br;
\end{codi}\]
is a pullback, then for any terms $a,b : X \to \dot\cU$ with $\typ a = \typ b$, if there is term $e : X \to \dot\cU$
with $\typ e = \Id(a,b)$, then $a = b$. In other words, given such a square, the natural model has extensional
identity types.
\begin{definition}
  A natural model $(\cC,\cU,\dot\cU)$ has extensional identity types if there exists a pullback square of the form
  \[\begin{codi}[tetragonal=base 6em height 4.5em]
    \obj {|(tl)| \dot\cU & |(tr)| \dot \cU \\ |(bl)| \dot\cU \pb[\typ]{\typ} \dot\cU  & |(br)| \cU \\};
    \mor tl {{\refl}}:-> tr \typ:-> br;
    \mor[swap] tl {{\delta}}:-> bl \Id:-> br;
  \end{codi}\]
\end{definition}
As with both dependent product and dependent sum types, all constructions used were in terms of finite 2-limits, and
so we can add these to our sketch for natural models, possibly with dependent sum and product types, to obtain a 
theory for natural models with identity types.

\subsection{Equivalence to categories with attributes}\label{sec:nm/equiv attr}
Already when they were introduced in \cite{dybjer1996}, \citeauthor{dybjer1996} showed how categories with families are in
fact equivalent to categories with attributes. When passing from a category with families to a category with attributes,
you simply need to recover the comprehension map, which we've already had to do in describing the common type constructors,
and show that this comprehension map has the properties required of a comprehension category. In the other direction,
you take the object of types to be the object of sections to the comprehension of a type, constructed in an appropriate
sense and shown to be a fibration. In fact, we can in this way show that the theories $\Attr$ and $\NM$ are biequivalent
as $\sF$-categories, and so that their categories of models are always equivalent, whether you consider strict morphisms,
pseudomorphisms, or lax morphisms.

To construct such a biequivalence requires constructing $\sF$-functors $F : \Attr \rightleftarrows \NM : G$ together with
natural $\sF$-equivalences $FG \equiv 1_\NM$ and $GF \equiv 1_\Attr$. To find such functors, it suffices to construct a model
$F$ of $\Attr$ inside the $\sF$-category $\NM$, and a model $G$ of $\NM$ inside $\Attr$, and then to show that the model $FG$
inside $\NM$ (resp. $GF$ inside $\Attr$) are equivalent to the ambient model of the theory. We will therefore split the proof
up into four parts: the constructions of the models are the first two parts, and then the equivalences the second.

The first part, the construction of a model of $\Attr$ inside $\NM$, we've already essentially done, since the obvious
comprehension map is the one induced by the map $\pi : \set{-} \to u$ we defined earlier. Thus it only remains to show
that this is indeed a comprehension category.
\begin{lemma}\label{lem:attr in nm}
  Working inside $\NM$, Let $\chi : \cU \to \cC^\to$ be the map induced by the 2-cell $\pi : \set{-} \to u$ described above.
  Then 
  \[\begin{codi}[hexagonal=horizontal side 6em angle 45] 
    \obj{\cU &   \cC^\to \\ & \cC \\};
    \mor  cU \chi:-> (cC ^to) \cod:-> cC;
    \mor[swap] cU u:-> cC;
    \end{codi}\]
  is a category with attributes, in the sense that it is the image of a finite $\sF$-limit preserving functor $F : \Attr \to \NM$.
\end{lemma}
\begin{proof}
  The only thing left to show is that $\chi$ preserves cartesian 2-cells, and since $u$ is a discrete fibration, every 2-cell in
  $\cU$ is cartesian. Therefore we must show that $\chi$ itself is cartesian, which is equivalent to showing that for every $f :
  A \to B : X \to \cU$, the following square is a pulback:
  \[\begin{codi}
    \obj {|(tl)| \set{A} & |(tr)| \set{B} \\ |(bl)| uA & |(br)| uB \\};
    \mor tl \set{f}:-> tr \pi B:-> br;
    \mor[swap] tl \pi A:-> bl uf:-> br;
  \end{codi}\]
  To see that this is indeed the case, consider an arbitrary span $uA \xleftarrow{\sigma} \Gamma \xrightarrow
  {\tau} \set{B} : X \to \cC$ with $u f \circ \sigma = \pi B \circ \tau$. We must show that there exists a unique map $\phi :
  \Gamma \to \set{A}$ with $\pi A \circ \phi = \sigma$ and $\set{f} \circ \phi = \tau$, for which it suffices to find a map
  $\bar\phi : \bar\Gamma \to \var A$ with $\dot u (\var f \circ \bar \phi) = \tau$ and $u (\epsilon A \circ \typ \bar \phi)
  = \sigma$, for some $\bar \Gamma$ with $\dot u \bar\Gamma = \Gamma$. To this end, let $\bar \tau : \bar \Gamma \to VB$
  denote the lift of $\tau : \Gamma \to \dot u V B$ in $\dot u$. Similarly let $\bar \sigma$ denote the $u$-lift of $\sigma
  : \Gamma \to uA$. Now we have
  \[u(\epsilon B \circ \typ \bar \tau) = \pi B \circ \tau = u(f \circ \bar\sigma),\]
  so since $u$-lifts are unique by $u$ being a discrete fibration, we have that $\epsilon B \circ \typ \bar \tau = f \circ \bar
  \sigma$. In particular, we see that $\bar \sigma$ is a map $\typ\bar\Gamma \to A$, so if we take
  \[\bar\phi \coloneq \var\bar\sigma \circ \eta \bar\Gamma : \bar\Gamma \to \var\typ\bar\Gamma \to \var A,\]
  we can compute
  \begin{align*}
    \dot u (\var f \circ \bar\phi) & = \dot u(\var(f \circ \bar\sigma) \circ \eta\bar\Gamma) \\
                                   & = \dot u(\var(\epsilon B \circ \typ\bar\tau) \circ \eta\bar\Gamma) \\
                                   & = \dot u(\var\epsilon B \circ \var\typ\bar\tau \circ \eta\bar\Gamma)) \\
                                   & = \dot u (\var\epsilon B \circ \eta\var B \circ \bar \tau) \\
                                   6 = \dot u\bar \tau = \tau
  \end{align*}
  and
  \[u(\epsilon A \circ \typ\bar\phi) = u(\epsilon A \circ \typ\var\bar\sigma \circ \typ\eta\bar\Gamma)
                                     = u(\bar\sigma \circ \epsilon \typ\bar\Gamma \circ \typ\eta\bar\Gamma)
                                     = u\bar\sigma = \sigma,\]
  as desired. Therefore we have that $\phi \coloneq \dot u \bar \phi$ yields $\set{f} \circ \phi = \dot u(\var f \circ \bar \phi)
  = \tau$ and $\pi A \circ \phi = u(\epsilon A \circ \typ\bar\phi) = \sigma$, showing that $\phi$ has the desired properties.
  Uniqueness of $\phi$ follows from the uniqueness of adjuncts and $\dot u,u$-lifts.

  Since $\chi$ is cartesian, it follows that 
  \[\begin{codi}[hexagonal=horizontal side 6em angle 45] 
    \obj{\cU &   \cC^\to \\ & \cC \\};
    \mor  cU \chi:-> (cC ^to) \cod:-> cC;
    \mor[swap] cU u:-> cC;
    \end{codi}\]
  is a model of $\Attr$, as we wanted to show.
\end{proof}

\begin{lemma}\label{lem:nm in attr}
  Working in $\Attr$, the map $\dot u$ as defined in \cref{subsec:compcat terms} is a discerete fibration, the map $\typ$
  has a right adjoint $\var$, and the diagram
  \[\begin{codi}[hexagonal=horizontal side 6em angle 45] 
    \obj{\dot\cU &   \cU \\ & \cC \\};
    \mor  (dot cU) ["\typ",swap]:-> cU u:-> cC;
    \mor[swap] (dot cU) \dot u:-> cC;
    \mor[swap] cU \var:[bend right,->] (dot cU);
    \mor typ [shorten=0.3em, |-] var;
  \end{codi}\]
  is a natural model in $\Attr$, in the sense that it is the image of a finite $\sF$-limit preserving functor $G : \NM \to \Attr$.
\end{lemma}
\begin{proof}
  We first show that $\dot u$ is a discerete fibration, for which it suffices to show that all $\dot u$-vertical maps (those maps
  $\alpha : a \to b : X \to \dot\cU$ for which $\dot u \alpha$ is identity) are identity, since $\dot u$ is a fibration by
  \cref{lem:comp term fibration}. To see that this holds, suppose $\alpha : a \to b : X \to \dot\cU$ is a vertical map. Then
  $u\typ\alpha = \dot u \alpha$ is the identity, so since $u$ is a discrete fibration, $\typ\alpha$ is identity as well. Thus we
  have $\typ a = \typ b$ and $\typ\alpha = 1_{\typ a}$. Now we see by the construction of $\typ$ that $\alpha$ must be identity,
  as desired. \pagebreak

  For the right adjoint $\var$ of $\typ$, we follow the intuition of categories with families. Let $\epsilon : \delta \to
  1_\cU : \cU \to \cU$ denote the cartesian lift of $\pi : \set{-} \to u$. Then we have the following diagram:
  \[\begin{codi}
    \obj {|(tl)| \set{-} \\ & |(tm)| \set{\delta} & |(tr)| \set{-} \\ & |(bm)| \set{-} & |(br)| u \\};
    \mor tm {{\set{\epsilon}}}:-> tr \pi:-> br;
    \mor[swap] tm \pi\delta:-> bm \pi:-> br;
    \mor:[bend left] bm {{1_{\set{-}}}}:<- tl {{1_{\set{-}}}}:-> tr;
    \mor:[dashed] tl \tau\var:-> tm;
  \end{codi}\]
  Since the bottom-right square is a pullback, there exists a section $\tau\var$ of $\pi\delta$, which then induces a term
  $\var : \cU \to \dot\cU$. Furthermore, since $\typ\var = \delta$, we see that the 2-cell $\epsilon : \delta \to 1_\cU$ is
  also a 2-cell $\typ\var \to 1_\cU$, which will be the counit of the adjunction. Furthermore, we have that $\dot u\var =
  u\typ\var = u\delta = \set{-}$. Since for natural models the unit $\eta : 1_{\dot\cU} \to \var\typ$ plays the role of
  the substitution induced by extending by the variable, we want $\eta$ to be the lift of $\tau : u\typ \to \set{\typ} =
  \dot u\var\typ$. Let $\bar\tau : a \to \var\typ$ denote this lift. To allow taking $\eta \coloneq \bar\tau$ to be the
  unit, we must that $a = 1_{\dot\cU}$. Note that $\typ\bar\tau$ is the $u$-lift of $\tau : u\typ \to u\typ\var\typ =
  u\delta\typ$, so since $u(\epsilon\typ \circ \typ\bar\tau) = \pi\typ \circ \tau = 1_{\dot u}$ and $u$-lifts are unique,
  we have that $\epsilon\typ \circ \typ\bar\tau = 1_{\typ}$. In particular, it follows that $a = 1_{\dot\cU}$, and so
  we can take $\eta = \bar\tau$. This also immediately gives us the one of the triangle equations: $\epsilon\typ \circ
  \typ\eta = 1_{\typ}$. For the other equation, $\var\epsilon \circ \eta\var = 1_{\var}$, it suffices to show that
  $\dot u\var\epsilon \circ \dot u\eta\var = 1_{\dot u\var}$, since $\dot u$ is a discerete fibration. But we have
  \[\dot u\var\epsilon \circ \dot u\eta\var = \set{\epsilon} \circ \tau\var = 1_{\set{-}} = 1_{\dot u\var}\]
  by the construction of $\var$ and $\eta$. Therefore $\epsilon$ and $\eta$ are the counit and unit respectively of an
  adjunction $\typ \dashv \var$, as desired.

  We conclude that the diagram
  \[\begin{codi}[hexagonal=horizontal side 6em angle 45] 
    \obj{\dot\cU &   \cU \\ & \cC \\};
    \mor  (dot cU) ["\typ",swap]:-> cU u:-> cC;
    \mor[swap] (dot cU) \dot u:-> cC;
    \mor[swap] cU \var:[bend right,->] (dot cU);
    \mor typ [shorten=0.3em, |-] var;
  \end{codi}\]
  forms a natural model in $\Attr$, with $G : \NM \to \Attr$ the corresponding $\sF$-functor.
  % In conclusion, $\dot u$ is a discrete fibration and $\typ$ has a right adjoint $\var$, so $\Attr$ supports the structure of a
  % natural model with $(u,\dot u, \typ, \var)$.
\end{proof}
\begin{remark}
  Most of this proof does not rely on discreteness of $u$, except in showing that $\dot u$ is also a discrete fibration.
  Therefore it could perhaps be adapted into an alternative proof of equivalence between the generalized categories of
  families of \cite{coraglia2024a} and comprehension categories to the one presented in \cite{coraglia2024b}, though this
  would need more thorough checking.
\end{remark}
\begin{remark}
  Though not used explicitly here, note that $\epsilon$ as defined in this proof is the counit of a comonad. In particular,
  $\delta$ is a so-called weakening-contraction comonad, which are introduced and shown equivalent to comprehension categories
  in \cite{jacobs1998}.
\end{remark}

We are now equipped to show that $FG \cong 1_\NM$ and $GF \cong 1_\Attr$.
\begin{theorem}
  $F$ and $G$ as defined above are weak inverses.
\end{theorem}
\begin{proof}
  We first show that $GF \cong 1_\Attr$. In fact, $GF = 1_\Attr$, since $GF$ fixes the type fibration $u : \cU \to \cC$ and,
  using $\chi'$ to denote the comprehension map of $GF$ and $\pi' : \set{-}' \to u$ the corresponding 2-cell, we have that
  $\pi' = u\epsilon = \pi$ by the definition of $\epsilon$, so $\chi' = \chi$. 
  
  For $FG \cong 1_\NM$, we will show $\typ : \dot\cU \to \cU$ is a limiting cone to the diagram defining $\dot\cU'$, i.e. that
  $\dot\cU$ is an object of sections to typing projections $\pi A : \set{A} \to uA$. Then we have from the uniqueness of limits
  that $\dot\cU \cong \dot\cU'$ in a way coherent with all the structure of a natural model, which shows that $FG \cong 1_\NM$.
  We first show that $\typ$ indeed provides such a cone: that is, we must show that there is a map $\tau : u\typ \to \set{\typ}$
  which is a section to $\pi\typ$. But we already constructed this map $\tau$ earlier. To show that this cone is limiting,
  we appeal to \cref{prop:chr of F-limits}. Thus we must show that for every type $A : X \to \cU$ and section $s : uA \to
  \set{A}$ of $\pi A : \set{A} \to uA$, there exists a unique map $\phi_s : X \to \dot\cU$ with $\typ a = A$ and $\tau a = s$,
  and for every additional type $B : X \to \cU$, section $t : uB \to \set{B}$ of $\pi B: \set{B} \to uB$, and 2-cell $h : A \to
  B$ with $\set{h} \circ s = t \circ uh$, there exists a unique 2-cell $\phi_h : \phi_s \to \phi_b$ with $h = \typ \phi_h$.

  We start by constructing $\phi_s : X \to \dot\cU$. Let $\bar s : a \to \var A$ denote the $\dot u$-lift of $s$. Since
  \[ u(\epsilon A \circ \typ\bar s) = \pi A \circ u\typ \bar s = \pi A \circ s = 1_{uA}\]
  has the unique $u$-lift $1_A$, we must have that the domain of $\typ\bar s$ is $A$, i.e. that $\typ a = A$. It remains to
  show that $\tau a = s$, for which we first show that $\eta a = \bar s$:
  \begin{align*}
    \bar s & = 1_{\var\typ a} \circ s \\
           & = (\var\epsilon \circ \eta\var)\typ a \circ \bar s \\
           & = \var\epsilon\typ a \circ \eta\var\typ a \circ \bar s \\
           & = \var\epsilon\typ a \circ \var\typ\bar s \circ \eta a \\
           & = \var(\epsilon \typ a \circ \typ \bar s) \circ \eta a \\
           & = 1_{\var\typ a} \circ \eta a \\
           & = \eta a
  \end{align*}
  With this, we see that $\tau a = \dot u \eta a = \dot u \bar s = s$, so taking $\phi_s = a$ gives the desired map with
  $\tau \phi_s = a$. Furthermore, it is unique, since any other map $a' : X \to \dot\cU$ satisfying $\tau a' = s$ has that
  $\dot u \eta a' = \dot u \eta \phi_s$ and $\typ a' = A$, so by the uniqueness of $u$-lifts, $\eta a' = \eta \phi_s$.
  
  Now, suppose we are given sections $s : uA \to \set{A}$ and $t : uB \to \set{B}$ and a map $h : A \to B$ with $\set{h} \circ
  s = t \circ uh$. We must construct a unique map $\phi_h: \phi_s \to \phi_t$. Using the same notation as above, let $\bar s :
  \phi_s \to A$ and $\bar t : \phi_t \to B$ denote the $\dot u$-lifts of $s$ and $t$ respectively. Now let $\bar h :
  a \to \phi_t$ denote the $\dot u$-lift of $uh : \dot u \phi_s \to \dot u \phi_t$, so that $\dot u\bar h =
  uh$. Then we have that
  \[
    u\typ\bar h = uh,\circ \eta\phi_s) = \set{h} \circ s = t \circ uh = u(\typ\eta\phi_t \circ h),
  \]
  and since $u$ is a discrete fibration, this means that $\typ\bar h = h$. Moreover, we have that
  \[\dot u (\eta \phi_t \circ \bar h) = t \circ uh = \set{h} \circ s = \dot u(\var h \circ \eta\phi_s),\]
  so since $\dot u$ is a discrete fibration and $\var h : \var A \to \var\typ \phi_t$ has the same codomain as $\eta \phi_t$,
  we see that $\eta \phi_t \circ \bar h = \var h \circ \eta\phi_s$. In particular, we have that $\bar h : \phi_s \to \phi_t$
  and $\typ\bar h = h$. Thus we can take $\phi_h = \bar h$. To see that this is unique, suppose we were given another 2-cell
  $h' : \phi_s \to \phi_t$ with $\typ h' = h$. Then
  \[\dot u h' = u\typ h' = u h = u \typ \phi_h = \dot u \phi_h : \dot u \phi_s \to \dot u \phi_t,\]
  so from the uniqueness of $\dot u$-lifts, we see that $h' = \phi_h$.
  
  Thus we have shown that $(\typ,\tau)$ forms a limiting cone for the object of sections of type projections $\pi : \set{-} \to
  u$, which gives us the coherent isomorphism $\dot\cU \cong \dot\cU'$ we required. From this, we conclude that $FG \cong 1_\NM$.

  Since we have shown that $FG \cong 1_\NM$ and $GF \cong 1_\Attr$, we have exactly that $F$ and $G$ constitute an equivalence
  of $\NM$ and $\Attr$.
\end{proof}

\begin{corollary}
  The 2-categories of natural models and categories with attributes are equivalent.
\end{corollary}

% \section{Inductive Types}
% In this section we show how to describe inductive and coinductive types using finite 2-limits. While the details
% differ quite a
% bit, the presentation of inductive types here was inspired by the presentation in \cite{basold2015}, where
% inductive types are defined as initial dialgebras for a pair of functors, the signature for the type, of a
% specific shape. To make sure that we have a well-behaved induction principle, we will require the existence
% of dependent sum types. This allows the derivation of the induction principle from the non-dependent recursion
% principle given by initiality.
%
% We will base the presentation on comprehension categories, since the definition of terms as substitutions
% turns out to be more convenient in describing the constructors, eliminators. and computation rules of
% inducitve types. However, the specfic construction is not inherently limited to comprehension categories,
% and should be possible to generalize to other models of type theory, such as natural models.
%
% \begin{definition}
%   A \emph{data type signature} with parameters $p : P \to \cC$ is a tuple $(F,\phi)$ where $F$ is a 1-cell
%   $P \pb[p]{u} \cU \loose \cC^n$ and $\phi$ is a 2-cell $F \to \delta^n u \pi_2$, where $\delta^n : \cC
%   \to \cC^n$ is the diagonal map. The intuition is that $F$ determines the arguments for each constructor
%   by extending the given context, and $\phi$ connects the extend context back to the original one to
%   facilitate the elimination rule.
%
%   An algebra for a signature $(F,\phi)$ is a map $(\alpha, A) : X \to P \pb[p]{u} \cU$ equipped with a 2-cell
%   $a : F(\alpha,A) \to \delta^n\set{A}$ such that $\phi(\rho,A) = \delta^n\pi A \circ a$, and an algebra homomorphism
%   $(\alpha,A,a) \to (\beta, B,b)$ is a 2-cell $(\rho,f) : (\alpha,A) \to (\beta,B)$ such that $\delta^n\set{f}
%   \circ a = b \circ F(\rho,f)$.
% \end{definition}
% \begin{remark}
%   An algebra for a signature $(F,\phi)$ is an $(F,\delta^n\set{\pi_2})$-\emph{dialgebra} subject to an
%   additional condition ensuring compatibility with the 2-cell $\phi$. Algebra homomorphisms are exactly
%   maps of dialgebras. 
% \end{remark}
%
% \begin{remark}
%   For a fixed signature $(F,\phi)$, we can using only finite $\sF$-weighted limits construct an object
%   of algebras $(F,\phi)\Alg$, such that any $(F,\phi)$-algebra is the same as a map into $(F,\phi)\Alg$,
%   and vice versa. This object comes equipped with a universal $(F,\phi)$-algebra, with underlying map
%   $R_{(F,\phi)} : (F,\phi) \Alg \to P \pb[u]{p} \cU$ and 2-cell $\rho : F R_{(F,\phi)} \to \delta^n\set
%   {R_{(F,\phi)}}$.
% \end{remark}
%
% We want an inductive type over a signature $(F,\phi)$ to be an inital algebra. Since this algebra is supposed
% to exist over aribtary parameters, this is best modeled by a map from the object of parameters to the object
% of algebras.
%
% \begin{definition}
%   An inductive type for a signature $(F,\phi)$ is a map $\mu(F,\phi) : P \to (F,\phi)\Alg$ such that
%   $\pi_1 R_{(F,\phi)} \mu(F,\phi) = 1_P$ and for every $A : X \to (F,\phi)\Alg$, there exists a unique
%   map $h : \mu(F,\phi)\pi_1R_{(F,\phi)} A \to A$ such that $\pi_1R_{(F,\phi)} h = 1_{\pi_1R_{(F,\phi)}A}$
% \end{definition}
% \begin{lemma}
%   Let a signature $(F,\phi)$ and $\mu(F,\phi) : P \to (F,\phi)\Alg$ be given. Then the following are equivalent:
%   \begin{enumerate}
%     \item $\mu(F,\phi)$ is an inductive data type.
%     \item $\mu(F,\phi)$ is left adjoint to $\pi_1 R_{(F,\phi)} : (F,\phi)\Alg \to P$ with identity unit.
%   \end{enumerate}
% \end{lemma}
% \begin{proof}
%   For $1 \implies 2$, note that by taking $A = 1_{(F,\phi)\Alg}$, we see that there exists a unique map $
%   \epsilon : \mu(F,\phi)\pi_1R_{(F,\phi)} \to 1_{(F,\phi)\Alg}$ such that $\pi_1 R_{(F,\phi)}\epsilon =
%   1_{\pi_1 R_{(F,\phi)}}$. Thus it only remains to show that $\epsilon\mu(F,\phi) = 1_{\mu(F,\phi)}$. But
%   there exists by uniqueness only one map $h : \mu(F,\phi)\pi_1R_{(F,\phi)}\mu(F,\phi) = \mu(F,\phi) \to
%   \mu(F,\phi)$ such that $\pi_1 R_{(F,\phi)} h = 1_{\pi_1 R_{(F,\phi)}\mu(F,\phi)} = 1_{1_{(F,\phi)\Alg}}.$
%   Since the identity map $1_{\mu(F,\phi)}$ and $\epsilon\mu(F,\phi)$ both satisfy this property, we conclude
%   that they are the same, thereby establishing that $\epsilon$ is the counit of an adjunction $\mu(F,\phi)
%   \dashv \pi_1R_{(F,\phi)}$ with identity unit.
%
%   For $2 \implies 1$, let $A : X \to (F,\phi)\Alg$ be given. Clearly $\epsilon A$ is a map $\mu(F,\phi)\pi_1
%   R_{(F,\phi)}A \to A$ satisfying the desired property, so it remains to show that this map is unique. To that
%   end, let $h : \mu(F,\phi)\pi_1R_{(F,\phi)}A \to A$ be given such that $\pi_1R_{(F,\phi)}h = 1_{\pi_1R_{(F,
%   \phi)}A}$. Note that
%   \[\epsilon A \circ \mu(F,\phi) \pi_1 R_{(F,\phi)} h = h \circ \epsilon \mu(F,\phi)\pi_1 R_{(F,\phi)}A = h,\]
%   but $\pi_1 R_{(F,\phi)} h = 1_{\pi_1 R_{(F,\phi)}}$, so we see that
%   \[\epsilon A = \epsilon A \circ \mu(F,\phi)\pi_1 R_{(F,\phi)} h = h,\]
%   as desired.
% \end{proof}
%
% \begin{remark}
%   The data types described here satisfy the $\eta$-rule, since $\epsilon\mu(F,\phi) = 1_{\mu(F,\phi)}$. We could
%   drop this property by asking for a weakly initial object, rather than an initial object. This would be removing
%   the uniqueness requirement from the definition of an inductive data type, or equivalently asking for a map
%   $\epsilon : \mu(F,\phi)\pi_1R_{(F,\phi)} \to 1_{(F,\phi)\Alg)}$ such that $\pi_1 R_{(F,\phi)}\epsilon =
%   1_{\pi_1R_{(F,\phi)}}$.
% \end{remark}
%
% \begin{example}[Empty type]
%   The empty type has no parameters, meaning we take $p = 1_\cC : \cC \to \cC$, and no constructors. Thus we
%   take $F$ to be the unique map into the 2-terminal object, and $\phi$ is trivially determined by this.
%   We use $\bot = \mu(F,\phi)$ to denote this inductive type. 
%
%   An algebra for $(F,\phi)$ is simply a type $A : X \loose \cU$ (up to isomorphism), and the universal property
%   then always gives us a map $\set{\bot uA} \to \set{A}$, which we interpret to say that for any context
%   containing $\bot$, we have an element of any type.
% \end{example}
%
% \begin{example}[Unit type]
%   The unit type also has no parameters, so $p = 1_\cC : \cC \to \cC$. It has exactly one constructor, and
%   this constructor takes no arguments. Thus we take $F = p\pi_1$ and $\phi = 1_{p\pi_1}$. We denote the
%   inductive type for this signature by $\top$.
%
%   An algebra for $(F,\phi)$ is a type $A : X \to \cU$ with a substitution $a : uA \to \set{A}$ such that
%   $\pi A \circ a = 1_{uA}$, i.e.\ a term of type $A$. We use $\top$ also to denote the underlying type
%   $\pi_2 R_{(F,\phi)}\top$ of the initial algebra. The constructor $* : u\top \to \set{\top}$ is the map
%   given by the algebra, formally $\pi_2 \rho_{(F,\phi)}$. The universal property then gives us a map
%   $\set{\pi_2R_{(F,\phi)}\epsilon A} : \set{\top uA} \to \set{A}$ such that $\set{\pi_2 R_{(F,\phi)}\epsilon A}
%   \circ * = a \circ p\pi_1R_{(F,\phi)}\epsilon A = a$, which is exactly the
%   $\beta$-rule.
% \end{example}
%
% \begin{example}[Binary sum types]
%   Two constructors, $P = \cU \pb[u]{u} \cU$, $p = u\pi_1 = u\pi_2$, $F_1 = \set{\pi_1\pi_1}$, $F_2 =
%   \set{\pi_2\pi_1}$. 
% \end{example}
%
% \begin{example}[Natural numbers]
%   No parameters, so $p = 1_\cC : \cC \to \cC$. Two constructors, the first with no arguments, so take
%   $F_1 = u\pi_2 = p\pi_1$, and the second with one, so take $F_2 = \set{\pi_2}$. An algebra is a type
%   $A : X \to \cU$ with two substitutions, $a_z : uA \to \set{A}$ and $a_s : \set{A} \to \set{A}$.
%   The initial algebra is a type $\bN : \cC \to \cU$ with two constructors $z : 1_\cC \to \set{\bN}$ and
%   $s : \set{\bN} \to \set{\bN}$, and the eliminator into $A$ gives $\epsilon A : \bN \to A$ such that
%   $\set{\pi_2 \epsilon A} \circ s = a_s \circ \set{\pi_2 \epsilon A}$ and $\set{\pi_2 \epsilon A} \circ z
%   = a_z$.
% \end{example}
%
% \begin{example}[W-types]
%   Suppose the theory has function types, whose type former we denote by $Fun : \cU \pb[u]{u} \cU \to \cU$.
%   $W$-types are parameterized by a type $A$ and a family $B$ indexed over the type. Thus we take $P = (\cU
%   \pb[\set{-}] {u} \cU)$ and $p = u\pi_1$. It has a single constructor, so $F : P \pb[p]{u} \cU \to \cC$,
%   and this constructor takes an element of the first type $A$ and, letting $X = \pi_2$, a map $Fun(B, X[
%   \pi A])$. Thus we take $F = \set{Fun(\pi_2\pi_1, r(\pi\pi_1\pi_1, \pi_2))}$, where $r : \cC \comma u \to
%   \cU$ is the lifting map for the fibration $u$ and $(\pi\pi_1\pi_1, \pi_2) : P \pb[p]{u} \cU \to \cC
%   \comma u$ is the map induced by the 2-cell $\pi\pi_1\pi_2 : \set{\pi_1\pi_1} \to u\pi_1\pi_1 = u\pi_2$.
% \end{example}
%
% \todo{Find better notation than the pointfree style used here. Also, $\pi_i$ is maybe a bad choice for the
% projections.}
%
% The one thing missing now is the general induction principle, which is where we require dependent sum types.
% Consider the induction rule for natural numbers:
% \[\infer{\Gamma.\bN \vdash B\;\cU \\ b_z : \Gamma \to \Gamma.B[z] \\ \pi B[z] \circ b_z = 1_\Gamma \\
%   b_s : \Gamma.\bN.B  \to \Gamma.\bN.B \\ \pi B \circ b_s = s \circ \pi B}
%   {\Gamma.\bN\vdash \elim(B,b_z,b_s) : B}\]
% Using dependent sum types, we can change the premise for the successor to be a map $b_s' : \Gamma.\sum_\bN B
% \to \Gamma.\sum_\bN B$ such that $\pi(\sum_\bN B) \circ b_s' = \pi(\sum_\bN B)$ and $\mathtt{fst} \circ b_s'
% = s \circ \mathtt{fst}$. Similarly, we can change the premise for the zero case to be $b_z' : \Gamma \to
% \Gamma.\sum_\bN B$ such that $\pi(\sum_\bN B) \circ b_z' = 1_\Gamma$ and $\mathtt{fst} \circ b_z' = z$.
% Effectively then, this depedent algebra $B$ over $\bN$ is equivalent to an algebra $b'$ on $\sum_\bN B$,
% so a map $b' : F(\sum_\bN B) \to \set{\sum_\bN B}^n$, such that $\fst^n \circ b' = a' \circ F\fst$, i.e.
% such that $\fst$ is an algebra homomorphism.



% \todo{Talk about induction principle and the use of $\Sigma$-types.}
\section{Multimode and Modal type theories}
Multimode dependent type theory (MTT) was introduced in \cite{gratzer2021} to provide a general family of type theories
with different modalities. MTT is parameterized by a 2-category $\cM$, a so-called mode theory, the objects of which
are called \emph{modes}, the 1-cells are called \emph{modalities}, and the 2-cells are called \emph{operations}.
The goal of this section is to show how the semantics presented in \cite{gratzer2021} can be seen as the models of
a finite $\sF$-limit sketch. A interesting feature of the semantics of MTT which diffrentiates it from most other
type theories, is that it more or less consists of a family of type theories, one at each mode, together with rules
connecting the different type theories with the modalities and operations. as dictated by the mode theory. This is
reflected in the semantics presented in \cite{gratzer2021} by having a family of natural models indexed by the modes,
together with functors and natural transformations between the models corresponding to the modalities and operations.
Because the semantics requires multiple categories of contexts, it is difficult to fit neatly into the functorial
semantics of \cite{uemura2023} or the framework of \cite{coraglia2024a}, which single out the context judgement, and
so the category of contexts, as a primitive notion.

To show that the semantics of \cite[{§5}]{gratzer2021} can be recovered as an finite $\sF$-limit theory, we will
presen the important features of \emph{op.\ cit.\ } and show that these can be constructed by finite 2-limits. We
begin with the defintions of a \emph{modal context structure} and \emph{modal natural models}.
\begin{definition}
  Let $\cM$ be a mode theory. A \emph{modal context structure} is a 2-functor $\cC[-] : \cM^{coop} \to \Cat$. 

  A \emph{modal natural model} on a modal context structure $\cC[-]$ is an family of maps of discrete fibrations
  $\typ_m : \dot\cU[m] \to \cU[m]$ over $\cC[m]$, indexed over the modes $m \in \cM$, such that for each modality
  $\mu : m \to n$, the map $\cC[\mu]^*\typ_m : \cC[\mu]^*\dot\cU[m] \to \cC[\mu]^*\cU[m]$ is a representable map of
  discrete fibrations.
\end{definition}
\begin{remark}
  In \cite{gratzer2021}, they require the category of contexts to have a terminal object, corresponding to the empty
  context. Since we have elided such a requirement from the start, we will continue to do so now. Note, however, that
  it is trivial to add the requirement by simply adding a terminal object $1$ in the sketch and right adjoints to the
  unique maps $\cC[m] \to 1$.
\end{remark}
We will describe a sketch for modal natural models directly, since the $\sF$-sketch for modal context structures is
just $((\cM^{coop})^-, \emptyset)$, the sketch with underlying $\sF$-category $\cM^{coop}$ with only identities tight
and with no cones. 
\begin{definition}
  The sketch for modal natural models consists of:
  \begin{itemize}
    \item For each mode $m$, a diagram
      \[\begin{codi}[hexagonal=horizontal side 6em angle 45] 
        \obj{|(tm _m)| \dot\cU_m &   |(ty _m)| \cU_m \\ & |(ctx _m)| \cC_m \\};
        \mor  (tm _m) ["\typ_m",swap]:≈> (ty _m) u_m:-> (ctx _m);
        \mor[swap] (tm _m) \dot u_m:-> (ctx _m);
        % \mor[swap] (ty _m) \var_m:[bend right,≈>] (tm _m);
        % \mor (typ _m) [shorten=0.3em, |-] (var _m);
      \end{codi}\]
      with cones and additional maps making $u_m, \dot u_m$ discrete fibrations.
    \item For each modality $\mu : m \to n$, squares
      \[\begin{codi}[tetragonal=base 4.5em height 4.5em]
        \obj { \lock_\mu^* \cU_m & \cU_m \\ \cC_n & \cC_m \\ };
        \mor (lock _mu ^* cU _m) -> (cU _m) u_m:-> (cC _m);
        \mor[swap] (lock _mu ^* cU _m) \lock_\mu^*u_m:-> (cC _n) \lock_\mu:-> (cC _m);
      \end{codi} \qquad
      \begin{codi}[tetragonal=base 4.5em height 4.5em]
        \obj { \lock_\mu^* \dot\cU_m & \dot\cU_m \\ \cC_n & \cC_m \\ };
        \mor (lock _mu ^* dot cU _m) -> (dot cU _m) \dot u_m:-> (cC _m);
        \mor[swap] (lock _mu ^* dot cU _m) \lock_\mu^*\dot u_m:-> (cC _n) \lock_\mu:-> (cC _m);
      \end{codi}\]
      and an adjunction $\lock_\mu^* \typ_m \dashv \var_\mu : \lock_\mu^* \cU_m \to \lock_\mu^*\dot\cU_m$,
      as well as cones and diagrams making the two squares pullbacks and $\lock_\mu^*\typ_m : \lock_\mu^*\dot\cU_m
      \to \lock_\mu^*\cU_m$ the map induced by $\typ_m : \dot\cU_m \to \cU_m$.
    \item For each operation $\alpha : \mu \to \nu$, a 2-cell $\key_\alpha : \lock_\nu \to \lock_\mu$.
  \end{itemize}
  Additionally we add equalites so that the assignments $m \mapsto \cC_m$, $(\mu : m \to n) \mapsto \lock_\mu$, and
  $(\alpha : \mu \to \nu) \mapsto \key_\alpha$ together constitute a strict 2-functor from $\cM^{coop}$.

  Following the same structure as in \cref{sec:nm}, we see that this produces a finite limit $\sF$-sketch, and so
  also an induced finite limit $\sF$-theory, which we will denote by $\MTT$.
\end{definition}
Using the same techinques for adding type formers as in \cref{sec:nm}, we can now add type formers to the different mode theories.

\begin{remark}
  With this definition, we see that, for every modality $\mu : m \to n \in \cM$, we have a model of $\NM$ inside $\MTT$ with
  $\cU \coloneq \lock_\mu^*\cU_n$, $\dot\cU \coloneq \lock_\mu^*\dot\cU_n$, $\typ \coloneq \lock_\mu^*\typ_n$, and $\var
  \coloneq \var_\mu$.
\end{remark}

\begin{remark}
  Since we often want to consider type theories with additional type formers, and not just the bare bones type theories given
  by natural models and comprehension categories, it would be interesting in the future to see if we can specify multimode
  type theories in terms of some tensor product of theories with the mode theory $\cM$, allowing you to specify a base type
  theory at each mode $m \in \cM$. Thus could also allow for a more modular approach to the semantics of multimode type
  theories, easily allowing a different base semantics, such as for example comprehension categories, hopefully without
  having to reprove all the theorems we would have for the semantics presented here in terms natural models.
\end{remark}

As with the other type formers, we can define the modal types by adding, for each modality $\mu : n \to m \in \cM$, a pullback
square as follows:
\[\begin{codi}[tetragonal=base 6em height 4.5em]
  \obj {|(tl)| \lock_\mu^*\dot\cU_n & |(tr)| \dot\cU_m \\ |(bl)| \lock_\mu^*\cU_n & |(br)| \cU_m \\};
  \mor tl \mod_\mu:-> tr u_m:-> br;
  \mor[swap] tl \lock_\mu^*u_m:-> bl \Mod_\mu:-> br;
\end{codi}\]
\begin{remark}
  This definition is stronger than the one presented in \cite{gratzer2021}, where it is shown that these modal types correspond
  to \emph{dependent right adjoints}. \textcite{gratzer2021} also provides a version of multimode type theory with a slgithyl
  weaker notion of modal types, modeled similarly to how \cite{awodey2017} models intentional identity types. Much like with
  intentional identity types, I have not had the time to figure out how those approaches fit into this framework. Thus we will
  take the slightly stronger notion, which however still captures many examples of modal types.
\end{remark}


% This does not yet account for the formation, introduction, elimination, and computation rules for the modal types,
% which is thus the next step. The modal type formation and introduction rules are given by
% \[\inferrule{\mu : m \to n \\ \Gamma.\lock_\mu \vdash A\;\cU_n}{\Gamma \vdash \seq{\mu \given A}\;\cU_m}
% \qquad\inferrule{\mu : m \to n \\ \Gamma.\lock_\mu \vdash a : A}{\Gamma \vdash \mathrm{mod}_\mu(a)
% : \seq{\mu \given A}}\]
% These then correspond to, for each $\mu : m \to n$, additional commuting square
% \[\begin{codi}[tetragonal=base 4.5em height 4.5em]
%   \obj { |(tl)| \lock_\mu^*\dot\cU_m & |(tr)| \dot\cU_n \\ |(bl)| \lock_\mu^*\cU_m & |(br)| \cU_n \\};
%   \mor tl {{\mathrm{mod}_\mu}}:-> tr \typ_n:-> br;
%   \mor[swap] tl \lock_\mu^*\typ_m:-> bl {{\seq{\mu\given -}}}:-> br;
% \end{codi}\]
% such that $u_n \circ \seq{\mu\given -} = \lock_\mu^* u_m$. For each $\nu : o \to n$ and $\mu : n \to m$, the elimination
% rule is given as
% \[\inferrule{\Gamma\;\cC_m \\ \Gamma.\lock_\mu.\lock_\nu \vdash A \; \cU_o \\
% \Gamma.\lock_\mu \dashv M_0 : \seq{\nu \given A} \\ \Gamma.(\mu\;\vert\;\seq{\nu \given A}) \vdash B\;\cU_m \\
% \Gamma.(\mu \circ \nu\;\vert\; A) \vdash M_1 : B[\set{\mathrm{mod}_\nu(\tau \var A)}]}
% {\Gamma \vdash \elim_\mu^\nu(M_0, M_1) : B[\set{M_0}]}\]
% \todo[inline]{Show how to encode the eliminator, it's not hard, just annoying.}

\ifSubfilesClassLoaded{\printbibliography}{}
\end{document}
