\documentclass[../thesis.tex]{subfiles}

\begin{document}
\chapter{Preliminaries}\label{ch:prelims}
This chapter is to lay out the groundworks of the concepts we will make use of in the rest of the thesis.

% \todo{Move the following lemma to preliminaries.}
% \begin{lemma}
%   The full embeddings are the right class of an orthogonal factorisation system on $\Cat$, where
%   the left class are surjective-on-objects functors.
% \end{lemma}
% \begin{proof}
%   Given any functor $F : A \to B$, let $A'$ denote the full subcategory of $B$ spanned by the
%   objects in the image of $F$. Clearly $F$ then factors through $A'$ as
%   \[\begin{codi}
%     \obj {A & & B \\
%           & A' & \\ };
%     \mor A F:-> B;
%     \mor[swap] A G:-> A' H:-> B;
%   \end{codi}\]
%   where $G$ is surjective on objects and $H$ is fully faithful and injective on objects.
% \end{proof}

\section{Finite limit theories and sketches}
The basic notion of theory we will generalize is that of finite limit theories, or essentially algebraic theories, which allow
for specifying not just total algebraic operations, but also partial ones whose domains are determined by algebraic equations.
For example, any algebraic theory is a finite limit theory, but also the theory of categories. 

\begin{definition}[Finite limit theory]
  A \emph{finite limit theory} is a small, finitely complete category $\cT$. A \emph{model} of a finite limit theory in a
  finitely complete category $\cC$ is a finite limit-preserving functor $\cT \to \cC$. A \emph{homomorphism} of models
  $M, N : \cT \to \cC$ is a natural transformation $M \to N$. We denote by $\Mod_\cT(\cC)$ the full subcategory of
  the functor category $\cC^\cT$ on models.
\end{definition}
\begin{literature}
\end{literature}

To ease the construction of finite limit theories, we will make use of the notion of finite limit sketches. Furthermore,
all of the theories we will construct in chapters 3 and 4 are given in terms of their sketch. The reason we consider
theories still is that they provide a good way to say when two theories are equivalent, namely by categorical equivalence,
which is slightly more involved for sketches.
\begin{definition}[Finite limit sketch]
  A \emph{finite limit sketch} $(\cC,\cS)$ consists of a small category $\cC$ together with a collection $\sS$ of finite
  cones over digrams in $\cC$. A \emph{model} of the finite limit sketch $(\cC,\cS)$ in a category $\cD$ is a functor
  $\cC \to \cD$ which maps every cone of $\sS$ to a limit cone in $\cD$. We denote the category of models in a category
  $\cD$ by $\Mod_{(\cC,\cS)}(\cD)$.
\end{definition}

\begin{remark}
  Any finite limit theory $\cT$ induces a sketch $(\cT,\cS_\cT)$ in which the cones are exactly the finite limit cones.
  Then a model of $(\cT,\cS_\cT)$ is exactly a finite limit-preserving functor, and so a model of $\cT$. Hence the categories
  of models are the same.
\end{remark}

\begin{theorem}
  Any finite limit sketch $(\cC,\cS)$ may be extended to a finite limit theory $\cT_{(\cC,\cS)}$, such that
  there is a fully faithful embedding $i : \cC \to \cT_{(\cC,\cS)}$ that provides a model of $(\cC,\cS)$ inside
  the sketch induced by $\cT_{(\cC,\cS)}$, and the restriction $i^* : \Mod_\cT(\cD) \to \Mod_{(\cC,\cS)}(\cD)$ is an
  equivalence for any finitely complete category $\cD$.
\end{theorem}

\begin{literature}
  Sketches were first introduced by \citeauthor{ehresmann1969} in \cite{ehresmann1969} as a formalisation of the notion of
  theory. A further generalization in the context of enriched categories is given in provided by \citeauthor{kelly1982a} in
  \cite{kelly1982a}, which we later make use of. The above theorem is also given in the generalized enriched context in
  \cite{kelly1982a}.
\end{literature}

\section{Enriched categories}
\label{sec:enriched}
Enriched categories are generally defined in terms of symmetric monoidal closed categories, but for our purposes
we restrict ourselves to cartesian closed categories, as these suffice for our purposes and lets us elide the
definition of monoidal categories. The main reference for enriched categories used in this thesis is \cite{kelly1982a}.

\begin{definition}[Enriched categories]\label{def:enriched category}
  Let $\sK$ be a cartesian category. An \emph{$\sK$-enriched category} (or $\sK$-category for short) $\bA$ consists of
  \begin{enumerate}
    \item A class of objects $\bA_0$,
    \item For each pair of objects $A,B \in \bA_0$, an object $\bA(A,B)$ of $\sK$,
    \item For each object $A \in \bA_0$ a morphism $\id_A : \mathbf{1} \to \bA(A,A)$,
    \item For each triple of objects $A,B,C \in \bA_0$, a morphism $\comp_{A,B,C} : \bA(B,C) \times \bA(A,B) \to \bA(A,C)$,
  \end{enumerate}
  such that for each $A,B,C,D \in \bA_0$, the following diagrams, corresponding to the unit and associativity laws, commute:
  \begin{mathpar}
    \begin{codi}[tetragonal=base 12em height 4.5em]
      \obj {|(tl)| \bA(A,B) & |(tr)| \bA(A,B) \times \bA(A,A) \\ & |(br)| \bA(A,B) \\};
      \mor tl {{1_{\bA(A,B)} \times \id_A}}:-> tr {{\comp_{A,A,B}}}:-> br;
      \mor[swap] tl {{1_{\bA(A,B)}}}:-> br;
    \end{codi} \and
    \begin{codi}[tetragonal=base 12em height 4.5em]
      \obj {|(tl)| \bA(A,B) & |(tr)| \bA(B,B) \times \bA(A,B) \\ & |(br)| \bA(A,B) \\};
      \mor tl {{\id_B \times 1_{\bA(A,B)}}}:-> tr {{\comp_{A,B,B}}}:-> br;
      \mor[swap] tl {{1_{\bA(A,B)}}}:-> br;
    \end{codi} \\
    \begin{codi}[tetragonal=base 20em height 4.5em]
      \obj {|(tl)| \bA(C,D) \times \bA(B,D) \times \bA(A,B) & |(tr)| \bA(B,D) \times \bA(A,B) \\
      |(bl)| \bA(C,D) \times \bA(A,C) & |(br)| \bA(A,D) \\};
      \mor tl {{\comp_{B,C,D} \times 1_{\bA(A,B)}}}:-> tr {{\comp_{A,B,D}}}:-> br;
      \mor[swap] tl {{1_{\bA(C,D)} \times \comp_{A,B,C}}}:-> bl {{\comp_{A,C,D}}}:-> br;
    \end{codi}
  \end{mathpar}
\end{definition}

\begin{example}[Locall small categories]
  Locally small 1-categories are $\Set$-enriched categories: they are equipped with a class of objects and a \emph{set} of
  morphisms.
\end{example}

\begin{example}[2-categories]\label{ex:2-cat}
  (Strict) 2-categories are exactly the $\Cat$-enriched categories: the 1-cells are the objects of the hom-categories and
  the 2-cells the morphisms. Vertical composition is the composition of the hom-categories and horizontal composition is
  the action on morphisms of the composition bifunctor of the $\Cat$-enriched category.
\end{example}

\begin{example}
  Let $\bA$ be a $\sK$-enriched category. Then $\bA^\op$, defined by taking the same objects and flipping the direction
  of objects (so $\bA^\op(A,B) = \bA(B,A)$), is a $\sK$-enriched category.
\end{example}

\begin{definition}[Enriched functors]\label{def:enriched functor}
  Let $\bA, \bB$ be $\sK$-enriched categories for some cartesian category $\sK$. An \emph{$\sK$-enriched functor}
  (or $\sK$-functor) $F : \bA \to \bB$ consists of
  \begin{enumerate}
    \item A function $F_0 : \bA_0 \to \bB_0$, and
    \item for each pair of objects $A,B \in \bA_0$, a morphism $F_{A,B} : \bA(A,B) \to \bA(F_0 A, F_0 B)$
  \end{enumerate}
  such that for each $A,B,C \in \bA$, we have
  \[F_{A,A} \id_A = \id_{F_0 A} \qquad\text{and}\qquad
  F_{A,C}\;\comp_{A,B,C} = \comp_{F_0A,F_0B,F_0C} (F_{B,C} \times F_{A,B}).\]

  The identity $\sK$-functors and composition of $\sK$-functors are defined in the obvious way.
\end{definition}

\begin{example}[Self-enrichment]\label{ex:self enrichment}
  Suppose $\sK$ is a cartesian closed category. Then $\sK$ can canonically be made into a $\sK$-category $\bK$ as follows:
  \begin{enumerate}
    \item The objects are those of $\sK$,
    \item for objects $A,B \in \sK$, the hom-object $\bK(A,B)$ is exactly the internal hom $B^A$,
    \item for each object $A \in \sK$, the identity is exactly the transpose of the projection $\mathbf{1} \times A \to A$,
    \item for each triple of objects $A,B,C \in \sK$, the composition is the internal composition map $C^B \times A^B \to C^A$.
  \end{enumerate}
\end{example}

\begin{example}
  Let $\bK$ be as before, $\bA$ be some $\sK$-category, and $X$ an object of $\bA$. Then there is a $\sK$-functor $\bA(X,-) :
  \bA \to \bK$ given on objects by $\bA(X,-)(A) = \bA(X,A)$. The morphism component $\bA(X,-)_{A,B} : \bA(A,B) \to \bA(X,B)^
  \bA(X,A)$ is given by the transpose of the composition map $\comp_{X,A,B} : \bA(A,B) \times \bA(X,A) \to \bA(X,B)$, for
  all objects $A,B \in \bA$.
\end{example}

\begin{definition}[Enriched natural transformation]\label{def:enriched transformation}
  Let $\bA, \bB$ be $\sK$-enriched categories for some cartesian category $\sK$ and $F,G : \bA \to \bB$ enriched
  functors between them. An \emph{$\sK$-enriched natural transformation} (or $\sK$-natural transformation)
  $\alpha : F \to G$ consists of a family $(\alpha_A : \mathbf{1} \to \bB(F_0 A, G_0 A))_{A \in \bA_0}$ such that
  for each $A,B \in \bA$ the following square, corresponding to naturality, commutes:
  \[\begin{codi}[tetragonal=base 20em height 4.5em]
    \obj {|(tl)| \bA(A,B) & |(tr)| \bB(F_0 B, G_0 B) \times \bB(F_0 A, F_0 B) \\
    |(bl)| \bB(G_0 A, G_0 B) \times \bB(F_0 A, G_0 A) & |(br)| \bB(F_0 A, G_0 B) \\};
    \mor tl {{\alpha_B \times F_{A,B}}}:-> tr {{\comp_{F_0 A, F_0 B, G_0 B}}}:-> br;
    \mor[swap] tl {{G_{A,B} \times \alpha_A}}:-> bl {{\comp_{F_0 A,G_0 A, G_0 B}}}:-> br;
  \end{codi}\]
\end{definition}

\begin{example}[Enriched functor category]\label{ex:enriched functor cateory}
  Let $\sK$ be a complete cartesian closed category. Then for any pair of $\sK$-categories $\bA, \bB$ such that the objects
  of $\bA$ form a set, we can form the $\sK$-functor category $[\bA,\bB]$, whose objects are $\sK$-functors $\bA \to \bB$ and
  hom-objects are an internalization of the $\sK$-natural transformations. Specifically, the hom-object $[\bA,\bB](F,G)$ is
  given as the largest subobject of the product \(\prod_{A \in \bA} \bB(F_0 A, G_0 A)\) such that for every $A,B \in \bA$
  the square
  \[\begin{codi}[tetragonal=base 20em height 4.5em]
    \obj {|(tl)| [\bA,\bB](F,G) \times \bA(A,B) & |(tr)| \bB(F_0 B, G_0 B) \times \bB(F_0 A, F_0 B) \\
      |(ml)| \bA(A,B) \times [\bA,\bB](F,G) \\
    |(bl)| \bB(G_0 A, G_0 B) \times \bB(F_0 A, G_0 A) & |(br)| \bB(F_0 A, G_0 B) \\};
    \mor tl {{\pi_B \times F_{A,B}}}:-> tr {{\comp_{F_0 A, F_0 B, G_0 B}}}:-> br;
    \mor[swap] tl \sigma:-> ml {{G_{A,B} \times \pi_A}}:-> bl {{\comp_{F_0 A,G_0 A, G_0 B}}}:-> br;
  \end{codi}\]
  where $\sigma : [\bA,\bB](F,G) \times \bA(A,B) \to \bA(A,B) \times [\bA,\bB](F,G)$ is the symmetry isomorphism of the
  product. This can be described as an equalizer, and so is well-defined as a limit. Note that the global elements of
  the hom-object $[\bA,\bB](F,G)$, i.e. the maps $\mathbf{1} \to [\bA,\bB](F,G)$ are exactly the $\sK$-natural transformations.
\end{example}

\subsection{Limits in enriched categories}
\label{sec:enriched/limits}
Given that the basis for this thesis is enriched finite limit theories, we will of course need to have and understanding of
enriched limits. This is especially important as the conical approach of standard category theory doesn't work out quite as
nicely for enriched categories. For example, unlike ordinary category theory, we find that for general $\sK$, the enriched
presheaf category $[\bA^\op,\bK]$ is not the free cocompletion. Furthermore, there are limit-like in for, example 2-categories,
that we would like to capture, but which aren't conical in the classical sense. Therefore we introduce, and will make heavy
use of, weighted limits. 

For the remainder of this section, we fix a cartesian closed category $\sK$, which will serve as the base of enrichment.
We will use $\bK$ to denote the canonical $\sK$-category induced by $\sK$ itself, as in \cref{ex:self enrichment}.

\begin{definition}[Weights, Diagrams, Weighted cones]
  Let $\bJ$ be a small $\sK$-category. A \emph{weight} is a $\sK$-functor $W : \bJ \to \bK$. For $\bA$ another $\sK$-category,
  a \emph{$\bJ$-shaped diagram} in $\bA$ is a $\sK$-functor $D : \bJ \to \bA$. For a weighted $W$ and diagram $D$ with common
  domain $\bJ$, a \emph{$W$-weighted cone over $D$} based at $A \in \bA$ is a $\sK$-natural transformation $[\bJ, \bK](W,
  \bA(A,D))$, where $\bA(A,D) = \bA(A,-) \circ D$.
\end{definition}

\begin{definition}[Weighted limit]\label{def:weighted limit}
  Let $W : \bJ \to \bK$ and $D : \bJ \to \bA$ be a weight and diagram respectively. The \emph{$W$-weighted limit of $D$} is
  an object $L \in \bA$ and a $\sK$-natural isomorphism
  \[\bA(-,L) \cong [\bJ,\bK](W,\bA(-,D)).\]
\end{definition}

\begin{example}[Conical limits]
  We recover the standard conical limits by taking the weight to be constantly the terminal object.
\end{example}

\begin{example}[Cotensors]
  Let $X \in \sK$ be some object, $\bA$ some $\sK$-category, and $A \in \bA$ some object of $\bA$. The \emph{cotensor}
  of $A$ by $X$, written $A^X$, is the limit with of the diagram $D(*) = A$, using the terminal $\sK$-category as shape,
  weighted by $W(*) = X$. Thus the copower $A^X$ is determined by a natural isomorphism
  \[\bA(-,A^X) \cong \bK(X,\bA(-,A)).\]
\end{example}

\section{2-categories}
\label{sec:2-cats}
We cover some basic properties of 2-categories and the limits therein. For a more thorough introduction to the subject of
2-categories, we refer the reader to chapter 7 of \cite{borceux1994a}, which has also served as the reference for most of
the definitions used here.

\begin{definition}[2-categories]
  A 2-category is a category enriched in $\Cat$, as shown in \cref{ex:2-cat}. A 2-functor between 2-categories $\cA \to \cB$
  is a $\Cat$-enriched functor, and a 2-natural transformations is a $\Cat$-enriched natural transformation.
\end{definition}

\begin{notation}
  We will sometimes denote the objects of a 2-category by 0-cells, the objects of the hom-categories by morphisms, maps or
  1-cells, and the morphisms of the hom-categories by 2-cells.

  We use $\circ$ for the object component of the composition functor and the composition of 2-cells, and $\star$ for the
  morphism part of the composition functor, the horizontal composition of 2-cells. We may also write composition of 1-cells
  by juxtaposition where convenient.
\end{notation}

\begin{example}
  Since $\Cat$ is cartesian closed, it cannonically induces a 2-category, whose objects are categories, morphisms are functors,
  and 2-cells are natural transformations.

  Similarly, many  categories of structured categories are often naturall viewed as 2-categories. For example, consider the
  2-category of finitely complete categories, whose objects are finitely complete categories, morphisms are lex functors, and
  2-cells are natural transformations.
\end{example}

\begin{example}
  Any category $\sA$ can be viewed as a 2-category $\cA$ by taking the hom-categories $\cA(A,B)$ to be the discrete category
  associated with the homset $\sA(A,B)$.
\end{example}

\begin{definition}[Whiskering]
  Let $\cA$ be a 2-category, $f : A \to B$ a morphism of $\cA$, and $\alpha : g \to h : B \to C$ a 2-cell of $\cA$. We
  define the left whiskering of $f$ and $\alpha$, written $f\alpha : fg \to fh : A \to C$, to be the horizontal composite
  $1_f \star \alpha$.  
\end{definition}
\begin{notation}
  We will generally write whiskering by juxtaposition, with left (resp. right) whiskering inferred from which is the 2-cell
  and which is the 1-cell. Note that, in a 2-category, we have that $f(g\alpha) = (fg)\alpha$ for composable 1-cells $f,g$
  and a 2-cell $\alpha$, which means that parentheses can generally be elided.
\end{notation}

\begin{definition}[Pseudonatural transformations]
  Let $F,G : \cA \to \cB$ be 2-functors. A pseudonatural transformation $\alpha : F \to G$ consists of the following
  data:
  \begin{enumerate}
    \item For each object $A \in \cA$, a 1-cell $\alpha_A : FA \to GA$,
    \item For each morphism $f : A \to B$ in $\cA$, an isomorphism $\alpha_f$ as shown in the following square
      \[\begin{codi}
        \obj {FA & FB \\ GA & GB \\}; 
        \mor FA Ff:-> FB \alpha_B:-> GB;
        \mor[swap] FA \alpha_A:-> GA Gf:-> GB;
        \mor GA \alpha_f:-2> FB;
      \end{codi}\]
  \end{enumerate}
  such that for every $A \in \cA$ we have $\alpha_{1_A} = 1_{\alpha_A}$ and for any pair of composable 1-cells $f,g$ in
  $\cA$ we have that $\alpha_{f \circ g} = \alpha_f \circ \alpha_g$.
\end{definition}

\begin{definition}[Modifications]
  Let $\alpha,\beta : F \to G : \cA \to \cB$ be two parallel pseudonatural transformations. A modification $m : \alpha \to
  \beta$ consists of a family $(m_A : \alpha_A \to \beta_A)_{A \in \cA}$ such that for any $\phi : f \to g : A \to B$ in $\cA$
  the equality $m_B \star F\phi = G\phi \star m_A$ holds.
\end{definition}

\begin{definition}[Sub-2-category]
  A \emph{sub-2-category} of a 2-category $\cA$ is a 2-category $\cB$ such that the objects of $\cB$ form a subclass of the
  objects of $\cA$, the hom-categories $\cB(A,B)$ are subcategories of $\cA(A,B)$, and identities and composition in $\cA$
  restrict to $\cB$.
\end{definition}

\begin{definition}[Locally fully faithful]
  A 2-functor $F : \cA \to \cB$ is \emph{locally fully faithful} if for every $A,B \in \cA$, the functor $F_{A,B} :
  \cA(A,B) \to \cB(FA, FB)$ is fully faithful.
\end{definition}

\begin{definition}[Locally full sub-2-category]
  A sub-2-category $\cB \subseteq \cA$ is \emph{locally full} if $\cB(A,B) = \cA(A,B)$ for each $A,B \in \cB$.
\end{definition}

\subsection{2-limits}
Given the definition of weighted limits in \cref{sec:enriched/limits}, we may take the 2-limits to be exactly weighted limits
in $\Cat$-enriched categories. This subsection will cover some useful examples and properties of 2-limits, as they form the
bases of the rest of the thesis. We begin by providing an explicit description of the weighted limits and cones, so as to
provide some intuition for how to construct the correct weights for the limits we wish to describe, and then follow up with
plenty of examples of 2-limits and their universal properties.

\begin{proposition}\label{prop:desc of 2-limits}
  Let $\cJ$, $\cA$ be 2-categories, $W : \cJ \to \Cat$ be a weight, and $D : \cJ \to \cA$ be a $\cJ$-shaped diagram in $\cA$.
  Then a $W$-weighted cone $(a,\alpha)$ over $D$, i.e. an object $a \in \cA$ and a 2-natural transformation $W \to \cA(a,D)$
  consists of the following data:
  \begin{itemize}
    \item for each object $x \in \cJ$ and $r \in W(x)$, a morphism $\alpha_{x,r} : a \to Dx$,
      and
    \item for each $p : r \to s \in W(x)$, a 2-cell $\alpha_{x,p} : \alpha_{x,r} \to \alpha_{x,s}$,
      such that
    \item for each morphism $f : x \to y$ in $\cJ$ and $r \in W(x)$, we have $Df \circ \alpha_%
      {x,r} = \alpha_{y,W(f)(r)}$, and
    \item for each $p : r \to s$ in $W(x)$ we have $Df \star \alpha_{x,p} = \alpha_{y,W(f)(p)}$.
  \end{itemize} 

  Furthermore, a morphism $m : \alpha \to \beta$ of parallel cones $W \to \cA(a,D)$, i.e.\ a modification,
  consists of a family $(m_{x,r} : \alpha_{x,r} \to \beta_{x,r})_{x \in \cJ, r \in W(x)}$ such that for every
  $f : x \to y$ in $\cJ$  and $r \in W(x)$ we have that
  \[m_{y,Wf(r)} = Df m_{x,r} : Df \circ \alpha_{x,r} = \alpha_{y,Wf(r)} \to \beta_{y,Wf(r)}
  = Df \circ \beta_{x,r}.\]
\end{proposition}
\begin{proof}
  This holds immediately after unfolding all the layers of definition.
\end{proof}

\begin{definition}[Pullback of cones]
  Let $\cJ, \cA$ be 2-categories, $W : \cJ \to \cF$ a weight, and $D : \cJ \to \cA$ a diagram.
  Given a $W$-weighted cone $(a,\alpha)$ over $D$ and a morphism $f : b \to a$ in $\cA$, we define
  the pullback of $(a,\alpha)$ along $f$ as the cone $(b,f^*\alpha)$ defined by
  \[(f^*\alpha)_{x,r} = \alpha_{x,r} \circ f\]
  on objects $x \in \cJ$ and $r \in W(x)$, and
  \[(f^*\alpha)_{x,p} = \alpha_{x,p} \star f\]
  on objects $x \in \cJ$ and morphisms $p : r \to s \in W(x)$.

  Given a modification $m : \alpha \to \alpha' : W \to \cA(a,D)$, we define the pullback $f^*m : f^*\alpha
  \to f^*\alpha'$ by
  \[(f^*m)_{x,r} = m_{x,r} \star f\]


  Furthermore, for $\phi : f \to g : b \to a$ a 2-cell in $\cA$, we define a modification $\phi^*\alpha :
  f^*\alpha \to g^*\alpha$ by \((\phi^*\alpha)_{x,r} = \alpha_{x,r} \star \phi\).

  These together form the data of the 2-functor $[\cJ,\Cat](W,\cA(-,D)) : \cA \to \Cat$ taking objects of $\cA$ to
  their 2-categories of cones based at that object, 1-cells $f : b \to a$ of $\cA$ to the pullback of cones as defined
  above, and similarly for 2-cells.
\end{definition}

\begin{proposition}\label{prop:chr of 2-limits}
  Given a weight $W : \cJ \to \Cat$, a diagram $D : \cJ \to \cA$ has a limit in $\cA$ if and only if there
  exists a $W$-weighted cone $(\ell,L)$ over $D$ in $\cA$ such that
  \begin{enumerate}
    \item for every cone $\alpha : W \to \cA(a,D)$, there exists a unique morphism $h_\alpha : a \to
      \ell$ such that $\alpha = h_\alpha^*L$,
    \item for every modification $m : \alpha \to \alpha' : W \to \cA(a,D)$ there exists a unique 2-cell
      $h_m : h_\alpha \to h_{\alpha'}$ such that $h_m^*\ell = m$.
  \end{enumerate}
  We will call a cone satisfying these properties a limiting cone.
\end{proposition}
\begin{proof}
  We start by unfolding the definition of the naturality squares of the isomorphism
  \[h : [\cJ,\Cat](W,\cA(-,D)) \cong \cA(-,\ell)\]
  which is supposed to exist for $\ell$ to be the $W$-weighted limit of $D$. For each $a \in \cA$, we find that
  $h_a : [\cJ,\Cat](W,\cA(a,D)) \cong \cA(a,\ell)$ is an isomorphism in $\Cat$, i.e.\ an invertible functor.
  Naturality tells us that given a morphism $\phi : b \to a$, we have that $h_b \circ \phi^* = \cA(\phi,\ell)
  \circ h_a$, where $\phi^*$ denotes the operation of pulling back a cone along a morphism of $\cA$. On cones
  $\alpha : W \to \cA(b,\ell)$, this gives us that $h_b(\phi^*\alpha) = h_a(\alpha) \circ \phi$, and on
  modifactions $m : \alpha \to \alpha' : W \to \cA(b,D)$, we have $h_b(\phi^*m) = h_a(m) \star \phi$.

  Now, given a natural isomorphism $h$ as above, we can consider the cone $L = h^{-1}(\id_\ell)$. Given any
  other cone $\alpha : W \to \cA(a,D)$, naturality tells us that
  \[h_a(h_a(\alpha)^*L) = h_\ell(L) \circ h_a(\alpha) = \id_\ell \circ h_a(\alpha) = h_a(\alpha),\]
  so $h_a(\alpha)^*L = \alpha$ since $h$ is an isomorphism. Similarly, for a modification $m : \alpha \to
  \alpha'$ we find that
  \[h_a(h_a(m)^*L) = h_a(m) = h_\ell(L) \star h_a(m) = \id_\ell \star h_a(m) = h_a(m),\]
  so again $h_a(m)^*L = m$.


  In the other direction, suppose we are given an object $\ell \in \cA$ and cone $L : W \to \cA(\ell,D)$
  such that (1) and (2) hold. Then we construct the natural isomorphism $h : [\cJ,\Cat](W,\cA(-,D)) \cong
  \cA(-,\ell)$ by $h_a(\alpha) = h_\alpha$ and $h_a(m : \alpha \to \alpha') = h_m$. Verification that $h_a$
  is indeed an isomorphism for each $a \in \cA$ is simple to verify, so we focus instead only on naturality:
  given $\phi : b \to a \in \cA$, we must show that $h_b \circ \phi^* = \cA(\phi,\ell) \circ h_a$. On cones,
  this amounts to showing that for any cone $\alpha : W \to \cA(a,D)$, we have that $h_b(\phi^*\alpha) =
  h_a(\alpha) \circ \phi$. But
  \[
    \phi^*\alpha = \phi^*h_a(\alpha)^*L = (h_a(\alpha) \circ \phi)^*L,
  \]
  so uniqueness of $h_b(h_a(\alpha) \circ \phi)$ gives us that $h_b(\phi^*\alpha) = h_a(\alpha) \circ \phi)$.
  For the naturality condition on 2-cells, let $m : \alpha \to \alpha'$ be a modification. We must show that
  $h_b(\phi^*m) = h_a(\alpha) \star m$, which much like for cones follows from uniqueness:
  \[\phi^*m = \phi^*(h_a(m)^*L) = (h_a(m)\star\phi)^*L,\]
  so $h_b(\phi^*m) = h_a(m)\star\phi$.
\end{proof}

\begin{example}[Arrow object]
  Let $\cA$ a 2-category and $A \in \cA$ be given. The arrow-object of $A$, written $A^\to$, is the cotensor by the category
  $2 = \set{0 \to 1}$. By the above explicit description, $A^\to$ is equipped with two arrows $\dom_A,\cod_A : A^\to \to A$ and
  a 2-cell $\lambda_A : \dom_A \to \cod_A$, and this classifies 2-cells into $A$, in the sense that for any 2-cell $\alpha
  : f \to g : X \to A$, there exists a unique 1-cell $h : X \to A^\to$ such that $f = \dom_A h$, $g = \cod_A h$, and
  $\alpha = \lambda_A h$, and a corresponding universal property for 2-cells.

  Equivalently, there is for every $X \in \cA$ an isomorphism of categories $\cA(X,A^\to) \cong \cA(X,A)^2$,
  where the latter denotes the arrow category, and this isomorphism is natural in $X$ in a suitable sense.
\end{example}

\begin{example}[Comma object]
  Let $\cA$ be a 2-category and $A \xrightarrow{f} C \xleftarrow{g} B$ a cospan in $\cA$. The comma object for $f,g$ is an
  object $f \comma g$ equipped with a limiting cone $(\dom_{f,g}, \cod_{f,g}, \lambda_{f,g})$ for the diagram given by the
  aforementioned cospan and weighted by $1 \xrightarrow{0_2} 2 \xleftarrow{1_0} 1$, where $2 = \set{0_2 \to 1_2}$ is the
  walking arrow and $1$ is the terminal category. The universal property tells us that for any span $A \xleftarrow{p} X
  \xrightarrow{q} B$ with a 2-cell $\alpha : fp \to gq$, there exists a unique 1-cell $h : X \to f \comma g$ such that
  $\dom_{f,g} h = p$, $\cod_{f,g} h = q$, and $\lambda_{f,g} h = \alpha$, as well as a corresponding property for 2-cells.
\end{example}

\begin{example}[1-limits]
  All 1-limits, such as products, pullbacks, equalizers, etc.\ lift to 2-limits by taking the weight to be constantly the
  terminal object. As such, we will speak of these 1-limits also in the context of 2-categories. 
\end{example}

\begin{example}[Inserters]
  Given two parallel 1-cells $f,g : a \to b$ in a 2-category $\cA$, their \emph{inserter} is the limit of these arrows
  weighted by
  \[\begin{codi}
    \obj {1 & 2 \\};
    \mor 1 0_2:[->,bend left] 2;
    \mor[swap] 1 1_2:[->,bend right] 2;
  \end{codi}\]
  with $1$ and $2$ defined as above. Thus the inserter consists of an object $v$, a 1-cell $i : v \to a$, and a 2-cell
  $\lambda : fi \to gi$, with a corresponding universal property.

  It can be also be constructed as the following pullback:
  \[\begin{codi}  
    \obj {v & b^\to \\ a & b \times b \\};
    \mor v \bar\lambda:-> b^to {{(\dom_b,\cod_b)}}:-> (b times b);
    \mor[swap] v i:-> a {{(f,g)}}:-> (b times b);
  \end{codi}\]
  where $\bar\lambda$ is the unique 1-cell such that $\lambda_b \bar\lambda = \lambda : fi \to gi$.
\end{example}

\section{Fibrations}
\label{sec:fibrations}
Originally introduced by Grothendieck in \cite{grothendieck1960} in the context of descent theory, fibrations
have turned out to be an immensely useful tool in the categorical modelling of substitution in type theory.
In essence, a fibration encodes the idea of a family of categories $(\cE_b)_{b \in \cB}$ indexed over a base
category $\cB$, equipped with actions of morphisms $f^* : \cE_b \to \cE_{b'}$ for each morphism $f : b' \to
b$. In essence, a fibration is a pseudofunctor $\cE_{(-)} : \cB \to \Cat$. However, rather than the complex
algebraic structure that is a pseudofunctor, it turns out that we can instead describe this structure by
a \emph{functor} $p : \cE \to \cB$, where we call $\cE$ the total category of the fibration $p$, with
the idea that we can define $\cE_{b}$ to be the fiber of $p$ over $b$: the objects $e$ are those of $\cE$
which map to $b$, i.e.\ $p^{-1}(b) = \set{e \given p(e) = b}$, and morphisms $k : e \to e'$ are those of
$\cE$ which lie over the identity, meaning that $p(k) = 1_b$. For a good textbook account of fibrations
in $\Cat$ and their applications to logic, I recommend \cite{jacobs1998}. 

Since we will be working in general 2-categories, rather than only $\Cat$, we will require a generalization
of fibrations to this setting. This was initially done by \citeauthor{street1974} in \cite{street1974}, and
was further developed in \cite{weber2007}. A good overview of these generalizations, both to 2-categories
and bicategories, can be found in \cite{loregian2020}. Here we will simply provide the definitions that
will be required for the later chapters, without much general motivation on why they are defined as they
are.

The main concept underlying fibrations is that of \emph{cartesian morphisms}. These are certain 2-cells satisfying
a universal property with respect to some fixed 1-cells $p : A \to B$ in a 2-category $\cA$, with the name coming
from the universal property in a sence mimicing that of pullbacks.
\begin{definition}[Cartesian morphisms]
  Let $p : A \to B$ be a morphism of an 2-category $\cA$. A 2-cell $\phi : a_1 \to a_2 : X \to A$ is
  called \emph{$p$-cartesian} if for every $g : Y \to X$, $a_0 : Y \to A$, $\alpha : a_0 \to a_2g$,
  and $\beta : pa_0 \to pa_1g$ with $\beta \circ p\phi = p\alpha$, there exists a unique $\gamma : a_0 \to
  a_1g$ such that $\beta = p\gamma$ and $\alpha = \phi g \circ \gamma$. We will call the data $(g,a_0,
  \alpha,\beta)$ satisfying the equality $\beta \circ p\phi = p\alpha$ a cartesian cone.
\end{definition}
\begin{example}
  Let $\cod : A^\to \to A$ denote the codomain projection for some object $A$ in a finitely complete 2-category $\cA$.
  A 2-cell $\alpha : a_1 \to a_2 : X \to A^\to$ is cartesian precisely if the following square is a pullback:
  \[\begin{codi}[tetragonal=base 6em height 4.5em]
    \obj {\dom a_1 & \dom a_2 \\ \cod a_1 & \cod a_2 \\};
    \mor (dom a_1) \dom\alpha:-> (dom a_2) \lambda a_2:-> (cod a_2);
    \mor[swap] (dom a_1) \lambda a_1:-> (cod a_1) \cod\alpha:-> (cod a_2);
  \end{codi}\]
\end{example}

\begin{definition}[Fibrations]
  A map $p : A \to B$ in a 2-category $\cA$ is a \emph{fibration} if every $\phi : b \to p a : X \to B$ has
  \emph{$p$-cartesian lift}, that is, a $p$-carteisan 2-cell $\bar\phi : \bar b \to a$ with $p\bar b = b$ and
  $p\bar\phi = \phi$.
\end{definition}

\begin{proposition}
  Let $p : A \to B$ be a morphism in an 2-category $\cA$ with finite weighted limits and $\phi : a_1 \to a_2
  : X \to A$ a 2-cell. Consider the diagram $D : \cJ \to \cA$ given by
  \[\begin{codi}
    \obj { X & A & B \\};
    \mor X ["a_1",name=x]:[->,bend left] A p:-> B;
    \mor X ["a_2",name=y,swap]:[->,bend right] A;
    \mor x \phi:-2> y;
  \end{codi}\]
  and weight $W : \cJ \to \Cat$ generated by
  \[\begin{codi}
    \obj { |(X)| 1 & |(A)| H & |(B)| 3 \\};
    \mor X ["1",name=x]:[->,bend left] A c:-> B;
    \mor X ["2",name=y,swap]:[->,bend right] A;
    \mor x \phi:-2> y;
  \end{codi}\]
  where $H = \set{0 \to 2 \from 1}$ is the walking cospan, and $c : H \to 3$ maps the morphism $1 \to 2$ to
  the corresponding morphism in $3$ and $0 \to 2$ to the composite $0 \to 1 \to 2$ in $3$.

  A $W$-weighted cone over $D$ is exactly a cartesian cone over $\phi$.
\end{proposition}

\begin{proposition}\label{prop:cartesian by limit}
  Let $\cA$ be an 2-category with finite weighted limits, $p : A \to B$ a 1-cell in $\cA$, and $\phi : a_1 \to
  a_2 : X \to A$ a 2-cell in $\cA$. Then $\phi$ is cartesian if and only if $\ell = (\cod, \dom, \phi\cod \circ
  \lambda, p\lambda)$ is a universal cartesian cone with apex $A \comma a_1$.
\end{proposition}
\begin{proof}
  In the left-to-right direction, suppose $\phi$ is cartesian. We must show that for every object $Y$, the following
  hold:
  \begin{enumerate}[label=(\arabic*)]
    \item for every cartesian cone $(g,a,\alpha,\beta)$ with apex $Y$, there exists a unique morphism $h : Y \to A
      \comma a$ such that $g = \cod h$, $a = \dom h$, $\alpha = (\phi\cod \circ \lambda)h$ and $\beta = p\lambda h$.
    \item for every morphism of cones $m : (g,a,\alpha,\beta) \to (g',a',\alpha',\beta')$ there exists a unique
      morphism $h_m : h \to h'$ such that $m = h_m^* \ell$.
  \end{enumerate}
  For (1), note that since $(g,a,\alpha,\beta)$ is a cartesian cone and $\phi$ is cartesian, there exists a unique
  2-cell $\gamma : a \to a_1g$ such that $\alpha = \phi \circ \gamma$ and $\beta = p\gamma$. By the universal
  property of $A \comma a_1$, there then exists a unique morphism $\bar\gamma : Y \to A \comma a_1$ with $\lambda\bar
  \gamma = \gamma$. In particular, we have that $\cod\bar\gamma = g$, $\dom\bar\gamma = a$, $\alpha = \phi \circ
  \lambda\bar\gamma$, and $\beta = p\lambda\bar\gamma$, so taking $h = \bar\gamma$ we are done. Furthermore, $h =
  \bar\gamma$ is the unique such map by uniqueness of $\bar\gamma$ and $\gamma$.

  For (2), suppose we are given a modification $m : (g,a,\alpha,\beta) \to (g',a',\alpha',\beta')$. Concretely, $m$
  consists of two morphisms $m_1 : g \to g'$ and $m_2 : a \to a'$ such that the following squares commute:
  \[\begin{codi}[tetragonal=base 6em height 4.5em]
    \obj { a & a' & & pa & pa' \\ a_2g & a_2g' & & pa_1g & pa_1g' \\};
    \mor[swap] a \alpha:-> a_2g a_2m_1:-> a_2g';
    \mor a m_2:-> a' \alpha':-> a_2g';

    \mor pa pm_2:-> pa' \beta':-> pa_1g';
    \mor[swap] pa \beta:-> pa_1g pa_1m_1:-> pa_1g';
  \end{codi}\]
  We want to show that the square
  \[\begin{codi}[tetragonal=base 6em height 4.5em]
    \obj {a & a' \\ a_1g & a_1g' \\};
    \mor a m_2:-> a' \gamma':-> a_1g';
    \mor[swap] a \gamma:-> a_1g a_1m_1:-> a_1g';
  \end{codi}\]
  commutes as well, as that induces the desired unique morphism $\bar m : \bar \gamma \to \bar \gamma'$ with the
  desired properties. To that end, consider the cartesian cone over $Y$ given by $(g', a, \alpha'' = \alpha' \circ
  m_2, \beta'' = \beta'  \circ pm_2)$. To see that this is indeed a cone, we must show that $p\alpha'' = p\phi \circ
  \beta''$, but this follows from simple computation:
  \[p\alpha'' = p(\alpha' \circ m_2) = p\alpha' \circ pm_2 = p\phi \circ \beta' \circ pm_2 = p\phi \circ \beta''\]
  Since $(g',a,\alpha'',\beta'')$ is a cartesian cone and $r$ is cartesian, there exists a unique lift $\gamma''$
  with $p\gamma'' = \beta''$ and $\alpha'' = \phi g' \circ \gamma''$. But we have already that
  \[p(\gamma' \circ m_2) = p\gamma' \circ pm_2 = \beta' \circ pm_2 = \beta'' \text{\quad and \quad}
  \phi g' \circ \gamma' \circ m_2 = \alpha' \circ m_2 = \alpha'',\]
  and similarly
  \[p(a_1m_1 \circ \gamma) = pa_1m_1 \circ p\gamma = pa_1m_1 \circ \beta = \beta''\]
  and
  \[\phi g' \circ a_1m_1 \circ \gamma = a_2m_1 \circ \phi g \circ \gamma = a_2 m_1 \circ \alpha = \alpha'',\]
  where we use the interhcange law to commute $\phi$ and $m_1$.
  In particular, we see that both $\gamma' \circ m_2$ and $a_1m_1 \circ \gamma$ provide universal morphisms for
  the cartesian cone $(g',a,\alpha'', \beta'')$, from which we conclude that they must be identical by the uniqueness
  of such morphisms. Hence the diagram
  \[\begin{codi}[tetragonal=base 6em height 4.5em]
    \obj {a & a' \\ a_1g & a_1g' \\};
    \mor a m_2:-> a' \gamma':-> a_1g';
    \mor[swap] a \gamma:-> a_1g a_1m_1:-> a_1g';
  \end{codi}\]
  commutes.

  For the right-to-left direction, suppose that $(\cod,\dom,\phi\cod \circ \lambda, p\lambda)$ is a limiting cone
  with apex $A \comma a_1$, and let $(g,a,\alpha,\beta)$ be any cartesian cone with apex $Y$. By the universal
  property of limiting cones, there exists a unique morphism $\bar\gamma : Y \to A \comma a_1$ such that
  $\cod\bar\gamma = g$, $\dom\bar\gamma = a$, $(\phi\cod \circ \lambda)\bar\gamma = \alpha$, and
  $p\phi\bar\gamma = \beta$. In particular, taking $\gamma = \lambda\bar\gamma : a \to a_1g$, we have the
  desired morphism, and it is unique by the universal property of $A \comma a_1$.
\end{proof}
\begin{corollary}
  Suppose $\cA, \cB$ are finitely complete 2-categories and $F : \cA \to \cB$ is a 2-functor preserving finite
  2-limits. Further let $p : A \to B \in \cA$ and $\alpha : a_1 \to a_2 : X \to A$ be given. If $\alpha$ is $p$%
  -cartesian, then $F\alpha$ is $Fp$-cartesian.
\end{corollary}

\begin{proposition}
  Let $p : A \to B$ be a morphism in an 2-category $\cA$ with finite 2-limits. The following are equivalent:
  \begin{enumerate}[label=(\arabic*)]
    \item $p$ is a fibration.
    \item There exists a map $\ell : B \comma p \to A$ with $p\ell = \dom$ and $p$-cartesian 2-cell $\epsilon :
      \ell \to \cod$ with $p\epsilon = \lambda$, where $\lambda : \dom \to p\cod$ is the universal 2-cell for $B
      \comma p$.
    \item The map $t : A^\to \to B \comma p$ has a section $r : B \comma p \to A^\to$ which is $p$-cartesian.
    \item The canonical map $t : A^\to \to B \comma p$ has a right adjoint $r : B \comma p \to A^\to$
      with identity counit.
    \item The map $i : A \to B \comma p$ induced by the identity $1_p$ has a right adjoint $\ell : B \comma p
      \to A$ in the slice $\cA/B$ with invertible unit.
  \end{enumerate}
\end{proposition}
\begin{proof}
  We show that $(1) \iff (2) \implies (3) \implies (4) \implies (2)$ and $(2) \iff (5)$.

  For $(1) \implies (2)$, note that since $p$ is a fibration, there exists a $p$-cartesian lift $\epsilon : \ell
  \to \cod : B \comma p \to A$ for the 2-cell $\lambda : \dom \to p\cod : B \comma p \to B$, and this lift
  satisfies $p\epsilon = \lambda$. This is exactly the map we require.

  For $(2) \implies (1)$, note that since $\epsilon : \ell \to \cod$ is $p$-cartesian, so is $\epsilon x$ for
  all $f : X \to B \comma p$. In particular, for any 2-cell $\phi : b \to pa$ with $b : X \to B$ and $a : X
  \to A$, we have an induced 1-cell $f : X \to B \comma p$ with $\lambda f = \phi$. Then $\epsilon f$ is $p$-%
  cartesian, and by the properties of $\epsilon$, we have that $p\epsilon f = \lambda f = \phi$, so $\epsilon f$
  is a cartesian lift for $\phi$. Since $\phi$ was an arbitrary such 2-cell, we conclude that every such 2-cell
  has a caresian lift, which is exactly what it means for $p$ to be a fibration.


  For $(2) \implies (3)$, note that $\epsilon$ induces a map $r : B \comma p \to A^\to$ with $\lambda'r =
  \epsilon$, where $\lambda' : \dom' \to \cod' : A^\to \to A$ is the universal 2-cell associated with $A^\to$.
  This map then satisfies that $\lambda tr = p\lambda' r = p\epsilon = \lambda$, so that $tr = 1_{B \comma p}$,
  and $r$ is $p$-cartesian by definition.

  suppose that $p$ is a fibration. We require first of all a map $r : B \comma p \to
  A^\to$, which we then show is a $p$-cartesian section to $t : A^\to \to B \comma p$. Such a map $r$ is
  equivalently a 2-cell $\lambda r : \dom r \to \cod r : B \comma p \to A$. Since $p$ is a fibration, there
  exists a $p$-cartesian lift $\bar r : x \to \cod' : B \comma p \to A$ for the map $\lambda' : \dom' \to p
  \cod' : B \comma p \to B$, where $x : B \comma p \to A$ is such that $px = \dom'$. This map $\bar r$ induces
  a 1-cell $r : B \comma p \to A^\to$ with $\lambda \bar r = r$. In particular, we see that $\lambda' t r =
  p\lambda r = p\bar r = \lambda'$, so that $tr = 1_{B \comma p}$, as desired. Hence $r : B \comma p \to A^\to$
  is a $p$-cartesian section of $t$.

  For $(3) \implies (4)$, suppose $r$ is a $p$-cartesian 1-cell with $tr = 1_{B \comma p}$. We show that $r$
  is also a right-adjoint to $t$, for which we must find a unit $\eta : 1_{A^\to} \to rt$. By the universal
  property of $A^\to$, $\eta$ is equivalent to a commuting square
  \[\begin{codi}[tetragonal=base 6em height 4.5em]
    \obj {\dom' & \dom'rt \\ \cod' & \cod'rt \\};
    \mor       (dom ') \eta_1:-> (dom 'rt) \lambda'rt:-> (cod 'rt);
    \mor[swap] (dom ') \lambda':-> (cod ') \eta_2:-> (cod 'rt);
  \end{codi}\]
  Since $\cod'rt = \cod trt = \cod t = \cod'$, it would suffices to find a map $\eta_1$ such that
  $\lambda' = \lambda'rt \circ \eta_1$, since $\eta_2 = 1_{\cod'}$ then makes the above square commute. Since
  $r$ is a $p$-cartesian 1-cell, $\lambda'rt$ is a $p$-cartesian 2-cell, whereby there exists a unique universal
  map $\gamma : \dom' \to \dom'rt$ induced by the cone $(\cod', \dom', \lambda', 1_{p\dom'})$ with $p\gamma =
  1_{p\dom'}$ and $\lambda' = \lambda'rt \circ \gamma$, so we may take $\eta_1 = \gamma$, and then $\eta$ is
  the universal map induced by the square
  \[\begin{codi}[tetragonal=base 6em height 4.5em]
    \obj {\dom' & \dom'rt \\ \cod' & \cod'rt \\};
    \mor       (dom ') \gamma:-> (dom 'rt) \lambda'rt:-> (cod 'rt);
    \mor[swap] (dom ') \lambda':-> (cod ') 1_{\cod'}:-> (cod 'rt);
  \end{codi}\]
  To verify that $\eta : 1_{A^\to} \to rt$ is indeed the unit of an adjunction $t \dashv r$, we must verify that
  $t\eta = 1_t$ and $\eta r = 1_r$. For the former, it suffices to show that $\dom t\eta = 1_{\dom t}$ and
  $\cod t\eta = 1_{\cod t}$. But these hold since
  \[\dom t\eta = p\dom'\eta = p\gamma = 1_{p\dom'} = 1_{\dom t} \qquad\text{ and }\qquad
  \cod t\eta = \cod'\eta = 1_{\cod'} = 1_{\cod t}.\]
  Hence $t\eta = 1_t$. For the latter equation $\eta r = 1_r$, we must show that $\gamma r = 1_{\dom' r}$,
  for which it suffices to see that $1_{\dom' r}$ and $\gamma r$ both are universal morphisms for the same
  $p$-cartesian cone over $\lambda'r : \dom' r \to \cod' r : B\comma p \to A$. The cone we consider is
  specifically $(1_{B \comma p}, \dom'r, \lambda'r, 1_{p\dom'r})$ with apex $B \comma p$, which clearly
  satisfies $p\lambda'r = p\lambda'r \circ 1_{p\dom'}$. Furthermore, we clearly see that $1_{\dom' r}$ is
  a universal map over the cone, and that $p\gamma r = 1_{p\dom' r}$ and $\lambda'r \circ \gamma r =
  \lambda'rtr \circ \gamma r = (\lambda'rt \circ \gamma)r = \lambda'r$, so $\gamma r$ is another such
  universal map. Hence $\gamma r = 1_{\dom' r}$, from which we conclude that $\eta r = 1_r$. Hence
  $\eta$ is the unit of an adjunction $t \dashv r$ with identity counit.

  For $(4) \implies (2)$, we must show that the right adjoint $r : B \comma p \to A^\to$ is $p$-cartesian, so let
  $(g,a,\alpha,\beta)$ be a $p$-cartesian cone over $\lambda' r : \dom' r \to \cod' r$ with apex $Y$. Thus
  $g : Y \to B \comma p$, $a : Y \to A$, $\alpha : a \to \cod'r g$, and $\beta : pa \to p\dom'rg = \dom trg
  = \dom g$.
\end{proof}

\begin{notation}
  For any fibration $p : A \to B$ with a chosen lifting $\epsilon : \ell \to \cod$ with $\ell : B \comma p \to A$
  as in (2), and any object $X$ and morphisms $a : X \to A$, $b : X \to B$, and $\sigma : b \to pa$, we write
  $A[\sigma] \coloneq \ell\bar\sigma$, where $\bar \sigma : X \to B \comma p$ is the map induced by $\sigma$.
\end{notation}

\ifSubfilesClassLoaded{\printbibliography}{}
  \end{document}
