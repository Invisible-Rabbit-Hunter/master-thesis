\documentclass[../thesis.tex]{subfiles}

\ifSubfilesClassLoaded{
  \externaldocument{../build/2-preliminaries}%
  \externaldocument{../build/3-logics}%
}

\begin{document}
\chapter{Conclusion}\label{ch:conclusion}
The goal of this thesis was to show how several of the categorical models of dependent type theory can be captured in terms
of some sort of 2-dimensional algebraic theory. Given that many of these models can be phrased in terms of fibred categories,
and fibrations are specific instances of finite 2-limits, it seemed like a good notion of 2-dimensional theory in this case
was that of finite 2-limit theories, generalizing the finite limit theories of categorical universal algebra. To further
make the homomorphisms of models to coincide with those already in literature, we took the step from $\Cat$-enriched
finite limit theories to $\sF$-enriched ones, following \cite{arkor2024}, as these provide fine control over which operations
of the theory are preserved strictly, and which are preserved only up to isomorphism. 

After recounting the basics of $\sF$-enriched limit theories we required, we showed how both comprehension categories and
natural models were describable in the framework. We furthermore showed how to model the basic type constructors, following
the general methods of \cite{coraglia2024a}, though rather than working in $\Cat$ directly, we were constructing the limits
in abstract finite limit $\sF$-sketches. We ended by showing how this framework can capture the multimode type theory of \cite{
gratzer2021}, which distinguishes the framework from the categories with representable maps of \cite{uemura2023} and judgemental
theories of \cite{coraglia2024a}. 

\begin{itemize}
  \item  While we somewhat compared finite finite limit $\sF$-theories with similar frameworks, such as categories with
    representable maps \cite{uemura2023} and judgemental theories \cite{coraglia2024a}, the comparisons have been superficial,
    and mostly focused on the case of multimode type theory. However, we have not explored how the different frameworks are
    related.

    For example, given that the intended interpretation for categories with representable maps is given in terms of discrete
    fibrations, one can intuitively construct, given any such category, a finite limit $\sF$-theory, such that their models
    coincide: simply consider the sketch embeds the given category with representable maps, such every map into the terminal
    object is made into a discrete fibration and any representable map is made into a left-adjoint. Of course, that the models
    of the induced $\sF$-theory coincide with the models of the category with representable maps needs to be checked more
    carefully, and one might naturally ask whether this assignment is functorial, and if so, whether this functor from
    representable map categories to $\sF$-theories commutes with the models.

    Similarly, one might expect that the models of any $\sF$-theory describing a type theory yields a judgemental theory.
    This question can be further extended to asking whether the models of \emph{any} $\sF$-theory yields a judgemental theory.
    Following this comparison with judgemental theories, one discovers a deficiency with $\sF$-theories: they do not allow for
    modeling contraviant operations. Similarly, it is impossible (to my knowledge) to model fibred Heyting algebras, which means
    that we cannot easily construct e.g.\ a theory corresponding to Lawvere's hyperdoctrines when compared to judgemental theories,
    something that is used by \cite[§4]{coraglia2024a} to show that first-order logic falls into the framework of judgemental
    theories.

  \item We have in this thesis only really made use of the basic features of $\sF$-theories and 2-dimensional theories in
    general. In particular, while we have shown that type theories induce $\sF$-theories, we have not made use of this fact
    to study their categories of models. This is closely connected with the 2-dimensional generalizations of locally finitely
    presentable and accessible categories, which are studied in \cite{bourke2021, diliberti2024}. 

  \item In \cref{sec:nm/equiv attr} we showed how the classical equivalence between natural models and categories with attributes
    adapts to the setting of $\sF$-theories. A more general version of the theorem is proven in \cite{coraglia2024b}, wherein
    a generalized notion of categories with families, allowing for general Grothendieck fibrations of types and terms, rather
    than only discrete ones, is proven equivalent to general comprehension categories. It would be interesting to see how their
    proof adapts to the setting of $\sF$-theories, and whether one thus can show that theory of generalized categories with
    families is equivalent to the one for comprehension categories, in a way compatible with the semantics. This would partially
    simplify the proof, in the sense that you get equivalence of the 2-categories of models ``for free'', whether the morphisms
    are strict, pseudo, lax, or even colax.
  %
  % \item Taking inspiration from the heavy use of finite 2-limits of $\sF$-theories, one might wonder whether there is a nice
  %   account of inductive types to be found, wherein we take the 
\end{itemize}

\ifSubfilesClassLoaded{\printbibliography}{}
\end{document}
