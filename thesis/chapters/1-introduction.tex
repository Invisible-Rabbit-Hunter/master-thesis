\documentclass[../thesis.tex]{subfiles}

\begin{document}
\chapter{Introduction}\label{ch:intro}

Since the introduction of dependent type theories, a plethora of categorical semantics have been considered. A good overview of
the landscape of these different semantics and their relations to each other can be found in \cite{ahrens2024}, but to paint a
picture of their multitude, we include small sample of them here:
\begin{itemize}
  \item \textcite{cartmell1984} introduced the more algebraic notions of categories with attributes (unpublished, but
    elaborated on by \textcite{moggi1991,pitts2001}).
  \item \textcite{jacobs1993} introduced comprehension categories as a minimal structure for studying context extension
    and comprehension, encompassing both categories with attributes and other notions of comprehension in logic, such as
    subset comprehension as found in toposes.
  \item \textcite{dybjer1996} gave an alternative formulation of categories with attributes, dubbed categories with
    families, making terms a core feature rather than a derived notion. These were later again reformulated by
    \textcite{awodey2017,fiore2012} in terms of representable maps of presheaves \stackstag{0023}. This reformulation
    was named natural models by \textcite{awodey2017}.
\end{itemize}
Furthermore, all of these semantics present only the most barebones of type theories, with only a notion of contexts, types,
sometimes terms, and context extension. We can of course extend these semantics to account for e.g. dependent product types,
but this is generally adhoc, with no framework to combine and compare different extensions.

In order to rectify the situation, \textcite{uemura2023} proposed a functorial semantics based on the natural models of
\cite{awodey2017}, wherein a type theory is taken to be a finitely complete 1-category equipped with a class of special maps,
dubbed representable maps, following the tradition of the functorial semantics introduced by \cite{lawvere1963}. The canonical
example of a representable maps of is a category of presheaves equipped with its class of representable maps, and a model of a
type theory as in \cite{uemura2023} is a base category $\cC$ of contexts equipped with a functor from the type theory to $\cC$
preserving the structure of categories with representable maps. Considering the representable map category generated by a single
non-trivial representable map, we then recover as its models exactly the natural models of \cite{awodey2017}. However, while this
semantics allows to capture many type theories that have been considered, it does not capture all the ones of interest. In
particular, it does not capture multimode type theory (MTT) as introduced by \textcite{gratzer2021}, which garnered a lot of
interest in recent times, in part due to its uniform treatment of multiple modalities. Since MTT essentially consists of several
type theories with maps between them corresponding to the modalities, and so semantically to several models of type theory, each
with its own category of contexts, it does not easily fit into the framework of \cite{uemura2023}, wherein a model has only one
category of contexts.

Another approach to a more general understanding of type theories, and in fact deductive systems in general, can be found in
\cite{coraglia2024a}. Here \textcite{coraglia2024a} introduce the notion of a \emph{judgemental theory}, consisting of a base
category of contexts $\ctx$, a family of \emph{judgements} $\cJ$ consisting of functors $\cF : F \to \ctx$,
a family of \emph{rules} $\cR$ consisting of arbitrary functors, and a family \emph{policies}, 2-dimensional filllers
of triangles
\[\begin{codi}[hexagonal]
  \obj{A & B \\ & C \\};
  \mor A {{\lambda \in \cR}}:-> B {{\cG \in \cJ \cup \cR}}:-> C;
  \mor[swap] A {{\cF \in \cJ \cup \cR}}:-> C;
\end{codi}\]
Of course these classes have to satisfy some closure conditions and other properties. Now a natural model is a judgemental
theory containing judgements, rules, and policies such that we have a diagram in $\Cat$ of the form
\[\begin{codi}[hexagonal=horizontal side 6em angle 45] 
  \obj{\dot\cU &   \cU \\ & |(cC)| \ctx \\};
  \mor  (dot cU) ["\typ",swap]:-> cU u:-> cC;
  \mor[swap] (dot cU) \dot u:-> cC;
  \mor[swap] cU \var:[bend right,->] (dot cU);
  \mor typ [shorten=0.3em, |-] var;
\end{codi}\]
where the inner triangle commutes strictly and $u,\dot u$ are discrete fibrations. Here we use that a map of discrete
fibrations corresponds to a representable map of presheaves if and only if it has a right adjoint, which was known already
in \cite{awodey2017}. In \cite{coraglia2024a} they further show how these judgemental theories can give rise to deductive
systems, and how the generated deductive syntax corresponding to these natural models is essentially the standard syntax of
type theory. Finally they show how to model certain type constructors in this framework, which differs slightly from, and is
simpler than, the approach via polynomial functors found in \cite{awodey2017,fiore2012,uemura2023}. However, similarly to
categories with representable maps, judgemental theories are defined with a specified base category of contexts, and so does
not clearly capture multimode type theories. It must be noted, though, that the authors of \cite{coraglia2024a} remark that
there is no inherent reason, other than tradition, that a special category of contexts is singled out, and that it in fact
makes their theory less clean than one might have wished.

With both of these previous approaches in mind, it seems natural that we should consider a model of type theory precisely
as some sort of 2-dimensional diagram in $\Cat$, since both \textcite{uemura2023} and \textcite{coraglia2024a} used the
same formulation of natural models in terms of discrete fibrations. Such a digram in $\Cat$ is the same data as a 2-functor
$D : \cT \to \Cat$, and since we want to consider not arbitrary diagrams, but diagrams satisfying certain properties, we
require some additional conditions. In particular, we see that $\cT$, the shape of the diagram, needs to be of the form
\[\begin{codi}[hexagonal=horizontal side 6em angle 45] 
  \obj{\dot\cU &   \cU \\ & |(cC)| \cC \\};
  \mor  (dot cU) ["\typ",swap]:-> cU u:-> cC;
  \mor[swap] (dot cU) \dot u:-> cC;
  \mor[swap] cU \var:[bend right,->] (dot cU);
  \mor typ [shorten=0.3em, |-] var;
\end{codi}\]
Since adjunctions in 2-categories are preserved by any 2-functor, we need no restriction on $D$ to make $D(\typ)$ a
representable map of discrete fibrations. However, we do need to require that $u$ and $\dot u$ are discrete fibrations.
It is shown in \cite{street1974} that we can define (Grothendieck) fibrations inside general 2-categories, and furthermore
that it can be done in terms of finite 2-limits. The same extends to discrete fibrations. So if we ask $\cT$ to have enough
finite 2-limits to specify that $u$ and $\dot u$ be discrete fibrations internal to $\cT$, and then ask $D$ to preserve those
finite 2-limits, then we have exactly that the image of $D$ consists of a natural model in the sense of \textcite{awodey2017},
up to the Grothendieck construction. This idea, that a type theory is merely a certain a 2-dimensional theory in the same vein
of Lawvere theories, is the heart of this thesis.

Of course, to talk about type theories as 2-dimensional theories, we first need a good notion of such theories. Many different,
but related, notions of 2-dimensional have been previously suggested:
\begin{itemize}
  \item \textcite{gray1973} takes a straightforward generalization of Lawvere theories to the 2-dimensional setting, defining
    a 2-algebraic theory to be a 2-category $\cA$ equipped with a finite product preserving 2-functor from the (locally
    discrete) 2-category of finite sets to $\cA$, and a model of the theory to be a finite product preserving 2-functor
    $\cA \to \Cat$. These 2-algebraic theories allow for encoding 2-categorical structures such as monoidal and cartesian
    categories, but since we only ask for preservation of products, we lack more complex limits, like comma objects, powers
    by categories, and pullbacks. Furthermore, we note that this notion is inherently single-sorted, though this is easily
    fixed by only asking that the theory be a 2-category with finite products, without requiring the identity-on-objects
    and finite product preserving 2-functor.

  \item \textcite{nishizawa2009} generalize Lawvere theories to be over finitely presentable enriched categories, rather than
    just set. Using their framework, one would define a 2-Lawvere theory to be a 2-category $\cA$ equipped with an identity
    on objects and finite 2-limit preserving functor $\Cat_f \to \cA$, where $\Cat_f$ is the full sub-2-category of $\Cat$
    consisting of the finitely presentable categories. These theories allow for more complex constructions than the previous,
    for example supporting a theory of categories with pullbacks and terminal objects. Moreover, while the theories specialized
    to $\Cat$ are single-sorted, another choice of finitely presentable category to take the theories over, such as $\Cat^n$
    for some finite cardinal $n$, would allow for $n$ distinct sorts.

  \item In a similar vein to the previous example, \textcite{kelly1982a} presents enriched finite limit theories, for which
    we again consider the special case of enrichment in $\Cat$. Here a theory is a 2-category with all finite 2-limits, and
    a model is a finite 2-limit preserving 2-functor into $\Cat$. Note, however, that the homomorphisms for such theories are
    inherently strict, and so for example the theory of monoidal categories has as homomorphisms those preserving the monoidal
    product strictly. Of course, this can be amended by asking for the homomorphisms to preserve the structure only up to
    isomorphism, but that must be treated with some care.

  % \item \todo{Mention Di Liberti and Osmond paper ``Bi-accessible and bipresentable 2-categories''}

  \item Finally, we mention the $\sF$-theories, which are finite limit theories in the sense of \cite{kelly1982a} in which
    the category $\sF$ of full embeddings of categories (see \cref{def:F}) is taken as the base of enrichment. These were
    introduced by \textcite{arkor2024} in terms of $\sF$-sketches, and any $\sF$-sketch generates a $\sF$-theory in this
    sense with the same (strict) models. The motivation for their introduction was to amend the notion of finite 2-limit
    sketches and theories in the sense of \cite{kelly1982a} to allow for fine control over which operations of the models
    are preserved strictly, and which are preserved only up to isomorphism. The prime example for a theory in which you
    want certain parts to preserved only up to isomorphism.
\end{itemize}

For our purposes, since we would like to not only recover the models of type theory, but also their morphisms, it turns out
that the last of these options is the most easily applicable. For example, we wish that the pseudomorphisms of our natural
models be pairs of maps of discrete fibrations preserving the structure up to isomorphism, as described in \cite[Chapter 2]
{newstead2021}. Since maps of discrete fibrations are strictly commuting squares, this requires our theories to provide
fine control over which parts of the structure is strictly preserved by homomorphisms, and which part is only weakly
preserved, which is exactly what $\sF$-theories provide.

The structure of the thesis is as follows:
\begin{itemize}
  \item In \cref{ch:prelims} we give some background definitions and theorems. We begin by recalling the notion of finite limit
    theories and sketches, as these serve as the ideological basis for the $\sF$-theories and sketches considered in
    \cref{ch:F-theories}. Then, for similar reasons, we cover some of the basic parts of the theory of categories enriched
    over cartesian closed categories, as well as their corresponding notion of their limits. We give special attention to
    the case of $\Cat$-enriched categories, better known as 2-categories, as well as providing an explicit description of
    2-limits. Finally, we cover the notion of cartesian morphisms and fibrations internal to 2-categories, and in particular
    their characterisation in terms of finite 2-limits.

  \item \Cref{ch:F-theories} first covers the basic features of $\sF$-enriched categories and their limits. Then we introduce
    $\sF$-theories and $\sF$-sketches as follows from the general theory of \cite{kelly1982a}, as well as the notions of models
    of $\sF$-theories and $\sF$-sketches and their properties as given \cite{arkor2024}. We end the chapter with some examples
    of $\sF$-theories, given in terms of $\sF$-sketches, to give a feel for how we will construct and deal with $\sF$-theories
    in the subsequent chapter.

  \item The main part of the thesis is \cref{ch:type theories}, wherein we show how comprehension categories \cite{jacobs1993},
    categories with attributes \cite{cartmell1978} (phrased in terms of discrete comprehension categories, see \cite{jacobs1993}),
    and categories with families \cite{dybjer1996} (phrased in terms of natural models \cite{awodey2017}) are models of their
    respective $\sF$-theories. We furthermore show that the theories for categories with attributes and categories with families
    are equivalent as $\sF$-categories, thereby recovering the standard result that categories with attributes are indeed
    equivalent to categories with families. Furthermore, this provides an equivalnce of 2-categories of their respective models,
    whether one considers strict or pseudo homomorphisms. Next we show how to extend the theory of categories with families to
    model type theories with type formers such as dependent product types, dependent sum types, and extensional identity types,
    whose description turns out to be almost verbatim the same as is found in \cite{coraglia2024a}, with the only difference that
    we work in a more formal setting, replacing the setting of $\Cat$ with that of an abstract finite complete $\sF$-category.
    \todo{Decide on whether to include inductive types or not}%
    Finally we show how we can capture more complex type theories using this framework, focusing on the multimode type theory
    of \textcite{gratzer2021}, which neither the categories with representable maps of \cite{uemura2023} nor the judgemental
    theories of \cite{coraglia2024a} are easily shown to model.
\end{itemize}

% \par\noindent\rule{\textwidth}{0.4pt}

% In more recent times, there has been a desire to find a functorial semantics for type theory, as can be seen by the work
% of \textcite{uemura2023}. In \emph{op.\ cit.\ }, the notion of type theory is taken to be a 1-category with additional
% structure, much as one one view an algebraic theory as a 1-category with finite products. The additional structure includes
% a class of morphisms, dubbed the representable maps, closed under certain constructions and satisfying certain properties,
% the canonical examples of which are presheaf categories with representable maps those as defined in \textcite[\stacksref{0023}]
% {stacks-project}. Models of a type theory $\bT$ are categories $\cC$ equipped with a functor $\bT \to \hat\cC$ taking
% representable maps to then functors preserving the struc and models
% are taken to be categories equipped with as categories equipped with
% additional discrete fibrations


% The goal of this thesis was to fill in a piece of folk-lore in the semantics of type theory: that the semantics of type theory
% can be described in terms of finite 2-dimensional limit theories. It is of course well known that, for example, categories with
% families, as introduced by \citeauthor{dybjer1996} in \cite{dybjer1996}, are essentially algebraic, and so are models of a
% 1-dimensional finite limit theory. This was, in fact, one of the main motivations for their introduction, and contrasts it
% with comprehension categories \cite{jacobs1993}. However, many times when morphisms of such models are considered, the
% homomorphisms between models of the finite limit theories are too strict. Instead one takes some notion of pseudo-homomorphism,
% preserving structure such as context extension. Taking this perspective, viewing the models as structured categories rather
% than structured sets, we see the need for 2-dimensional theories, taking models in $\Cat$ and homomorphisms those which
% preserve (some) structure only up to isomorphism. 
%
%
%
% One approach to the semantics of type theory in this vein is the categories with represntable maps of \textcite{uemura2023},
% wherein a type theory is a 1-category with a terminal object equipped with a class of maps, called the representable maps,
% and models are taken in discrete fibrations. 
%
% This is, in essence, the approach taken in \cite{coraglia2024a}, where they introduce judgemental theories as a means to
% model the 2-dimensional structure of judgements, inference rules, and deductions, with examples taken from type theory in
% particular. These judgemental theories consist of two classes of functors, the judgements and the rules, as well as a
% class of lax triangles connecting them, such that these clases obey certain axioms. They further show how any natural model
% of type theory, as defined in \cite{awodey2017,fiore2012}, induce a judgemental theory, and how adding additional inference
% rules and triangles allows to capture certain type formers as well. 
%
% Following the idea that models of type theories are certain diagrams in $\Cat$, and noting that these models, e.g. comprehension
% categories, can be captured in entirely in terms of finite 2-limits, naturally leads us to consider these models to be
% exactly the image of a fintie 2-limit preserving functor from some abstract model. In other words, we may take the perspective
% that semantics of type theory should be nothing more than certain finite 2-limit theories, and it is this idea that is the
% foundation of the thesis. 
%
% It should be mentioned that this  
%
% \todo[inline]{Mention related work and give an overview on the topic of type theory as much as it is related.}
% \todo[inline]{Mention at least \citeauthor{uemura2023} \cite{uemura2023}}
% \todo[inline]{Comment from Ivan: I would mention many things: Uemura, Awodey, Jacobs, Bourke et al, Coraglia et
% al, even some works by Power (Why tricategories?). Mimram and 2-dimensional Lawvere theories.}
% \todo[inline]{Comment from Ivan: You could start with a panorama on the topic of functorial semantics of dependent
%   type theory, landing on the contribution of Uemura. Then say that in the last years
%   some 2-dimensional aspects are emerging, saying that \cite{coraglia2024a} was the first paper
%   highling these aspects. This could be a good place to mention Bourke, and Power,
%   and also a good place to say that the community of rewriting is coming to a similar
% conclusion indepedendently (Mimram).}
%
% Since the introduction of Lawvere theories in \cite{lawvere1963}, there has been much work 
% capture different logics as certain structured categories, and models as functors preserving this structure.
% For example, 
%
% This thesis was inspired by the judgemental theories of \cite{coraglia2024a}. Judgemental theories consist of
% a category (the category of contexts), two collections of functors, the judgements and the rules, and a
% collection of \emph{policies}, which are 2-cells satisfying a certain property. The first thing to note
% is that, really, there is no greater distinction between the judgements and the rules. In particular, there
% is no reason to single out a singular category of contexts, which was already noted as a potential
% generalisation in \emph{op.\ cit.} Removing the distinction between rules and judgements completely, we
% can view a theory as a \emph{strict} 2-functor to $\Cat$, with the domain being the free 2-category with
% the rules and policies. The final step is to note that much of the structure, such as fibrations, we desire
% from judgemental theories can be captured by finite 2-limits. Restricting to only constructions involving
% finite 2-limits, we can thus simplify the notion of theory to be some notion of 2-dimensional finite limit
% theory.
%
% \begin{remark}
%   There is a stark difference between 2-dimensional finite limit theories and judgemental theories. The finite
%   limit theories behave more like a class of judgemental theories, and the judgemental theories behave more
%   like the models of finite limit theories. For example, in \cite{coraglia2024a} they specify a class of
%   judgemental theories, the plain dependent type theories, which we will show are the models of a specific
%   finite 2-limit theory. 
% \end{remark}
%
% There have been several 2-dimensional generalisations of finite limit theories and finite limit sketches. In
% the strictest sense, we may consider the enriched finitary essentially algebraic theories and sketches of
% \cite{kelly1982b} in the special case of enrichment over $\Cat$. However, while these allow to describe the
% \emph{structures} we're after, the homomorphisms are generally too strict. In this framework, the theory of
% cartesian categories has as models (in $\Cat$) categories with chosen finite products, but the homomorphisms
% must preserve these products on the nose, rather than only up to isomorphism. 
%
% \todo[inline]{Figure out how to tie in type theory. In my mind, this is harder than I expected, since while
%   this thesis is about describing different models of type theory in terms of finite 2-limit theories, it feels
% more like an application of the framework and my understanding of it.} 
%
% \todo[inline]{Give an overview of the thesis, touching on the various topics I will introduce, represent, and use.}
% A timeline for the history of categorical semantics for type dependency:
% \begin{itemize}
%   \item[1984] \citeauthor{cartmell1984}'s \citetitle{cartmell1984} 
%   \item[1993] \citeauthor{jacobs1993}'s \citetitle{jacobs1993} 
% \end{itemize}
%
% For 2-dimensional theories:
% \begin{itemize}
%   \item 2-Lawvere theories \cite{}.
% \end{itemize}

\ifSubfilesClassLoaded{\printbibliography}{}%
\end{document}
