\documentclass[../thesis.tex]{subfiles}

\begin{document}
  \chapter{Introduction}
  
  

  Since the introduction of Lawvere theories in \cite{lawvere1963}, there has been much work 
  capture different logics as certain structured categories, and models as functors preserving this structure.
  For example, 

  This thesis was inspired by the judgemental theories of \cite{coraglia2024a}. Judgemental theories consist of
  a category (the category of contexts), two collections of functors, the judgements and the rules, and a
  collection of \emph{policies}, which are 2-cells satisfying a certain property. The first thing to note
  is that, really, there is no greater distinction between the judgements and the rules. In particular, there
  is no reason to single out a singular category of contexts, which was already noted as a potential
  generalisation in \emph{op.\ cit.} Removing the distinction between rules and judgements completely, we
  can view a theory as a \emph{strict} 2-functor to $\Cat$, with the domain being the free 2-category with
  the rules and policies. The final step is to note that much of the structure, such as fibrations, we desire
  from judgemental theories can be captured by finite 2-limits. Restricting to only constructions involving
  finite 2-limits, we can thus simplify the notion of theory to be some notion of 2-dimensional finite limit
  theory.

  \begin{remark}
    There is a stark difference between 2-dimensional finite limit theories and judgemental theories. The finite
    limit theories behave more like a class of judgemental theories, and the judgemental theories behave more
    like the models of finite limit theories. For example, in \cite{coraglia2024a} they specify a class of
    judgemental theories, the plain dependent type theories, which we will show are the models of a specific
    finite 2-limit theory. 
  \end{remark}

  There have been several 2-dimensional generalisations of finite limit theories and finite limit sketches. In
  the strictest sense, we may consider the enriched finitary essentially algebraic theories and sketches of
  \cite{kelly1982b} in the special case of enrichment over $\Cat$. However, while these allow to describe the
  \emph{structures} we're after, the homomorphisms are generally too strict. In this framework, the theory of
  cartesian categories has as models (in $\Cat$) categories with chosen finite products, but the homomorphisms
  must preserve these products on the nose, rather than only up to isomorphism. 
  
  \todo[inline]{Figure out how to tie in type theory. In my mind, this is harder than I expected, since while
  this thesis is about describing different models of type theory in terms of finite 2-limit theories, it feels
  more like an application of the framework and my understanding of it.} 
  % A timeline for the history of categorical semantics for type dependency:
  % \begin{itemize}
  %   \item[1984] \citeauthor{cartmell1984}'s \citetitle{cartmell1984} 
  %   \item[1993] \citeauthor{jacobs1993}'s \citetitle{jacobs1993} 
  % \end{itemize}
  %
  % For 2-dimensional theories:
  % \begin{itemize}
  %   \item 2-Lawvere theories \cite{}.
  % \end{itemize}
\end{document}
